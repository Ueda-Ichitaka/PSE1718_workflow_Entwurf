\chapter{Einleitung}
%\thispagestyle{plain}
    Dieses Entwurfsdokument dient der Beschreibung der Architektur des Workflow Systems.
    Neben den Paket-, Klassen- und Methodenbeschreibungen enthält das vorliegende Dokument verschiedene Arten von UML %\Gls{UML} 
    Diagrammen die einzelne Abläufe anschaulicher darstellen. \newline
    Das Komponentendiagramm zeigt die Struktur des modellierten Systems und die Schnittstellen der Komponenten. \newline
    Das Entity Relationship (ER) Diagramm zeigt, wie die Daten in einer relationalen Datenbank verwaltet werden.\newline
    Die Sequenzdiagramme zeigen den zeitlichen Ablauf typischer Vorgänge.\newline 
    Im Anhang finden Sie ein großformatiges Klassendiagramm, wobei die Klassen der Übersichtlichkeit halber farblich markiert sind.
    \newline
    \section{Entwurfsziele}
    In der Enwurfsphase wurde auf die folgende Aspekte besonders viel Wert gelegt. 
        \begin{itemize}
            \item Erweitbarkeit
                
            \item Veränderbarkeit
            \item Moderne Entwicklungsinstrumente
            \item Zuverlässigkeit
            \item Nutzerfreundlichkeit \newline
                Die grafische Benutzeroberfläche (im folgenden GUI genannt) soll möglichst intuituv sein. Bei auftretenden Fehlern oder ungültigen Eingaben soll dem Benutzer ein entsprechender Hinweis angezeigt und die Möglichkeit einer erneuten Eingabe angeboten werden.
        \end{itemize}
    \section{Hello}
        \lipsum[20]
        
    \section{World}
        \lipsum[21]
        
    \section{Foo}
        
        % \clearpage
        % %%% -----------------------------------------------
        % \KOMAoptions{paper=a3,paper=landscape}
        % %%% -----------------------------------------------
        % \areaset{\dimexpr \textwidth+.5\paperwidth}{\textheight}
        % \noindent\begin{minipage}{\textwidth}
        %     \rule{\textwidth}{\dimexpr\textheight-3\baselineskip\relax}
        %     \captionof{figure}{Eine ziemlich breite Abbildung}
        % \end{minipage}

        % %\includepdf[noautoscale=true, fitpaper=true,pagecommand={\rhead{\includegraphics[height=1cm]{logo.png}}\lfoot{}\cfoot{}\rfoot{\thepage}}]{diagrams/test.pdf}
        
        % \clearpage
        % \KOMAoptions{paper=a4,paper=portrait}
        % \areaset{\dimexpr \textwidth-\paperwidth}{\textheight}
        
        % wieder a4
        \lipsum[21]
        
    \section{Bar}
        \lipsum[45]
        