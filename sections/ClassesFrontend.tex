    \section{Client Klassen}
    
        \subsection{services}
    
        \subsection{models}
    
    		\subsubsection{Workflow}
    		
    		Attribute:
                \begin{center}
                	\renewcommand{\arraystretch}{1.5}
    	            \setlength\tabcolsep{5pt}
                	\begin{tabularx}{\textwidth}{|l|l|X|}
                		\hline
                        \rowcolor[gray]{0.75}[4.85pt]	
                	    Typ & Name & Beschreibung \\ \hline
                		number & id &  \\ \hline
                		string & name &  \\ \hline
                		boolean & shared &  \\ \hline
                		string & description &  \\ \hline
                		number & creator_id &  \\ \hline
                		WorkflowVertex[] & vertices &  \\ \hline
                		WorkflowEdge[] & edges &  \\ \hline
                		number & created_timestamp &  \\ \hline
                		number & updated_timestamp &  \\ \hline
                	\end{tabularx}
                \end{center}
                
    		\subsubsection{WorkflowVertex}
    		
    		Attribute:
                \begin{center}
                	\renewcommand{\arraystretch}{1.5}
    	            \setlength\tabcolsep{5pt}
                	\begin{tabularx}{\textwidth}{|l|l|X|}
                		\hline
                        \rowcolor[gray]{0.75}[4.85pt]
                	    Typ & Name & Beschreibung \\ \hline
                		number & id &  \\ \hline
                		number & process_id &  \\ \hline
                		number & x &  \\ \hline
                		number & y &  \\ \hline
                		WorkflowVertexStatus & status &  \\ \hline
                		WorkflowVertex[] & vertices &  \\ \hline
                		WorkflowEdge[] & edges &  \\ \hline
                		WorkflowArtefact<'input'> & input_artefacts &  \\ \hline
                		WorkflowArtefact<'output'> & output_artefacts &  \\ \hline
                	\end{tabularx}
                \end{center}
                
    		\subsubsection{WorkflowEdge}
    		
    		Attribute:
                \begin{center}
                	\renewcommand{\arraystretch}{1.5}
    	            \setlength\tabcolsep{5pt}
                	\begin{tabularx}{\textwidth}{|l|l|X|}
                		\hline
                        \rowcolor[gray]{0.75}[4.85pt]
                	    Typ & Name & Beschreibung \\ \hline
                		number & id &  \\ \hline
                		WorkflowVertex & a &  \\ \hline
                		WorkflowVertex & b &  \\ \hline
                		number & input_id &  \\ \hline
                		number & output_id &  \\ \hline
                	\end{tabularx}
                \end{center}
                
    		\subsubsection{WorkflowArtefact<T>}
    		
    		Attribute:
                \begin{center}
                	\renewcommand{\arraystretch}{1.5}
    	            \setlength\tabcolsep{5pt}
                	\begin{tabularx}{\textwidth}{|l|l|X|}
                		\hline
                        \rowcolor[gray]{0.75}[4.85pt]
                	    Typ & Name & Beschreibung \\ \hline
                		number & id &  \\ \hline
                		number & workflow_id &  \\ \hline
                		number & paramerer_id &  \\ \hline
                		T & parameter_role &  \\ \hline
                		string & format &  \\ \hline
                		string & data &  \\ \hline
                		number & created_timestamp &  \\ \hline
                		number & updated_timestamp &  \\ \hline
                	\end{tabularx}
                \end{center}
                
    		\subsubsection{WorkflowVertexState}
    		
    		ENUM
                \begin{center}
                	\renewcommand{\arraystretch}{1.5}
    	            \setlength\tabcolsep{5pt}
                	\begin{tabularx}{\textwidth}{|l|X|}
                		\hline
                        \rowcolor[gray]{0.75}[4.85pt]
                	    Name & Beschreibung \\ \hline
                		WAITING &   \\ \hline
                		RUNNING &   \\ \hline
                		FINISHED  &  \\ \hline
                		FAILED  &  \\ \hline
                		DEPRECATED  &  \\ \hline
                	\end{tabularx}
                \end{center}
    
        \section{components}