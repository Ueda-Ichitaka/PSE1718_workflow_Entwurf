\chapter{Klassen und Pakete}

    % table Muster Attribute
    %
%     \begin{center}
%     	\renewcommand{\arraystretch}{1.5}
%         \setlength\tabcolsep{5pt}
%     	\begin{tabularx}{\textwidth}{|l|l|X|}
%     		\hline
%             \rowcolor[gray]{0.75}[4.85pt]            		
%     	    Name & Datentyp & Beschreibung \\ \hline
%            
%           \hline            		
%     	\end{tabularx}
%     \end{center}


        % table Muster Methoden
% 		\begin{center}
% 		    \setlength\tabcolsep{5pt}
%         	\renewcommand{\arraystretch}{1.5}
%             	\begin{tabularx}{\textwidth}{|l|l|l|X|}
%             	\hline
%             	\rowcolor[gray]{0.75}[4.85pt]
%         		Name & Rückgabetyp & Parameter & Beschreibung \\ \hline 
%
%                \hline
%             	\end{tabularx}
% 		\end{center}

    \section{Server Klassen}

    \subsection{edu.kit.scc.pseworkflow.views}

		\subsubsection{WorkflowView}

        Attribute:
        \begin{center}
        	\renewcommand{\arraystretch}{1.5}
            \setlength\tabcolsep{5pt}
        	\begin{tabularx}{\textwidth}{|l|l|X|}
        		\hline
                \rowcolor[gray]{0.75}[4.85pt]            		
        	    Name & Datentyp & Beschreibung \\ \hline
        	    model & Workflow &  \\ \hline		
        	\end{tabularx}
        \end{center}
        
        Methoden:
        \begin{center}
        	\setlength\tabcolsep{5pt}
        	\renewcommand{\arraystretch}{1.5}
        	
        	\begin{tabularx}{\textwidth}{|l|l|l|X|}
        		\hline
        		\rowcolor[gray]{0.75}[4.85pt]
        		Name & Rückgabetyp & Parameter & Beschreibung \\ \hline 
        		index & HTTPResponse & request: HTTPRequest & Gibt zurück HTML aus der Template-Datei. Die enthält Verlinung auf Code, der auf den Client geladen wird und dort ausgeführt wird \\ \hline
        		get & HTTPResponse & request: \thead{HTTPRequest\\id: int} & Sucht in der Datenbank nach dem Workflow mit der übergebenen ID und gibt ihn im JSON-Format zurück \\ \hline
        		create & HTTPResponse & request: HTTPRequest & Erzeugt einen neuen Workflow und gibt seine ID zurück \\ \hline
        		update & HTTPResponse & \thead{request: HTTPRequest\\id: int} & Speichert die Daten in die Datenbank; Überschreibt die Standard Methode. In dieser Methode wird die Workflow enthaltende JSON Datei geparst und danach werden einzelne Elemente eines Workflows separat gespeichert. \\ \hline
        		delete & HTTPResponse & \thead{request: HTTPRequest\\id: int} & Löscht den Workflow mit der übergebenen ID \\ \hline
        		start & HTTPResponse & \thead{request: HTTPRequest\\id: int} & Führt den Workflow mit der übergebenen ID aus \\ \hline
        		stop & HTTPResponse & \thead{request: HTTPRequest\\id: int} & Stoppt die Ausführung des Workflows mit der übergebenen ID \\ \hline
        	\end{tabularx}
        \end{center}

		\subsubsection{EditorView}
		
		Methoden:
		\begin{center}
		    \setlength\tabcolsep{5pt}
        	\renewcommand{\arraystretch}{1.5}
            	\begin{tabularx}{\textwidth}{|l|l|l|X|}
            	\hline
            	\rowcolor[gray]{0.75}[4.85pt]
        		Name & Rückgabetyp & Parameter & Beschreibung \\ \hline 
                index & HTTPResponse & request: HTTPRequest &  \\ \hline
            	\end{tabularx}
		\end{center}
		
		\subsubsection{SettingsView}
		
		Attribute:
        \begin{center}
        	\renewcommand{\arraystretch}{1.5}
            \setlength\tabcolsep{5pt}
        	\begin{tabularx}{\textwidth}{|l|l|X|}
        		\hline
                \rowcolor[gray]{0.75}[4.85pt]            		
        	    Name & Datentyp & Beschreibung \\ \hline
        	    template_name & string & Name der Template-Datei \\ \hline		
        	\end{tabularx}
        \end{center}

		\subsubsection{WorkflowsView}

        Attribute:
        \begin{center}
        	\renewcommand{\arraystretch}{1.5}
            \setlength\tabcolsep{5pt}
        	\begin{tabularx}{\textwidth}{|l|l|X|}
        		\hline
                \rowcolor[gray]{0.75}[4.85pt]            		
        	    Name & Datentyp & Beschreibung \\ \hline
        	    model & Workflow &  \\ \hline
        		template\_name & string & Name der Template Datei \\ \hline				
        	\end{tabularx}
        \end{center}
		
		\subsubsection{UserView}
		
		Methoden:
		\begin{center}
		    \setlength\tabcolsep{5pt}
        	\renewcommand{\arraystretch}{1.5}
            	\begin{tabularx}{\textwidth}{|l|l|l|X|}
            	\hline
            	\rowcolor[gray]{0.75}[4.85pt]
        		Name & Rückgabetyp & Parameter & Beschreibung \\ \hline 
                index & HTTPResponse & request: HTTPRequest & Gibt zurück Accountdaten des eingeloggten Users oder Meldung, dass User nicht eingeloggt ist, sollte das der Fall sein \\ \hline
            	\end{tabularx}
		\end{center}
		
		\subsubsection{ProcessView}
		
		Methoden:
		\begin{center}
		    \setlength\tabcolsep{5pt}
        	\renewcommand{\arraystretch}{1.5}
            	\begin{tabularx}{\textwidth}{|l|l|l|X|}
            	\hline
            	\rowcolor[gray]{0.75}[4.85pt]
        		Name & Rückgabetyp & Parameter & Beschreibung \\ \hline 
                index & HTTPResponse & request: HTTPRequest & Gibt zurück Liste allen verfügbaren Processes \\ \hline
                get & HTTPResponse & \thead{request: HTTPRequest\\id: int} & Gibt zurück Details des Processes mit übergegebenem ID \\ \hline
                create & HTTPResponse & request: HTTPRequest & Erstellt einen neuen Process und gibt seine ID zurück im Response \\ \hline
                update & HTTPResponse & \thead{request: HTTPRequest\\id: int} & Speichert in Request übergegebene Daten im Process mit übergegebenem ID \\ \hline
                delete & HTTPResponse & \thead{request: HTTPRequest\\id: int} & Löscht den Process mit übergegebenem ID \\ \hline
            	\end{tabularx}
		\end{center}
		
		\subsubsection{WPSView}
		
		Methoden:
		\begin{center}
		    \setlength\tabcolsep{5pt}
        	\renewcommand{\arraystretch}{1.5}
            	\begin{tabularx}{\textwidth}{|l|l|l|X|}
            	\hline
            	\rowcolor[gray]{0.75}[4.85pt]
        		Name & Rückgabetyp & Parameter & Beschreibung \\ \hline 
                index & HTTPResponse & request: HTTPRequest & Gibt zurück Liste allen verfügbaren WPS Servern \\ \hline
                get & HTTPResponse & \thead{request: HTTPRequest\\id: int} & Gibt zurück Details des WPS WPS mit übergegebenem ID \\ \hline
                create & HTTPResponse & request: HTTPRequest & Erstellt einen neuen WPS Server und gibt seine ID zurück im Response \\ \hline
                update& HTTPResponse & \thead{request: HTTPRequest\\id: int} & Speichert in Request übergegebene Daten im WPS Server mit übergegebenem ID \\ \hline
                delete & HTTPResponse & \thead{request: HTTPRequest\\id: int} & Löscht den WPS Server mit übergegebenem ID \\ \hline
                refresh & HTTPResponse & \thead{request: HTTPRequest\\id: int} & Lädt von dem WPS Server mit übergegebenem ID Liste aller verfügbaren Processes und aktualisiert Processe in unserer Datenbank, wenn Änderungen vorliegen \\ \hline
            	\end{tabularx}
		\end{center}

    \subsection{edu.kit.scc.pseworkflow.cron}

%%% KLASSE CRON
		\subsubsection{WorkflowJob}

        Die Klasse erledigt Aufgaben, die ständig im Hintergrund laufen und nicht vom Benutzer gesteuert werden. Methoden in dieser Klasse werden in regelmäßigen Zeitabständen von cron aufgerufen (z.B. einmal pro Minute).
        \newline\newline
        Methoden:
        \begin{center}
        	\setlength\tabcolsep{5pt}
        	\renewcommand{\arraystretch}{1.5}
        	
        	\begin{tabularx}{\textwidth}{|l|l|l|X|}
        		\hline
        		\rowcolor[gray]{0.90}[4.85pt]
        		Name & Rückgabetyp & Parameter & Beschreibung \\ \hline
        		check_runnig_tasks & void & - & Fragt für jeden Laufenden Task (status = RUNNING) sein Status bei zugehörigem WPS-Server ab und, falls, der Task erfolgreich beendet wurde (status wurde auf FINISHED gesetzt), setzt ggf. status von nächsten Tasks in diesem Workflow auf READY \\ \hline
        		do & void & - & Geht alle Tasks im Status READY durch und schickt sie an den zugehörigen WPS Server, falls er frei ist \\ \hline
        	\end{tabularx}
        \end{center}

%%% ENDE DER KLASSE CRON

    
%%% MODELS

%%% KLASSE WORKFLOW
        \subsection{edu.kit.scc.pseworkflow.models}
            \subsubsection{Workflow}
	        Diese Klasse beschreibt einen Workflow. Ein Workflow enthält einen oder mehrere Tasks. \newline
	        Diese Klasse erbt von der Django Klasse namens \glqq Model\grqq .
                    
            Attribute:
            \begin{center}
            	\renewcommand{\arraystretch}{1.5}
	            \setlength\tabcolsep{5pt}
            	\begin{tabularx}{\textwidth}{|l|l|X|}
            		\hline
                    \rowcolor[gray]{0.75}[4.85pt]            		
            	    Name & Datentyp & Beschreibung \\ \hline
            		id & int & Primärschlüssel in der Datenbank. Wird von Django automatisch generiert \\ \hline
					identifier & string & Eindeutige Kennung des Workflows\\ \hline
					name & string & Anzeigetitel und Name des Workflows\\ \hline
					description & string & Beschreibungstext des Workflows\\ \hline				
					tasks & Task[] & Liste der Tasks, die zu dem Workflow gehören\\ \hline
					percent\_done & int & Fortschritt des Workflows in Prozent\\ \hline
					created\_at & Date & Erzeugungsdatum des Workflows \\ \hline
					updated\_at & Date & Datum der letzten Veränderung \\ \hline 
					creator & User & Benutzer, der den Workflow erstellt hat\\ \hline
					last\_modifier & User & Benutzer, der am letzten den Workflow verändert hat\\ \hline
            	\end{tabularx}
            \end{center}

        
             Methoden:
	        \begin{center}
	        	\setlength\tabcolsep{5pt}
	        	\renewcommand{\arraystretch}{1.5}
	        	
	        	\begin{tabularx}{\textwidth}{|l|l|l|X|}
	        		\hline
	        		\rowcolor[gray]{0.75}[4.85pt]
	        		Name & Rückgabetyp & Parameter & Beschreibung \\ \hline 
	        		save & void & keine & Speichert die Daten in die Datenbank; Überschreibt die Standard Methode. In dieser Methode wird die Workflow enthaltende JSON Datei geparst und danach werden einzelne Elemente eines Workflows separat gespeichert. \\ 
	        		\hline
	        	\end{tabularx}
	        \end{center}
%%% ENDE DER KLASSE WORKFLOW
%%% KLASSE SESSION
        \subsubsection{Session}
        
        Attribute:
		\begin{center}
        	\renewcommand{\arraystretch}{1.5}
            \setlength\tabcolsep{5pt}
        	\begin{tabularx}{\textwidth}{|l|l|X|}
        		\hline
                \rowcolor[gray]{0.75}[4.85pt]            		
        	    Name & Datentyp & Beschreibung \\ \hline
        	    id & int & Die ID der Session \\ \hline
        	    user & User & Der zugehörige User \\ \hline
        	    last_workflow & Workflow & Der Workflow der in der Session erstellt wurde \\ \hline
        	\end{tabularx}
        \end{center}
%%% ENDE DER KLASSE SESSION

%%% KLASSE TASK
        \subsubsection{Task}
    		Beschreibung eines einzelnen Tasks innerhalb eines Workflows, die in der Benutzeroberfläche als Knoten des Graphes dargestellt sind. Task ist eine Instanz eines WPS Prozesses, der von dem WPS Server zur Verfügung gestellt wird. Es kann mehrere Tasks geben, die später als derselbe WPS Prozess ausgeführt werden. \newline
    		Diese Klasse erbt von der Django Klasse namens \glqq Model\grqq .
    		
    		Attribute:
			\begin{center}
				\setlength\tabcolsep{5pt}
				\renewcommand{\arraystretch}{1.5}
				
				\begin{tabularx}{\textwidth}{|l|l|X|}
					\hline
					\rowcolor[gray]{0.75}[4.85pt]
					Name & Datentyp & Beschreibung \\ \hline 
	           		id & int & Primärschlüssel in der Datenbank. Wird von Django automatisch generiert \\ \hline
	           		workflow & Workflow & Workflow, dessen Tasks die Kante verbindet \\ \hline
	           		process & Process & Der zugehörende WPS Prozess, der von dem WPS Server zur Verfügung gestellt wird. \\\hline
	           		x & int & X Koordinate des Tasks in der Benutzeroberfläche. Damit werden Tasks immer genau an der Stelle im Editor angezeigt, wo die gespeichert wurden\\ \hline
	           		y & int & Y Koordinate des Tasks in der Benutzeroberfläche\\ \hline
%TODO TODO TODO
	           		status & Status & Ausführungsstatus des Tasks. Siehe workflow.Status Klasse \\ \hline
	           		input\_artefacts & InputArtefact[] & Liste der Eingabedaten \\ \hline
	           		output\_artefacts & OutputArtefact[] & Liste der Endergebnissen \\ \hline
	           		title & string & Titel des Tasks \\ \hline
	           		abstract & string & Kurze Beschreibung des Tasks \\ \hline
	           		status\_url & string & Url des Tasks auf dem WPS Server. Dient der  regelmäßigen Aktualisierung des Ausführungsstatus \\ \hline
	           		started\_at & Date & Datum und die Uhrzeit des Ausführungstarts \\ \hline
				\end{tabularx}
			\end{center}			
%%% ENDE DER KLASSE TASK
%%% ENUMERATION EXECUTION STATUS
		\subsubsection{Status (Enum)}	
			Beschreibt die möglichen Ausführungsstati von Workflows und Tasks. \newline
			
			Attribute:
			\begin{center}
				\setlength\tabcolsep{5pt}
				\renewcommand{\arraystretch}{1.5}
				
				\begin{tabularx}{\textwidth}{|l|l|X|}
					\hline
					\rowcolor[gray]{0.75}[4.85pt]
					Name & Rückgabetyp & Beschreibung \\ \hline 
	           		status\_value & string & Ausführungstatus\newline
	           		READY = Bereit zur Ausführung, wartet auf keine Inputs von anderen Tasks \newline
	           		RUNNING = Befindet sich gerade in der Ausführung auf einem WPS Server \newline 
	           		FINISHED = Workflow oder Task komplett ausgeführt und Ergebnis liegt vor \newline
	           		FAILED = Bei der Ausführung des Tasks ist ein Fehler aufgetreten \newline
	           		DEPRECATED = Der WPS Prozess steht aus eigenen Gründen nicht mehr auf dem WPS Server zur Verfügung \\
	           		\hline
				\end{tabularx}
			\end{center}
%%% ENDE DER ENUMERATION EXECUTION STATUS
%%% KLASSE EDGE
        \subsubsection{Edge}	
    			Diese Klasse beschreibt eine Kante, die in der Benutzeroberfläche zwei Taskknoten miteinander verbindet. \\newline
    			Diese Klasse erbt von der Django Klasse namens \glqq Model\grqq .
    			
    			Attribute:
    			\begin{center}
    				\setlength\tabcolsep{5pt}
    				\renewcommand{\arraystretch}{1.5}
    				
    				\begin{tabularx}{\textwidth}{|l|l|X|}
    					\hline
    					\rowcolor[gray]{0.75}[4.85pt]
    					Name & Rückgabetyp & Beschreibung \\ \hline 
    	           		id & int & Primärschlüssel in der Datenbank. Wird von Django automatisch generiert \\ \hline
    	           		workflow & Workflow & Workflow, dessen Tasks die Kante verbindet \\ \hline
    	           		from\_task & Task & Ausgangsknoten \\ \hline
    	           	    to\_task & Task & Eingangsknoten \\ \hline
    	           		input & Input & Die Eingangsdaten eines Tasks\\ \hline
    	           		output & Output & Das Ergebnis eines Tasks \\
    	           		
    	           		\hline
    				\end{tabularx}
    			\end{center}
%%% ENDE DER KLASSE EDGE
%%% KLASSE PROCESS
		\subsubsection{Process}
			Diese Klasse beschreibt den WPS Prozess, der auf dem WPS Server als Code gespeichert ist. Alle Daten werden von dem WPS Server als XML Datei mittels HTTP Response (Describe Processes) empfangen und danach auf dem Django Server erstmal geparst und in der Datenbank gespeichert. \newline
			Diese Klasse erbt von der Django Klasse namens \glqq Model\grqq .
			
			Attribute:
			\begin{center}
				\setlength\tabcolsep{5pt}
				\renewcommand{\arraystretch}{1.5}
				
				\begin{tabularx}{\textwidth}{|l|l|X|}
					\hline
					\rowcolor[gray]{0.75}[4.85pt]
					Name & Rückgabetyp & Beschreibung \\ \hline 
					id & int & Primärschlüssel in der Datenbank. Wird von Django automatisch generiert \\ \hline
					identifier & String & Eindeutige Kennung des WPS Prozesses \\ \hline
					wps & WPS & Beschreibt den WPS Server, der den WPS Prozess zur Verfügung stellt   \\ \hline
					title & string & Titel des WPS Prozesses \\ \hline
					abstract & string & Beschreibung des WPS Prozesses\\
					\hline
				\end{tabularx}
			\end{center}
%%% ENDE DER KLASSE PROCESS 
%%% KLASSE WPS
        \subsubsection{WPS}
			Diese Klasse beschreibt den WPS Server, der die WPS Prozesse zur Verfügung stellt. \newline
			Diese Klasse erbt von der Django Klasse namens \glqq Model\grqq .
			
			Attribute:
			\begin{center}
				\setlength\tabcolsep{5pt}
				\renewcommand{\arraystretch}{1.5}
				
				\begin{tabularx}{\textwidth}{|l|l|X|}
					\hline
					\rowcolor[gray]{0.75}[4.85pt]
					Name & Rückgabetyp & Beschreibung \\ \hline 
					id & int & Primärschlüssel in der Datenbank. Wird von Django automatisch generiert \\ \hline
					wps\_provider & WPSProvider & Der Besitzer des WPS Servers \\ \hline
					title & string & Name des WPS Servers \\ \hline
					abstract & string & Beschreibung des WPS Servers \\ \hline
					capabilities\_url & string & URL des WPS Servers für die GetCapabilities Operation \\ \hline
					describe\_url & string & URL des WPS Servers für die DescribeProcess Operation \\ \hline
					execute\_url & string & URL des WPS Servers für die Execute Operation \\ \hline
					
				\end{tabularx}
			\end{center}
%%% ENDE DER KLASSE WPS
%%% KLASSE WPSPROVIDER
        \subsubsection{WPSProvider}
			Beschreibt den Besitzer den WPS Servers. Obwohl man ganz viele optionale Attribute auf dem WPS Server eingeben kann, werden nur die wichtigste von denen in der Datenbank gespeichert. \newline
			Diese Klasse erbt von der Django Klasse namens \glqq Model\grqq .
			
			Attribute:
			\begin{center}
				\setlength\tabcolsep{5pt}
				\renewcommand{\arraystretch}{1.5}
				
				\begin{tabularx}{\textwidth}{|l|l|X|}
					\hline
					\rowcolor[gray]{0.75}[4.85pt]
					Name & Rückgabetyp & Beschreibung \\ \hline 
					id & int & Primärschlüssel in der Datenbank. Wird von Django automatisch generiert \\ \hline
					provider\_name & string & Name des WPS Server Providers\\ \hline
					provider\_site & string & Internetadresse des WPS Providers \\ \hline
					individual\_name & string & Name der Person\\ \hline
					position\_name & string & Stelle der Person \\ \hline
				\end{tabularx}
			\end{center}
%%% ENDE DER KLASSE WPSPROVIDER
%%% KLASSE INPUTOUTPUT
		\subsubsection{<<abstract>> InputOutput}
			Beshreibt die Inputs und Outputs eines WPS Prozesses. Die Klasse ist abstakt, dass heißt, dass es keine Tabelle namens InputOutput von Django erzeugt wird. \newline
			Diese Klasse erbt von der Django Klasse namens \glqq Model\grqq .
			
			Attribute:
			\begin{center}
				\setlength\tabcolsep{5pt}
				\renewcommand{\arraystretch}{1.5}
				
				\begin{tabularx}{\textwidth}{|l|l|X|}
					\hline
					\rowcolor[gray]{0.75}[4.85pt]
					Name & Rückgabetyp & Beschreibung \\ \hline 
					id & int & Primärschlüssel in der Datenbank. Wird von Django automatisch generiert \\ \hline
					process & Process & WPS Prozess, zu dem die Inputs bzw. Outputs gehören \\ \hline
					role & Role & Stellt fest, ob die Instanz ein Input oder Output ist \\ \hline
					identifier & string & Eindeutige Kennung des Inputs oder Outputs \\ \hline
					title & string & Titel des Inputs oder Outputs \\ \hline
					abstract & string & Kurze Beschreibung des Inputs oder Outputs \\ \hline
					datatype & Datatype & Bestimmt den Datentyp des Inputs oder Outputs. 
%%%% TODO  TODO TODO
					Siehe .... \\ \hline
					min\_occurs & int & Minimales Vorkommen eines Inputs oder Outputs \\ \hline
					max\_occurs & int & Maximales Vorkommen eines Inputs oder Outputs \\ \hline
				\end{tabularx}
			\end{center}
			
%%% ENDE DER KLASSE INPUTOUTPUT
%%% KLASSE DATATYPE (Enumeration)
        \subsubsection{Datatype (Enum)}	
			Beschreibt die möglichen Datentypen des WPS Prozesses. \newline
			
			Attribute:
			\begin{center}
				\setlength\tabcolsep{5pt}
				\renewcommand{\arraystretch}{1.5}
				
				\begin{tabularx}{\textwidth}{|l|l|X|}
					\hline
					\rowcolor[gray]{0.75}[4.85pt]
					Name & Rückgabetyp & Beschreibung \\ \hline 
	           		datatype & string & Datentyp des WPS Prozesses\newline
	           		LITERAL =  Literal Data ist ein String, normalerweise kurzer. Es wird verwendet um numerische oder textuelle Parameter zu übergeben.   \newline
	           		COMPLEX = Complex Data sind normalerweise die Raster- oder Vektorgrafiken, es können aber auch beliebige dateibasierte Daten sein\newline 
	           		BOUNDING_BOX = Koordinatenpaaren für 2D oder 3D Räume \\ \hline
				\end{tabularx}
			\end{center}

%%% ENDE DER KLASSE  DATATYPE
%%% KLASSE INPUT
    	\subsubsection{Input}
			Beschreibt den Input (Eingangsdaten) des WPS Prozesses. \newline
			Diese Klasse erbt die Attribute von der abstrakten Klasse InputOutput (siehe oben).

%%% ENDE DER KLASSE INPUT
%%% KLASSE OUTPUT 
        \subsubsection{Ouput}
			Beschreibt den Output (Endergebniss) des WPS Prozesses. \newline
			Diese Klasse erbt die Attribute von der abstrakten Klasse InputOutput (siehe oben).
 
%%% ENDE DER KLASSE OUTPUT
%%% ENUMERATION ACTSAS
		\subsubsection{Role (Enum)}
			Beschreibt die möglichen Werte des Attributs acts\_as der abstrakten InputOutput Klasse. \newline
			
			Attribute:
			\begin{center}
				\setlength\tabcolsep{5pt}
				\renewcommand{\arraystretch}{1.5}
				
				\begin{tabularx}{\textwidth}{|l|l|X|}
					\hline
					\rowcolor[gray]{0.75}[4.85pt]
					Name & Rückgabetyp & Beschreibung \\ \hline 
	           		status\_value & string & Ausführungstatus\newline
	           		INPUT = Eingabedaten eine WPS Prozesses \newline
	           		OUTPUT = Endergebniss eines WPS Prozesses \\ \hline
				\end{tabularx}
			\end{center}
%%% ENDE DER ENUMERATION ACTSAS
%%% KLASSE ARTEFACT
        \subsubsection{<<abstract>> Artefact}
			Beschreibt die eigentlichen Eingangsdaten oder die Endegebnisse eines Tasks. Klasse ist abstrakt, das heißt, dass es keine Tabelle in der Datenbank namens \glqq Artefact \grqq erzeugt wird. \newline 
			Diese Klasse erbt von der Django Klasse namens \glqq Model\grqq .
			
			Attribute:
			\begin{center}
				\setlength\tabcolsep{5pt}
				\renewcommand{\arraystretch}{1.5}
				
				\begin{tabularx}{\textwidth}{|l|l|X|}
					\hline
					\rowcolor[gray]{0.75}[4.85pt]
					Name & Rückgabetyp & Beschreibung \\ \hline 
					id & int & Primärschlüssel in der Datenbank. Wird von Django automatisch generiert \\ \hline
					task & Task & Der Task, zu dem die Inputs oder Outputs gehören\\ \hline
					parameter\_id & int & Identifikator des Parameters ???? \\ \hline
					role & Role & Stellt fest, ob die Instanz ein Input oder Output ist \\ \hline
					format & string & Format des Inputs oder Outputs \\ \hline
					data & string & Der Wert dieses Attributs hängt von dem Datentyp ab. Falls der Datentyp Literal Data ist, dann wird es wirklich ein String mit dem Ergebniss gespeichert. Falls der Datentyp Complex Data oder Bounding Box Data ist, also falls das Ergebniss zu groß ist, um es in der Datenbank zu speichern, wird in diesem Attribut eine URL gespeichert, die zum Ergebniss auf dem WPS Server führt \\ \hline
					created\_at & datetime & Erstellungsdatum des Inputs/Outputs \\ \hline
					updated\_at & datetime & Datum und die Uhrzeit der letzten Änderung \\ \hline
				\end{tabularx}
			\end{center}
%%% ENDE DER KLASSE ARTEFACT
%%% KLASSE INPUTARTEFACT
        \subsubsection{InputArtefact}
			Beschreibt den die Eingangsdaten des Tasks. \newline
			Diese Klasse erbt alle Attribute von der abstrakten Klasse Artefact (siehe oben).
			
%%%ENDE DER KLASSE INPUTARTEFACT
%%% KLASSE OUTPUTARTEFACT 
        \subsubsection{OutputArtefact}
			Beschreibt das Endergebniss des Tasks. \newline
			Diese Klasse erbt alle Attribute von der abstrakten Klasse Artefact (siehe oben).
			
%%% ENDE DER KLASSE OUTPUTARTEFACT
		

%     \begin{center}
%     	\renewcommand{\arraystretch}{1.5}
%         \setlength\tabcolsep{5pt}
%     	\begin{tabularx}{\textwidth}{|l|l|X|}
%     		\hline
%             \rowcolor[gray]{0.75}[4.85pt]            		
%     	    Name & Datentyp & Beschreibung \\ \hline
%            
%           \hline            		
%     	\end{tabularx}
%     \end{center}


        % table Muster Methoden
% 		\begin{center}
% 		    \setlength\tabcolsep{5pt}
%         	\renewcommand{\arraystretch}{1.5}
%             	\begin{tabularx}{\textwidth}{|l|l|l|X|}
%             	\hline
%             	\rowcolor[gray]{0.75}[4.85pt]
%         		Name & Rückgabetyp & Parameter & Beschreibung \\ \hline 
%
%                \hline
%             	\end{tabularx}
% 		\end{center}

\newpage

    \section{Client Klassen}
    
        \subsection{services}
        
            \subsubsection{AuthService}
            
                Attribute:
                \begin{center}
                	\renewcommand{\arraystretch}{1.5}
                    \setlength\tabcolsep{5pt}
                	\begin{tabularx}{\textwidth}{|l|l|X|}
                		\hline
                        \rowcolor[gray]{0.75}[4.85pt]            		
                	    Name & Datentyp & Beschreibung \\ \hline
                        auth & Observable<User> & \\ \hline
                	\end{tabularx}
                \end{center}
                
                Methoden:
        		\begin{center}
        		    \setlength\tabcolsep{5pt}
                	\renewcommand{\arraystretch}{1.5}
                    	\begin{tabularx}{\textwidth}{|l|l|l|X|}
                    	\hline
                    	\rowcolor[gray]{0.75}[4.85pt]
                		Name & Rückgabetyp & Parameter & Beschreibung \\ \hline 
                        constructor & - & HTTPClient & Konstruktor \\ \hline
                        logout & Promise<boolean> & - & Nutzer-logout \\ \hline
                        login & Promise<boolean> & Nutzer-login \thead{email: string\\password: string} & \\ \hline
                    	\end{tabularx}
        		\end{center}
            
            
            \subsubsection{WPSService}
                
                Methoden:
        		\begin{center}
                \setlength\tabcolsep{5pt}
                	\renewcommand{\arraystretch}{1.5}
                    	\begin{tabularx}{\textwidth}{|l|l|l|X|}
                    	\hline
                    	\rowcolor[gray]{0.75}[4.85pt]
                		Name & Rückgabetyp & Parameter & Beschreibung \\ \hline 
                        constructor & - & HTTPClient & Konstruktor\\ \hline
                        all & Observable<WPS[]> & - & Gibt eine Liste aller WPS Services zurück\\ \hline
                        get & Observable<WPS> & id: number & Gibt den WPS Service mit der übergebenen ID zurück \\ \hline
                        create & Observable<WPS> & process: Process & Erzeugt einen neuen WPS Service mit dem übergebenen Prozess \\ \hline
                        remove & Promise<boolean> & id: number & Entfernt den WPS Service mit der übergebenen ID \\ \hline
                        update & Observable<WPS> & \thead{id: number\\Partial<WPS>} & Aktualisiert den WPS Service mit der übergebenen ID mit dem zweiten übergebenen Parameter \\ \hline
                        \end{tabularx}
        		\end{center}
            
            \subsubsection{WorkflowService}
                
                Methoden:
        		\begin{center}
                \setlength\tabcolsep{5pt}
                	\renewcommand{\arraystretch}{1.5}
                    	\begin{tabularx}{\textwidth}{|l|l|l|X|}
                    	\hline
                    	\rowcolor[gray]{0.75}[4.85pt]
                		Name & Rückgabetyp & Parameter & Beschreibung \\ \hline 
                        constructor & - & HTTPClient & Konstruktor \\ \hline
                        all & Observable<Workflow[]> & - & Gibt eine Liste aller Workflows zurück\\ \hline
                        get & Observable<Workflow> & id: number & Gibt den Workflow mit der übergebenen ID zurück \\ \hline
                        create & Observable<Workflow> & workflow: Workflow & Erzeugt einen neuen Workflow mit dem übergebenen Workflow \\ \hline
                        remove & Promise<boolean> & id: number & Entfernt den Workflow mit der übergebenen ID \\ \hline
                        update & Observable<Workflow> & \thead{id: number\\Partial<Workflow>} & Aktualisiert den Workflow mit der übergebenen ID mit dem zweiten übergebenen Parameter \\ \hline
                        validate & boolean & workflow: Workflow & Überprüft die Syntax des übergebenen Workflows und gibt zurück ob er korrekt ist oder nicht \\ \hline
                        start & void & id: number & Startet den Workflow mit der übergebenen ID \\ \hline
                        stop & void & id: number & Stoppt den Workflow mit der übergebenen ID \\ \hline
                        \end{tabularx}
        		\end{center}
            
            \subsubsection{ProcessService}
            
                Attribute:
                \begin{center}
                \setlength\tabcolsep{5pt}
                	\renewcommand{\arraystretch}{1.5}
                    	\begin{tabularx}{\textwidth}{|l|l|l|X|}
                    	\hline
                    	\rowcolor[gray]{0.75}[4.85pt]
                		Name & Rückgabetyp & Parameter & Beschreibung \\ \hline 
                        constructor & - & HTTPClient & Konstruktor \\ \hline
                        all & Observable<Process[]> & - & Gibt eine Liste aller Prozesse zurück \\ \hline
                        get & Observable<Process> & id: number & Gibt den Prozess mit der übergebenen ID zurück \\ \hline
                        create & Observable<Process> & process: Process & Erzeugt einen neuen Prozess mit dem übergebenen Prozess \\ \hline
                        remove & Promise<boolean>  & id: number & Entfernt den Prozess mit der übergebenen ID \\ \hline
                        update & Observable<Process> & \thead{id: number\\Partial<Process>} & Aktualisiert den Prozess mit der übergebenen ID mit dem zweiten übergebenen Parameter \\ \hline
                	\end{tabularx}
                \end{center}
    
        \subsection{models}
            
    		\subsubsection{Workflow}
    		
    		Attribute:
            \begin{center}
            	\renewcommand{\arraystretch}{1.5}
                \setlength\tabcolsep{5pt}
            	\begin{tabularx}{\textwidth}{|l|l|X|}
            		\hline
                    \rowcolor[gray]{0.75}[4.85pt]            		
                    Name & Datentyp & Beschreibung \\ \hline
                    edges & WorkflowEdge[] & Die Verbindungen zwischen Tasks im Workflow \\ \hline
                    tasks & Task[] & Die Tasks im Workflow \\ \hline
                    creator_id & number & Die ID des Workflow-Erstellers \\ \hline
                    shared & boolean & Diese Variable speichert ob der Workflow öffentlich oder privat ist \\ \hline
                    name & string & Der Name des Workflows \\ \hline
                    id & number & Die ID des Workflows \\ \hline
                    created_at & number & Das Datum an dem der Workflow erstellt wurde \\ \hline
                    updated_at & number & Das Datum an dem der Workflow das letzte mal aktualisiert wurde \\ \hline
            	\end{tabularx}
            \end{center}
                
            \subsubsection{Edge}
    		
    		Attribute:
            \begin{center}
            	\renewcommand{\arraystretch}{1.5}
                \setlength\tabcolsep{5pt}
            	\begin{tabularx}{\textwidth}{|l|l|X|}
            		\hline
                    \rowcolor[gray]{0.75}[4.85pt]            		
                    Name & Datentyp & Beschreibung \\ \hline
                     id & number & Die ID der Kante \\ \hline
                     a & Task & Der Task am Anfang der Kante \\ \hline
                     b & Task & Der Task am Ende der Kante \\ \hline
                     input_id & number & Die ID des Input-Punktes mit dem die Kante verbunden ist \\ \hline
                     output_id & number & Die ID des Output-Punktes mit dem die Kante verbunden ist \\ \hline
            	\end{tabularx}
            \end{center}
            
            \subsubsection{Task}
            
            Attribute:
            \begin{center}
            	\renewcommand{\arraystretch}{1.5}
                \setlength\tabcolsep{5pt}
            	\begin{tabularx}{\textwidth}{|l|l|X|}
            		\hline
                    \rowcolor[gray]{0.75}[4.85pt]            		
                    Name & Datentyp & Beschreibung \\ \hline
                    id & number & Die ID des Tasks \\ \hline
                    y & number & Die y-Koordinate des Tasks\\ \hline
                    x & number & Die x-Koordinate des Tasks \\ \hline
                    status & TaskState & Der Status des Tasks \\ \hline
                    input_artefacts & Artefact <'input'> & Die Input-Parameter des Tasks \\ \hline
                    process_id & number & Die ID des Prozesses der im Task ausgeführt wird \\ \hline
                    output_artefacts & Artefact <'output'> & Die Output-Parameter des Tasks \\ \hline
                    created_at & number & Das Datum an dem der Task erstellt wurde \\ \hline
                    updated_at & number & Das Datum an dem der Task das letzte mal aktualisiert wurde \\ \hline
            	\end{tabularx}
            \end{center}
                
    		\subsubsection{Artefact<T>}
    		
    		Attribute:
            \begin{center}
            	\renewcommand{\arraystretch}{1.5}
                \setlength\tabcolsep{5pt}
            	\begin{tabularx}{\textwidth}{|l|l|X|}
            		\hline
                    \rowcolor[gray]{0.75}[4.85pt]            		
                    Name & Datentyp & Beschreibung \\ \hline
            	    parameter_id & number & Die ID des Input- oder Output-Knotens zu der die Artefact-Instanz gehört \\ \hline
            	    task_id & number & Die ID des Tasks zu dem die Artefact-Instanz gehört \\ \hline
            	    workflow_id & number & Die ID des Workflows zu dem die Artefact-Instanz gehört \\ \hline
            	    role & T & Die role gibt an ob die Artefact-Instanz als Input oder als Output verwendet wird \\ \hline
                    format & string & Gibt das Format der Artefact-Instanz an \\ \hline
                    data & string & Übergibt die Daten die in der Artefact-Instanz verwendet werden \\ \hline
                    created_at & number & Das Datum an dem die Artefact-Instanz erstellt wurde \\ \hline
            	    updated_at & number & Das Datum an dem die Artefact-Instanz das letzte mal aktualisiert wurde\\ \hline
            	\end{tabularx}
            \end{center}
            
    		\subsubsection{TaskState (Enum)}
			Beschreibt die möglichen Werte des Attributs acts\_as der abstrakten InputOutput Klasse. \newline
			
			Attribute:
			\begin{center}
            	\renewcommand{\arraystretch}{1.5}
	            \setlength\tabcolsep{5pt}
            	\begin{tabularx}{\textwidth}{|l|X|}
            		\hline
                    \rowcolor[gray]{0.75}[4.85pt]
            	    Name & Beschreibung \\ \hline
            	    READY & Der Task ist bereit ausgeführt zu werden \\ \hline
            		WAITING & In diesem Zustand wartet ein Task auf den Input eines noch nicht beendeten Tasks \\ \hline
            		RUNNING & Der Task ist momentan in bearbeitung \\ \hline
            		FINISHED & Der Task wurde erfolgreich ausgeführt \\ \hline
            		FAILED & Bei der Ausführung des Tasks ist ein Fehler aufgetreten \\ \hline
            		DEPRECATED & Es ist aus bestimmten Gründen nichtmehr sicher ob der Task aktuell oder verfügbar ist\\ \hline
				\end{tabularx}
			\end{center}
                
    		\subsubsection{Process}
    		
    		Attribute:
            \begin{center}
            	\renewcommand{\arraystretch}{1.5}
                \setlength\tabcolsep{5pt}
            	\begin{tabularx}{\textwidth}{|l|l|X|}
            		\hline
                    \rowcolor[gray]{0.75}[4.85pt]            		
                    Name & Datentyp & Beschreibung \\ \hline
            		id & number & Die ID des Prozesses \\ \hline
            		title & string & Der Titel des Prozesses \\ \hline
            		abstract & string & Eine kurze Beschreibung des Prozesses \\ \hline
            		identifier & string & Die ID des Prozesses die vom WPS-Server festgelegt wird \\ \hline
                    inputs & ProcessParameter<'input'>[] & Der Typ der Input-Parameter des Prozesses \\ \hline
                    outputs & ProcessParameter<'output'>[] & Der Typ der Output-Parameter des Prozesses \\ \hline
            		wps_id & number & Die ID des WPS-Servers der den Prozess bereit stellt \\ \hline
            		created_timestamp & number & Das Datum an dem der Prozess erstellt wurde \\ \hline
            		updated_timestamp & number & Das Datum an dem der Prozess das letzte mal aktualisiert wurde \\ \hline
            	\end{tabularx}
            \end{center}
                
    		\subsubsection{ProcessParameter<T>}
    		
    		Attribute:
            \begin{center}
            	\renewcommand{\arraystretch}{1.5}
                \setlength\tabcolsep{5pt}
            	\begin{tabularx}{\textwidth}{|l|l|X|}
            		\hline
                    \rowcolor[gray]{0.75}[4.85pt]            		
                    Name & Datentyp & Beschreibung \\ \hline
                    id: number+ role: T+ parameter_type: ProcessParameterType+ title: string+ abstract: string+ min_occurs: number+ max_occurs: number
            		id & number &  \\ \hline
            		role & T & \\ \hline
            		T & parameter_type &  \\ \hline
            		ProcessParamenterType & paramerer_type &  \\ \hline
            		string & title &  \\ \hline
            		string & abstract &  \\ \hline
            		number & min_occurs &  \\ \hline
            		number & max_occurs &  \\ \hline
            	\end{tabularx}
            \end{center}
                
    		\subsubsection{ProcessParamenterType}
    		
    		ENUM
            \begin{center}
            	\renewcommand{\arraystretch}{1.5}
	            \setlength\tabcolsep{5pt}
            	\begin{tabularx}{\textwidth}{|l|X|}
            		\hline
                    \rowcolor[gray]{0.75}[4.85pt]
            	    Name & Beschreibung \\ \hline
            		COMPLEX &   \\ \hline
            		LITERAL &   \\ \hline
            		BOUNDING_BOX  &  \\ \hline
            	\end{tabularx}
            \end{center}
                
    		\subsubsection{WPS}
    		
    		Attribute:
            \begin{center}
            	\renewcommand{\arraystretch}{1.5}
                \setlength\tabcolsep{5pt}
            	\begin{tabularx}{\textwidth}{|l|l|X|}
            		\hline
                    \rowcolor[gray]{0.75}[4.85pt]            		
                    Name & Datentyp & Beschreibung \\ \hline
            		number & id &  \\ \hline
            		string & abstract &  \\ \hline
            		WPSProvider & provider &  \\ \hline
            	\end{tabularx}
            \end{center}
                
    		\subsubsection{WPSProvider}
    		
    		Attribute:
            \begin{center}
            	\renewcommand{\arraystretch}{1.5}
                \setlength\tabcolsep{5pt}
            	\begin{tabularx}{\textwidth}{|l|l|X|}
            		\hline
                    \rowcolor[gray]{0.75}[4.85pt]            		
                    Name & Datentyp & Beschreibung \\ \hline
            		number & id &  \\ \hline
            		string & name &  \\ \hline
            		string & site &  \\ \hline
            	\end{tabularx}
            \end{center}
                
    		\subsubsection{User}
            
            \begin{center}
            	\renewcommand{\arraystretch}{1.5}
                \setlength\tabcolsep{5pt}
            	\begin{tabularx}{\textwidth}{|l|l|X|}
            		\hline
                    \rowcolor[gray]{0.75}[4.85pt]            		
            	    Name & Datentyp & Beschreibung \\ \hline
                    id & number & Die ID des Nutzers \\ \hline
                    created_at & number & Das Datum an dem die Nutzer-Instanz erstellt wurde \\ \hline
                    updated_at & number & Das Datum an dem die Nutzer-Instanz das letzte mal aktualisiert wurde\\ \hline
                    last_workflow_id & number & Die ID des Workflows den der Nutzer zuletzt im Editor geladen hatte \erade geladen hat\ \hline
                    email & string & Die E-Mail Adresse des Nutzers \\ \hline
                    last_name & string & Der Nachname des Nutzers\\ \hline
                    first_name & string & Der Vorname des Nutzers\\ \hline
                    username & string & Der Nutzername des Nutzers\\ \hline
                    group & UserGroup & Die UserGroup des Nutzers\\ \hline
                    \hline            		
            	\end{tabularx}
            \end{center}
                
    		\subsubsection{UserGroup}
    		
    		ENUM
            \begin{center}
            	\renewcommand{\arraystretch}{1.5}
	            \setlength\tabcolsep{5pt}
            	\begin{tabularx}{\textwidth}{|l|X|}
            		\hline
                    \rowcolor[gray]{0.75}[4.85pt]
            	    Name & Beschreibung \\ \hline
            		REGULAR &  \\ \hline
            		ADMIN &  \\ \hline
            	\end{tabularx}
            \end{center}
    
        \subsection{components}
        
            \subsubsection{AppComponent}
            
                Attribute:
                \begin{center}
                	\renewcommand{\arraystretch}{1.5}
                    \setlength\tabcolsep{5pt}
                	\begin{tabularx}{\textwidth}{|l|l|X|}
                		\hline
                        \rowcolor[gray]{0.75}[4.85pt]            		
                        Name & Datentyp & Beschreibung \\ \hline
                        
                	\end{tabularx}
                \end{center}
                
                Methoden:
        		\begin{center}
                \setlength\tabcolsep{5pt}
                	\renewcommand{\arraystretch}{1.5}
                    	\begin{tabularx}{\textwidth}{|l|l|l|X|}
                    	\hline
                    	\rowcolor[gray]{0.75}[4.85pt]
                		Name & Rückgabetyp & Parameter & Beschreibung \\ \hline 
                        
                        \end{tabularx}
        		\end{center}
	
	\section{Entwurfsmuster}
	
	\begin{itemize}
	    \item Fassade - WorkflowView ist eine Fassade für Workflow, Task und Edge Klassen
	    \item Strategie - Anzeigen der Ergebnissen im Dialogfenster hängt von dem Format - ob es Bild ist, XML oder String 
	\end{itemize}
	die verwendeten entwurfsmuster sollen ja auch irgendwo erwähnt werden.
	also die identifikation von entwurfsmustern um struktur gröber zu beschreiben
	zb iterator für die dynamische einbindung der implementierten task klassen
