\chapter{Klassen und Pakete}

    \section{edu.kit.scc.pseworkflow.views}

		\subsection{Workflows\newline
		extends django.views.View}

		\subsection{Services\newline
		extends django.views.View}

    \section{edu.kit.scc.pseworkflow.cron}

		\subsection{Cron}

    \section{edu.kit.scc.pseworkflow.models.workflows}

    \section{edu.kit.scc.pseworkflow.models.services}

    \section{edu.kit.scc.pseworkflow.models.sessions}
        
    \section{Klassen}
%%% KLASSE WORKFLOW
        \subsection{edu.kit.scc.pseworkflow.models.Workflow}
	        Diese Klasse beschreibt einen Workflow. Ein Workflow enthält einen oder mehrere Tasks.
                    
            Attribute:
            \begin{center}
            	\renewcommand{\arraystretch}{1.5}
	            \setlength\tabcolsep{5pt}
            	\begin{tabularx}{\textwidth}{|l|l|l|X|}
            		\hline
                    \rowcolor[gray]{0.75}[4.85pt]            		
            	    Typ & Name & Rückgabetyp & Beschreibung \\ \hline
            		Attribut & id & int & Primärschlüssel in der Datenbank. Wird von Django automatisch generiert \\ \hline
					Attribut & identifier & string & Eindeutige Kennung des Workflows\\ \hline
					Attribut & name & string & Anzeigetitel und Name des Workflows\\ \hline
					Attribut & description & string & Beschreibungstext des Workflows\\ \hline				Attribut & tasks & Task & Liste der Tasks, die zu dem Workflow gehören\\ \hline
					%Attribut & wf\_status & string & Ausführungsstatus des Workflows\\ \hline
					Attribut & percent\_done & int & Fortschritt des Workflows in Prozent\\ \hline
					Attribut & created\_at & Date & Erzeugungsdatum des Workflows \\ \hline
					Attribut & updated\_at & Date & Datum der letzten Veränderung \\ \hline 
					Attribut & creator & User & Benutzer, der den Workflow erstellt hat\\ \hline
					%Attribut & wf\_shared\_with & string & Liste der Benutzer, die diesen Workflow %ebenfalls sehen und bearbeiten dürfen; Komma getrennte Liste\\ \hline
					Attribut & last\_modifier & User & Benutzer, der am letzten den Workflow verändert hat\\ \hline
					
					%Attribut & wf\_executable & boolean & Hilfsattribut für Automatischen Workflow %Scheduler; Scheduler sendet den Workflow erst zur Ausführung, wenn wf\_executable = %true; Standardwert false \\				
					\hline            		
            	\end{tabularx}
            \end{center}

        
             Methoden:
	        \begin{center}
	        	\setlength\tabcolsep{5pt}
	        	\renewcommand{\arraystretch}{1.5}
	        	
	        	\begin{tabularx}{\textwidth}{|l|l|l|X|}
	        		\hline
	        		\rowcolor[gray]{0.75}[4.85pt]
	        		Typ & Name & Rückgabetyp & Beschreibung \\ \hline 
	        		Methode & save & void & Speichert die Daten in die Datenbank; Überschreibt die Standard Methode. In dieser Methode wird die Workflow enthaltende JSON Datei geparst und danach werden einzelne Elemente eines Workflows separat gespeichert. \\ 
	        		\hline
	        	\end{tabularx}
	        \end{center}
%%% ENDE DER KLASSE WORKFLOW
%%% KLASSE TASK
        \subsection{edu.kit.scc.pseworkflow.models.Task}	
    		Beschreibung eines einzelnen Tasks innerhalb eines Workflows, die in der Benutzeroberfläche als Knoten des Graphes dargestellt sind. Task ist eine Instanz eines WPS Prozesses, der von dem WPS Server zur Verfügung gestellt wird. Es kann mehrere Tasks geben, die später als derselbe WPS Prozess ausgeführt werden. 
    		
    		Atttribute:
			\begin{center}
				\setlength\tabcolsep{5pt}
				\renewcommand{\arraystretch}{1.5}
				
				\begin{tabularx}{\textwidth}{|l|l|l|X|}
					\hline
					\rowcolor[gray]{0.75}[4.85pt]
					Typ & Name & Rückgabetyp & Beschreibung \\ \hline 
	           		Attribut & id & int & Primärschlüssel in der Datenbank. Wird von Django automatisch generiert \\ \hline
	           		Attribut & process & Process & Der zugehörende WPS Prozess, der von dem WPS Server zur Verfügung gestellt wird. \\\hline
	           		Attribut & x & int & X Koordinate des Tasks in der Benutzeroberfläche. Damit werden Tasks immer genau an der Stelle im Editor angezeigt, wo die gespeichert wurden\\ \hline
	           		Attribut & y & int & Y Koordinate des Tasks in der Benutzeroberfläche\\ \hline
%TODO TODO TODO
	           		Attribut & status & Status & Ausführungsstatus des Tasks. Siehe workflow.Status Klasse \\ \hline
	           		Attribut & input\_artefacts & InputArtefact[] & Liste der ??? \\ \hline
	           		Attribut & output\_artefacts & OutputArtefact[] & Liste der ??? \\ \hline
	           		Attribut & title & string & Titel des Tasks \\ \hline
	           		Attribut & abstract & string & Kurze Beschreibung des Tasks \\ \hline
	           		Attribut & status\_url & string & Url des Tasks auf dem WPS Server. Dient der  regelmäßigen Aktualisierung des Ausführungsstatus \\ \hline
	           		Attribut & started\_at & Date & Datum und die Uhrzeit des Ausführungstarts \\ \hline
				\end{tabularx}
			\end{center}
    			
			Methoden:
			\begin{center}
				\setlength\tabcolsep{5pt}
				\renewcommand{\arraystretch}{1.5}
				
				\begin{tabularx}{\textwidth}{|l|l|l|X|}
					\hline
					\rowcolor[gray]{0.75}[4.85pt]
					Typ & Name & Rückgabetyp & Beschreibung \\ \hline
					Methode & save & void & Speichert den Task in der Datenbank \\ 
					\hline
				\end{tabularx}
			\end{center}			
%%% ENDE DER KLASSE TASK
%%% KLASSE EDGE
        \subsection{edu.kit.scc.pseworkflow.models.ExecutionStatus}	
    			Diese Klasse beschreibt eine Kante, in der Benutzeroberfläche zwei Taskknoten miteinander verbindet. 
    			
    			Atttribute:
    			\begin{center}
    				\setlength\tabcolsep{5pt}
    				\renewcommand{\arraystretch}{1.5}
    				
    				\begin{tabularx}{\textwidth}{|l|l|l|X|}
    					\hline
    					\rowcolor[gray]{0.75}[4.85pt]
    					Typ & Name & Rückgabetyp & Beschreibung \\ \hline 
    	           		Attribut & id & int & Primärschlüssel in der Datenbank. Wird von Django automatisch generiert \\ \hline
    	           		Attribut & from & Task & Ausgangsknoten \\ \hline
    	           		Attribut & to & Task & Eingangsknoten \\ \hline
    	           		Attribut & input & Input & Die Eingangsdaten des \\
    	           		
    	           		\hline
    				\end{tabularx}
    			\end{center}
    			
    			Methoden:
    			\begin{center}
    				\setlength\tabcolsep{5pt}
    				\renewcommand{\arraystretch}{1.5}
    				
    				\begin{tabularx}{\textwidth}{|l|l|l|X|}
    					\hline
    					\rowcolor[gray]{0.75}[4.85pt]
    					Typ & Name & Rückgabetyp & Beschreibung \\ \hline
    					&&& \\ 
    					\hline
    				\end{tabularx}
    			\end{center}
    			
    			
    
%%% ENDE DER KLASSE EDGE
    
		\subsection{edu.kit.scc.pseworkflow.models.ExecutionStatus}	
			Beschreibt die Möglichen Ausführungsstati von Workflows und Tasks.
			
			Methoden:
			\begin{center}
				\setlength\tabcolsep{5pt}
				\renewcommand{\arraystretch}{1.5}
				
				\begin{tabularx}{\textwidth}{|l|l|l|X|}
					\hline
					\rowcolor[gray]{0.75}[4.85pt]
					Typ & Name & Rückgabetyp & Beschreibung \\ \hline
					&&& \\ 
					\hline
				\end{tabularx}
			\end{center}
			
			Atttribute:
			\begin{center}
				\setlength\tabcolsep{5pt}
				\renewcommand{\arraystretch}{1.5}
				
				\begin{tabularx}{\textwidth}{|l|l|l|X|}
					\hline
					\rowcolor[gray]{0.75}[4.85pt]
					Typ & Name & Rückgabetyp & Beschreibung \\ \hline 
	           		Attribut & id & int & Primärschlüssel in Datenbank, wird von Django automatisch generiert \\ \hline
	           		Attribut & status\_value & string & Ausführungstatus; READY = Bereit zur Ausführung, RUNNING = Wird ausgeführt, FINISHED = Workflow oder Task komplett ausgeführt und Ergebnis liegt vor, FAILED = Bei der Ausführung ist ein Fehler aufgetreten \\
	           		\hline
				\end{tabularx}
			\end{center}

		\subsection{edu.kit.scc.pseworkflow.models.DataType}
			Auflistung der möglichen Datentypen für Input und Output, die PyWPS bereitstellt.
			
			Methoden:
			\begin{center}
				\setlength\tabcolsep{5pt}
				\renewcommand{\arraystretch}{1.5}
				
				\begin{tabularx}{\textwidth}{|l|l|l|X|}
					\hline
					\rowcolor[gray]{0.75}[4.85pt]
					Typ & Name & Rückgabetyp & Beschreibung \\ \hline 
					&&& \\
					\hline
				\end{tabularx}
			\end{center}
			
			Atttribute:
			\begin{center}
				\setlength\tabcolsep{5pt}
				\renewcommand{\arraystretch}{1.5}
				
				\begin{tabularx}{\textwidth}{|l|l|l|X|}
					\hline
					\rowcolor[gray]{0.75}[4.85pt]
					Typ & Name & Rückgabetyp & Beschreibung \\ \hline 
					Attribut & id & int & Primärschlüssel in Datenbank, wird von Django automatisch generiert \\ \hline
					Attribut & type\_value & string & Datentyp aus PyWPS; LiteralData umfasst Text, Zahlen und boolsche Werte, ComplexData umfasst Vektordateien und BoundingBoxData sind Boxdaten/Rasterdaten [?] \\
					\hline
				\end{tabularx}
			\end{center}
		
		\subsection{edu.kit.scc.pseworkflow.models.ProcessHandler}
			Die Namen der von den PyWPS-Servern bereitgestellten Prozesse, entspricht ProcessIdentifier im PyWPS GML.
			
			Methoden:
			\begin{center}
				\setlength\tabcolsep{5pt}
				\renewcommand{\arraystretch}{1.5}
				
				\begin{tabularx}{\textwidth}{|l|l|l|X|}
					\hline
					\rowcolor[gray]{0.75}[4.85pt]
					Typ & Name & Rückgabetyp & Beschreibung \\ \hline 
					&&& \\
					\hline
				\end{tabularx}
			\end{center}
			
			Atttribute:
			\begin{center}
				\setlength\tabcolsep{5pt}
				\renewcommand{\arraystretch}{1.5}
				
				\begin{tabularx}{\textwidth}{|l|l|l|X|}
					\hline
					\rowcolor[gray]{0.75}[4.85pt]
					Typ & Name & Rückgabetyp & Beschreibung \\ \hline 
					Attribut & id & int & Primärschlüssel in Datenbank, wird von Django automatisch generiert \\ \hline
					Attribut & handler\_name & string & Name der Klasse der Funktion, die der PyWPS-Server zur Verfügung stellt; entspricht ProcessIdentifier im PyWPS GML \\
					\hline
				\end{tabularx}
			\end{center}
		
		
		\subsection{edu.kit.scc.pseworkflow.models.Session}
			[[[Bitte ausfüllen von denen, die Ahnung davon haben]]]
			
			Methoden:
			\begin{center}
				\setlength\tabcolsep{5pt}
				\renewcommand{\arraystretch}{1.5}
				
				\begin{tabularx}{\textwidth}{|l|l|l|X|}
					\hline
					\rowcolor[gray]{0.75}[4.85pt]
					Typ & Name & Rückgabetyp & Beschreibung \\ \hline 
					&&& \\
					\hline
				\end{tabularx}
			\end{center}
			
			Atttribute:
			\begin{center}
				\setlength\tabcolsep{5pt}
				\renewcommand{\arraystretch}{1.5}
				
				\begin{tabularx}{\textwidth}{|l|l|l|X|}
					\hline
					\rowcolor[gray]{0.75}[4.85pt]
					Typ & Name & Rückgabetyp & Beschreibung \\ \hline 
					Attribut & id & int & Primärschlüssel in Datenbank, wird von Django automatisch generiert \\ \hline
					Attribut & user\_id & int & bitte ausfüllen \\ \hline
					Attribut & last\_workflow\_id & int & bitte ausfüllen \\
					\hline
				\end{tabularx}
			\end{center}
		
		\subsection{edu.kit.scc.pseworkflow.models.ServiceProvider}
			[[[bitte ausfüllen]]]
			
			Methoden:
			\begin{center}
				\setlength\tabcolsep{5pt}
				\renewcommand{\arraystretch}{1.5}
				
				\begin{tabularx}{\textwidth}{|l|l|l|X|}
					\hline
					\rowcolor[gray]{0.75}[4.85pt]
					Typ & Name & Rückgabetyp & Beschreibung \\ \hline 
					&&& \\
					\hline
				\end{tabularx}
			\end{center}
			
			Atttribute:
			\begin{center}
				\setlength\tabcolsep{5pt}
				\renewcommand{\arraystretch}{1.5}
				
				\begin{tabularx}{\textwidth}{|l|l|l|X|}
					\hline
					\rowcolor[gray]{0.75}[4.85pt]
					Typ & Name & Rückgabetyp & Beschreibung \\ \hline 
					Attribut & id & int & Primärschlüssel in Datenbank, wird von Django automatisch generiert \\ \hline
					Attribut & provider\_name & string & Name des ServiceProviders \\ \hline
					Attribut & provider\_site & string & Website des Providers \\
					\hline
				\end{tabularx}
			\end{center}
		
		\subsection{edu.kit.scc.pseworkflow.models.Service}
			[[[bitte ausfüllen]]]
			
			Methoden:
			\begin{center}
				\setlength\tabcolsep{5pt}
				\renewcommand{\arraystretch}{1.5}
				
				\begin{tabularx}{\textwidth}{|l|l|l|X|}
					\hline
					\rowcolor[gray]{0.75}[4.85pt]
					Typ & Name & Rückgabetyp & Beschreibung \\ \hline 
					&&& \\
					\hline
				\end{tabularx}
			\end{center}
			
			Atttribute:
			\begin{center}
				\setlength\tabcolsep{5pt}
				\renewcommand{\arraystretch}{1.5}
				
				\begin{tabularx}{\textwidth}{|l|l|l|X|}
					\hline
					\rowcolor[gray]{0.75}[4.85pt]
					Typ & Name & Rückgabetyp & Beschreibung \\ \hline 
					Attribut & id & int & Primärschlüssel in Datenbank, wird von Django automatisch generiert \\ \hline
					Attribut & service\_provider & ServiceProvider & [[bitte ausfüllen]] \\ \hline
					Attribut & capabilities\_url & string & get capabilities url des providers \\ \hline
					Attribut & describe\_url & string & describe url des providers \\ \hline
					Attribut & execute\_url & string & execute url des providers \\ \hline
					Attribut & service\_title & string & Titel des Service \\ \hline
					Attribut & service\_abstract & string & Beschreibung \\
					\hline
				\end{tabularx}
			\end{center}
		
		
%		\subsection{edu.kit.scc.pseworkflow.models.User}
%			Datenmodell eines Users.
%			
%			Methoden:
%			\begin{center}
%				\setlength\tabcolsep{5pt}
%				\renewcommand{\arraystretch}{1.5}
%				
%				\begin{tabularx}{\textwidth}{|l|l|l|l|X|}
%					\hline
%					\rowcolor[gray]{0.75}[4.85pt]
%					Typ & Name & Rückgabetyp & Beschreibung \\ \hline 
%					Methode & save & void & Speichert die Daten in die Datenbank; Überschreibt die Standard Methode \\ \hline
%					Methode & comparePwHash & bool & Vergleicht den Hashwert eines eingegebenen Passworts mit dem gespeicherten Hashwert des Benutzerpassworts \\
%					\hline
%				\end{tabularx}
%			\end{center}
%			
%			Atttribute:
%			\begin{center}
%				\setlength\tabcolsep{5pt}
%				\renewcommand{\arraystretch}{1.5}
%				
%				\begin{tabularx}{\textwidth}{|l|l|l|l|X|}
%					\hline
%					\rowcolor[gray]{0.75}[4.85pt]
%					Typ & Name & Rückgabetyp & Beschreibung \\ \hline 
%					Attribut & id & int & Primärschlüssel in Datenbank, wird von Django automatisch generiert \\ \hline
%					Attribut & user\_login & string & Loginname des Benutzers \\
%					Attribut & user\_pw & string & Hashwert des Benutzerpassworts\\ \hline
%					Attribut & user\_name & string & Nachname des Benutzers \\ \hline
%					Attribut & user\_prename & string & Vorname des Benutzers \\ \hline
%					Attribut & user\_mail & string & Mailadresse des Benutzers \\ \hline
%					Attribut & user\_level & string & Rechtelevel des Benutzers; admin = Adminuser mit allen Konfigurationsrechten, normal = Normaler Benutzer ohne weitere Konfigurationsrechte \\
%					\hline
%				\end{tabularx}
%			\end{center}
%		
		
		
 
        
            
            
         \subsection{edu.kit.scc.pseworkflow.models.Data}   
	         Datenmodellierung der Input- und Outputdaten.
	         
	         Methoden:
	         \begin{center}
	         	\setlength\tabcolsep{5pt}
	         	\renewcommand{\arraystretch}{1.5}
	         	
	         	\begin{tabularx}{\textwidth}{|l|l|l|X|}
	         		\hline
	         		\rowcolor[gray]{0.75}[4.85pt]
	         		Typ & Name & Rückgabetyp & Beschreibung \\ \hline 
	         		&&& \\
	         		\hline
	         	\end{tabularx}
	         \end{center}
	         
	         Atttribute:
	         \begin{center}
	         	\setlength\tabcolsep{5pt}
	         	\renewcommand{\arraystretch}{1.5}
	         	
	         	\begin{tabularx}{\textwidth}{|l|l|l|X|}
	         		\hline
	         		\rowcolor[gray]{0.75}[4.85pt]
	         		Typ & Name & Rückgabetyp & Beschreibung \\ \hline 
	         		Attribut & id & int & Primärschlüssel in Datenbank, wird von Django automatisch generiert \\ \hline
	         		Attribut & task\_id & int & ID des zugehörigen Tasks\\ \hline
	         		Attribut & data\_IO & string & Identifikationswert, ob es sich bei dem Dateneintrag um einen Input oder einen Output handelt\\ \hline
	         		Attribut & data\_identifier & string & Eindeutige Kennung des Dateninputs bzw -outputs\\ \hline
	         		Attribut & data\_title & string & Anzeigetitel und Name des Datenobjekts\\ \hline
	         		Attribut & data\_abstract & string & Beschreibungstext des Datenobjekts\\ \hline
	         		Attribut & data\_type & string & Datentyp des Datenobjekts; Wert aus DataType Klasse \\ \hline
	         		Attribut & data\_min\_occurs & int & Minimale Anzahl an Vorkommen des gleichen Inputs oder Outputs \\ \hline
	         		Attribut & data\_max\_occurs & int & Maximale Anzahl an Vorkommen des gleichen Inputs oder Outputs \\ \hline
	         		Attribut & data\_value & string & Wert des Inputs oder Outputs als String. Alle Werte müssen vor der Ausführung des Tasks oder jeglicher weiterer Weiterverarbeitung explizit von string auf den benötigen Datentyp gecastet werden\\
	         		\hline
	         	\end{tabularx}
	         \end{center}
            
            
%        \subsection{edu.kit.scc.pseworkflow.View.Overview}
%        \subsection{edu.kit.scc.pseworkflow.View.EditorView}
%        \subsection{edu.kit.scc.pseworkflow.View.OverviewView}
%                    
%        \subsection{edu.kit.scc.pseworkflow.   .Workflow}    
%        \subsection{edu.kit.scc.pseworkflow.   .Task}
%        \subsection{edu.kit.scc.pseworkflow.   .WorkflowElement}
%        \subsection{edu.kit.scc.pseworkflow.   .Edge}
%        \subsection{edu.kit.scc.pseworkflow.   .User}
%        
%        \subsection{edu.kit.scc.pseworkflow.   .Database}
%        \subsection{edu.kit.scc.pseworkflow.   .DatabaseWorker}
%        
%        \subsection{edu.kit.scc.pseworkflow.   .Session}
%        \subsection{edu.kit.scc.pseworkflow.   .Client}
%        \subsection{edu.kit.scc.pseworkflow.   .Server}
	
	\section{Entwurfsmuster}
	die verwendeten entwurfsmuster sollen ja auch irgendwo erwähnt werden.
	also die identifikation von entwurfsmustern um struktur gröber zu beschreiben
	zb iterator für die dynamische einbindung der implementierten task klassen
	        