\chapter{Klassen und Pakete}

    \section{Server Klassen}

    \subsection{edu.kit.scc.pseworkflow.views}

		\subsubsection{Workflows\newline
		extends django.views.generic.list.ListView}

        Attribute:
        \begin{center}
        	\renewcommand{\arraystretch}{1.5}
            \setlength\tabcolsep{5pt}
        	\begin{tabularx}{\textwidth}{|l|l|l|X|}
        		\hline
                \rowcolor[gray]{0.75}[4.85pt]            		
        	    Typ & Name & Beschreibung \\ \hline
        	    Workflow & model & needed? \\ \hline
        		string & template_name & index.html \\ \hline
				% Attribut & identifier & string & Eindeutige Kennung des Workflows\\ \hline
				% Attribut & name & string & Anzeigetitel und Name des Workflows\\ \hline
				% Attribut & description & string & Beschreibungstext des Workflows\\ \hline				Attribut & tasks & Task & Liste der Tasks, die zu dem Workflow gehören\\ \hline
				% %Attribut & wf\_status & string & Ausführungsstatus des Workflows\\ \hline
				% Attribut & percent\_done & int & Fortschritt des Workflows in Prozent\\ \hline
				% Attribut & created\_at & Date & Erzeugungsdatum des Workflows \\ \hline
				% Attribut & updated\_at & Date & Datum der letzten Veränderung \\ \hline 
				% Attribut & creator & User & Benutzer, der den Workflow erstellt hat\\ \hline
				% %Attribut & wf\_shared\_with & string & Liste der Benutzer, die diesen Workflow %ebenfalls sehen und bearbeiten dürfen; Komma getrennte Liste\\ \hline
				% Attribut & last\_modifier & User & Benutzer, der am letzten den Workflow verändert hat\\ \hline
				
				%Attribut & wf\_executable & boolean & Hilfsattribut für Automatischen Workflow %Scheduler; Scheduler sendet den Workflow erst zur Ausführung, wenn wf\_executable = %true; Standardwert false \\				
				\hline            		
        	\end{tabularx}
        \end{center}

    
         Methoden:
        \begin{center}
        	\setlength\tabcolsep{5pt}
        	\renewcommand{\arraystretch}{1.5}
        	
        	\begin{tabularx}{\textwidth}{|l|l|l|X|}
        		\hline
        		\rowcolor[gray]{0.75}[4.85pt]
        		Typ & Name & Rückgabetyp & Beschreibung \\ \hline 
        		JSON & index & to  \\ \hline
        		Methode & save & void & Speichert die Daten in die Datenbank; Überschreibt die Standard Methode. In dieser Methode wird die Workflow enthaltende JSON Datei geparst und danach werden einzelne Elemente eines Workflows separat gespeichert. \\ 
        		\hline
        	\end{tabularx}
        \end{center}

		\subsubsection{Editor}

    \subsection{edu.kit.scc.pseworkflow.cron}

		\subsubsection{Cron}
		
		hier zwei Tabellen für 


    \subsection{edu.kit.scc.pseworkflow.models.workflows}

    \subsection{edu.kit.scc.pseworkflow.models.services}

    \subsection{edu.kit.scc.pseworkflow.models.sessions}

%%% KLASSE WORKFLOW
        \subsection{edu.kit.scc.pseworkflow.models.Workflow}
	        Diese Klasse beschreibt einen Workflow. Ein Workflow enthält einen oder mehrere Tasks. \newline
	        Diese Klasse erbt von der Django Klasse namens "Model".
                    
            Attribute:
            \begin{center}
            	\renewcommand{\arraystretch}{1.5}
	            \setlength\tabcolsep{5pt}
            	\begin{tabularx}{\textwidth}{|l|l|l|X|}
            		\hline
                    \rowcolor[gray]{0.75}[4.85pt]            		
            	    Typ & Name & Rückgabetyp & Beschreibung \\ \hline
            		Attribut & id & int & Primärschlüssel in der Datenbank. Wird von Django automatisch generiert \\ \hline
					Attribut & identifier & string & Eindeutige Kennung des Workflows\\ \hline
					Attribut & name & string & Anzeigetitel und Name des Workflows\\ \hline
					Attribut & description & string & Beschreibungstext des Workflows\\ \hline				Attribut & tasks & Task & Liste der Tasks, die zu dem Workflow gehören\\ \hline
					%Attribut & wf\_status & string & Ausführungsstatus des Workflows\\ \hline
					Attribut & percent\_done & int & Fortschritt des Workflows in Prozent\\ \hline
					Attribut & created\_at & Date & Erzeugungsdatum des Workflows \\ \hline
					Attribut & updated\_at & Date & Datum der letzten Veränderung \\ \hline 
					Attribut & creator & User & Benutzer, der den Workflow erstellt hat\\ \hline
					%Attribut & wf\_shared\_with & string & Liste der Benutzer, die diesen Workflow %ebenfalls sehen und bearbeiten dürfen; Komma getrennte Liste\\ \hline
					Attribut & last\_modifier & User & Benutzer, der am letzten den Workflow verändert hat\\ \hline
					
					%Attribut & wf\_executable & boolean & Hilfsattribut für Automatischen Workflow %Scheduler; Scheduler sendet den Workflow erst zur Ausführung, wenn wf\_executable = %true; Standardwert false \\				
					\hline            		
            	\end{tabularx}
            \end{center}

        
             Methoden:
	        \begin{center}
	        	\setlength\tabcolsep{5pt}
	        	\renewcommand{\arraystretch}{1.5}
	        	
	        	\begin{tabularx}{\textwidth}{|l|l|l|X|}
	        		\hline
	        		\rowcolor[gray]{0.75}[4.85pt]
	        		Typ & Name & Rückgabetyp & Beschreibung \\ \hline 
	        		Methode & save & void & Speichert die Daten in die Datenbank; Überschreibt die Standard Methode. In dieser Methode wird die Workflow enthaltende JSON Datei geparst und danach werden einzelne Elemente eines Workflows separat gespeichert. \\ 
	        		\hline
	        	\end{tabularx}
	        \end{center}
%%% ENDE DER KLASSE WORKFLOW
%%% KLASSE TASK
        \subsection{edu.kit.scc.pseworkflow.models.Task}	
    		Beschreibung eines einzelnen Tasks innerhalb eines Workflows, die in der Benutzeroberfläche als Knoten des Graphes dargestellt sind. Task ist eine Instanz eines WPS Prozesses, der von dem WPS Server zur Verfügung gestellt wird. Es kann mehrere Tasks geben, die später als derselbe WPS Prozess ausgeführt werden. \newline
    		Diese Klasse erbt von der Django Klasse namens "Model".
    		
    		Atttribute:
			\begin{center}
				\setlength\tabcolsep{5pt}
				\renewcommand{\arraystretch}{1.5}
				
				\begin{tabularx}{\textwidth}{|l|l|l|X|}
					\hline
					\rowcolor[gray]{0.75}[4.85pt]
					Typ & Name & Rückgabetyp & Beschreibung \\ \hline 
	           		Attribut & id & int & Primärschlüssel in der Datenbank. Wird von Django automatisch generiert \\ \hline
	           		Attribut & process & Process & Der zugehörende WPS Prozess, der von dem WPS Server zur Verfügung gestellt wird. \\\hline
	           		Attribut & x & int & X Koordinate des Tasks in der Benutzeroberfläche. Damit werden Tasks immer genau an der Stelle im Editor angezeigt, wo die gespeichert wurden\\ \hline
	           		Attribut & y & int & Y Koordinate des Tasks in der Benutzeroberfläche\\ \hline
%TODO TODO TODO
	           		Attribut & status & Status & Ausführungsstatus des Tasks. Siehe workflow.Status Klasse \\ \hline
	           		Attribut & input\_artefacts & InputArtefact[] & Liste der ??? \\ \hline
	           		Attribut & output\_artefacts & OutputArtefact[] & Liste der ??? \\ \hline
	           		Attribut & title & string & Titel des Tasks \\ \hline
	           		Attribut & abstract & string & Kurze Beschreibung des Tasks \\ \hline
	           		Attribut & status\_url & string & Url des Tasks auf dem WPS Server. Dient der  regelmäßigen Aktualisierung des Ausführungsstatus \\ \hline
	           		Attribut & started\_at & Date & Datum und die Uhrzeit des Ausführungstarts \\ \hline
				\end{tabularx}
			\end{center}
    			
			Methoden:
			\begin{center}
				\setlength\tabcolsep{5pt}
				\renewcommand{\arraystretch}{1.5}
				
				\begin{tabularx}{\textwidth}{|l|l|l|X|}
					\hline
					\rowcolor[gray]{0.75}[4.85pt]
					Typ & Name & Rückgabetyp & Beschreibung \\ \hline
					Methode & save & void & Speichert den Task in der Datenbank \\ 
					\hline
				\end{tabularx}
			\end{center}			
%%% ENDE DER KLASSE TASK
%%% KLASSE EDGE
        \subsection{edu.kit.scc.pseworkflow.models.Edge}	
    			Diese Klasse beschreibt eine Kante, die in der Benutzeroberfläche zwei Taskknoten miteinander verbindet. \\newline
    			Diese Klasse erbt von der Django Klasse namens "Model".
    			
    			Atttribute:
    			\begin{center}
    				\setlength\tabcolsep{5pt}
    				\renewcommand{\arraystretch}{1.5}
    				
    				\begin{tabularx}{\textwidth}{|l|l|l|X|}
    					\hline
    					\rowcolor[gray]{0.75}[4.85pt]
    					Typ & Name & Rückgabetyp & Beschreibung \\ \hline 
    	           		Attribut & id & int & Primärschlüssel in der Datenbank. Wird von Django automatisch generiert \\ \hline
    	           		Attribut & from\_task & Task & Ausgangsknoten \\ \hline
    	           		Attribut & to\_task & Task & Eingangsknoten \\ \hline
    	           		Attribut & input & Input & Die Eingangsdaten eines Tasks\\ \hline
    	           		Attribut & output & Output & Das Ergebnis eines Tasks \\
    	           		
    	           		\hline
    				\end{tabularx}
    			\end{center}
    			
    			Methoden:
    			\begin{center}
    				\setlength\tabcolsep{5pt}
    				\renewcommand{\arraystretch}{1.5}
    				
    				\begin{tabularx}{\textwidth}{|l|l|l|X|}
    					\hline
    					\rowcolor[gray]{0.75}[4.85pt]
    					Typ & Name & Rückgabetyp & Beschreibung \\ \hline
    					methode& save & void & Speichert die Kante in der Datenbank\\ 
    					\hline
    				\end{tabularx}
    			\end{center}
    			
    			
    
%%% ENDE DER KLASSE EDGE
%%% ENUMERATION EXECUtiON STATUS
		\subsection{edu.kit.scc.pseworkflow.models.Status (Enum)}	
			Beschreibt die möglichen Ausführungsstati von Workflows und Tasks. \newline
			
			Atttribute:
			\begin{center}
				\setlength\tabcolsep{5pt}
				\renewcommand{\arraystretch}{1.5}
				
				\begin{tabularx}{\textwidth}{|l|l|l|X|}
					\hline
					\rowcolor[gray]{0.75}[4.85pt]
					Typ & Name & Rückgabetyp & Beschreibung \\ \hline 
	           		Attribut & status\_value & string & Ausführungstatus\newline
	           		READY = Bereit zur Ausführung, wartet auf keine Inputs von anderen Tasks \newline
	           		RUNNING = Befindet sich gerade in der Ausführung auf einem WPS Server \newline 
	           		FINISHED = Workflow oder Task komplett ausgeführt und Ergebnis liegt vor \newline
	           		FAILED = Bei der Ausführung des Tasks ist ein Fehler aufgetreten \newline
	           		DEPRECATED = Der WPS Prozess steht aus eigenen Gründen nicht mehr auf dem WPS Server zur Verfügung \\
	           		\hline
				\end{tabularx}
			\end{center}
%%% ENDE DER ENUMERATION
%%% KLASSE PROCESS
		\subsection{edu.kit.scc.pseworkflow.models.Process}
			Diese Klasse beschreibt den WPS Prozess, der auf dem WPS Server als Code gespeichert ist. Alle Daten werden von dem WPS Server als XML Datei mittels HTTP Response (Describe Processes) empfangen und danach auf dem Django Server erstmal geparst und in der Datenbank gespeichert. \newline
			Diese Klasse erbt von der Django Klasse namens "Model".
			
			Atttribute:
			\begin{center}
				\setlength\tabcolsep{5pt}
				\renewcommand{\arraystretch}{1.5}
				
				\begin{tabularx}{\textwidth}{|l|l|l|X|}
					\hline
					\rowcolor[gray]{0.75}[4.85pt]
					Typ & Name & Rückgabetyp & Beschreibung \\ \hline 
					Attribut & id & int & Primärschlüssel in der Datenbank. Wird von Django automatisch generiert \\ \hline
					Attribut & identifier & String & Eindeutige Kennung des WPS Prozesses \\ \hline
					Attribut & inputs & Input[] &  Liste der Inputs, die für den WPS Prozess erforderlich sind\\ \hline
					Attribut & outputs & Output[] & Liste der Outputs, die für den WPS Prozess erforderlich sind \\ \hline
					Attribut & service & Service & Beschreibt den WPS Server, der den WPS Prozess zur Verfügung stellt   \\ \hline
					Attribut & title & string & Titel des WPS Prozesses \\ \hline
					Attribut & abstract & string & Beschreibung des WPS Prozesses\\
					\hline
				\end{tabularx}
			\end{center}
			
			Methoden:
			\begin{center}
				\setlength\tabcolsep{5pt}
				\renewcommand{\arraystretch}{1.5}
				
				\begin{tabularx}{\textwidth}{|l|l|l|X|}
					\hline
					\rowcolor[gray]{0.75}[4.85pt]
					Typ & Name & Rückgabetyp & Beschreibung \\ \hline 
					Methode & save & void & Speichert den WPS Prozess in der Datenbank  \\
					\hline
				\end{tabularx}
			\end{center}
			
			
%%% ENDE DER KLASSE PROCESS 
%%% KLASSE SERVICE

        \subsection{edu.kit.scc.pseworkflow.models.Service}
			Diese Klasse beschreibt den WPS Server, der die WPS Prozesse zur Verfügung stellt. \newline
			Diese Klasse erbt von der Django Klasse namens "Model".
			
			Atttribute:
			\begin{center}
				\setlength\tabcolsep{5pt}
				\renewcommand{\arraystretch}{1.5}
				
				\begin{tabularx}{\textwidth}{|l|l|l|X|}
					\hline
					\rowcolor[gray]{0.75}[4.85pt]
					Typ & Name & Rückgabetyp & Beschreibung \\ \hline 
					Attribut & id & int & Primärschlüssel in der Datenbank. Wird von Django automatisch generiert \\ \hline
					Attribut & service\_provider & ServiceProvider & Der Besitzer des WPS Servers \\ \hline
					Attribut & title & string & Name des WPS Servers \\ \hline
					Attribut & abstract & string & Beschreibung des WPS Servers \\ \hline
					Attribut & capabilities\_url & string & URL des WPS Servers für die GetCapabilities Operation \\ \hline
					Attribut & describe\_url & string & URL des WPS Servers für die DescribeProcess Operation \\ \hline
					Attribut & execute\_url & string & URL des WPS Servers für die Execute Operation \\ \hline
					
				\end{tabularx}
			\end{center}
			
			Methoden:
			\begin{center}
				\setlength\tabcolsep{5pt}
				\renewcommand{\arraystretch}{1.5}
				
				\begin{tabularx}{\textwidth}{|l|l|l|X|}
					\hline
					\rowcolor[gray]{0.75}[4.85pt]
					Typ & Name & Rückgabetyp & Beschreibung \\ \hline 
					Methode & save & void & Speichert den Service in der Datenbank \\
					\hline
				\end{tabularx}
			\end{center}

%%% ENDE DER KLASSE SERVICE
%%% KLASSE SERVICE PROVIDER
        \subsection{edu.kit.scc.pseworkflow.models.ServiceProvider}
			Beschreibt den Besitzer den WPS Servers. Obwohl man ganz viele optionale Attribute auf dem WPS Server eingeben kann, werden nur die wichtigste von denen in der Datenbank gespeichert. \newline
			Diese Klasse erbt von der Django Klasse namens \"Model".
			
			Atttribute:
			\begin{center}
				\setlength\tabcolsep{5pt}
				\renewcommand{\arraystretch}{1.5}
				
				\begin{tabularx}{\textwidth}{|l|l|l|X|}
					\hline
					\rowcolor[gray]{0.75}[4.85pt]
					Typ & Name & Rückgabetyp & Beschreibung \\ \hline 
					Attribut & id & int & Primärschlüssel in der Datenbank. Wird von Django automatisch generiert \\ \hline
					Attribut & provider\_name & string & Name des WPS Server Providers\\ \hline
					Attribut & provider\_site & string & Internetadresse des WPS Providers \\ \hline
					Attribut & individual\_name & string & Name der Person\\ \hline
					Attribut & position\_name & string & Stelle der Person \\ \hline
				\end{tabularx}
			\end{center}
			
			Methoden:
			\begin{center}
				\setlength\tabcolsep{5pt}
				\renewcommand{\arraystretch}{1.5}
				
				\begin{tabularx}{\textwidth}{|l|l|l|X|}
					\hline
					\rowcolor[gray]{0.75}[4.85pt]
					Typ & Name & Rückgabetyp & Beschreibung \\ \hline 
					Methode & save & void & Speicher den Service Provider in der Datenbank \\
					\hline
				\end{tabularx}
			\end{center}
			
%%% ENDE DER KLASSE SERVICE PROVIDER 
		\subsection{edu.kit.scc.pseworkflow.models.ProcessHandler}
			Die Namen der von den PyWPS-Servern bereitgestellten Prozesse, entspricht ProcessIdentifier im PyWPS GML.
			
			Methoden:
			\begin{center}
				\setlength\tabcolsep{5pt}
				\renewcommand{\arraystretch}{1.5}
				
				\begin{tabularx}{\textwidth}{|l|l|l|X|}
					\hline
					\rowcolor[gray]{0.75}[4.85pt]
					Typ & Name & Rückgabetyp & Beschreibung \\ \hline 
					&&& \\
					\hline
				\end{tabularx}
			\end{center}
			
			Atttribute:
			\begin{center}
				\setlength\tabcolsep{5pt}
				\renewcommand{\arraystretch}{1.5}
				
				\begin{tabularx}{\textwidth}{|l|l|l|X|}
					\hline
					\rowcolor[gray]{0.75}[4.85pt]
					Typ & Name & Rückgabetyp & Beschreibung \\ \hline 
					Attribut & id & int & Primärschlüssel in Datenbank, wird von Django automatisch generiert \\ \hline
					Attribut & handler\_name & string & Name der Klasse der Funktion, die der PyWPS-Server zur Verfügung stellt; entspricht ProcessIdentifier im PyWPS GML \\
					\hline
				\end{tabularx}
			\end{center}
		
		
		\subsection{edu.kit.scc.pseworkflow.models.Session}
			[[[Bitte ausfüllen von denen, die Ahnung davon haben]]]
			
			Methoden:
			\begin{center}
				\setlength\tabcolsep{5pt}
				\renewcommand{\arraystretch}{1.5}
				
				\begin{tabularx}{\textwidth}{|l|l|l|X|}
					\hline
					\rowcolor[gray]{0.75}[4.85pt]
					Typ & Name & Rückgabetyp & Beschreibung \\ \hline 
					&&& \\
					\hline
				\end{tabularx}
			\end{center}
			
			Atttribute:
			\begin{center}
				\setlength\tabcolsep{5pt}
				\renewcommand{\arraystretch}{1.5}
				
				\begin{tabularx}{\textwidth}{|l|l|l|X|}
					\hline
					\rowcolor[gray]{0.75}[4.85pt]
					Typ & Name & Rückgabetyp & Beschreibung \\ \hline 
					Attribut & id & int & Primärschlüssel in Datenbank, wird von Django automatisch generiert \\ \hline
					Attribut & user\_id & int & bitte ausfüllen \\ \hline
					Attribut & last\_workflow\_id & int & bitte ausfüllen \\
					\hline
				\end{tabularx}
			\end{center}
		
		
%		\subsection{edu.kit.scc.pseworkflow.models.User}
%			Datenmodell eines Users.
%			
%			Methoden:
%			\begin{center}
%				\setlength\tabcolsep{5pt}
%				\renewcommand{\arraystretch}{1.5}
%				
%				\begin{tabularx}{\textwidth}{|l|l|l|l|X|}
%					\hline
%					\rowcolor[gray]{0.75}[4.85pt]
%					Typ & Name & Rückgabetyp & Beschreibung \\ \hline 
%					Methode & save & void & Speichert die Daten in die Datenbank; Überschreibt die Standard Methode \\ \hline
%					Methode & comparePwHash & bool & Vergleicht den Hashwert eines eingegebenen Passworts mit dem gespeicherten Hashwert des Benutzerpassworts \\
%					\hline
%				\end{tabularx}
%			\end{center}
%			
%			Atttribute:
%			\begin{center}
%				\setlength\tabcolsep{5pt}
%				\renewcommand{\arraystretch}{1.5}
%				
%				\begin{tabularx}{\textwidth}{|l|l|l|l|X|}
%					\hline
%					\rowcolor[gray]{0.75}[4.85pt]
%					Typ & Name & Rückgabetyp & Beschreibung \\ \hline 
%					Attribut & id & int & Primärschlüssel in Datenbank, wird von Django automatisch generiert \\ \hline
%					Attribut & user\_login & string & Loginname des Benutzers \\
%					Attribut & user\_pw & string & Hashwert des Benutzerpassworts\\ \hline
%					Attribut & user\_name & string & Nachname des Benutzers \\ \hline
%					Attribut & user\_prename & string & Vorname des Benutzers \\ \hline
%					Attribut & user\_mail & string & Mailadresse des Benutzers \\ \hline
%					Attribut & user\_level & string & Rechtelevel des Benutzers; admin = Adminuser mit allen Konfigurationsrechten, normal = Normaler Benutzer ohne weitere Konfigurationsrechte \\
%					\hline
%				\end{tabularx}
%			\end{center}
%		
		
		
 
        
            
            
         \subsection{edu.kit.scc.pseworkflow.models.Data}   
	         Datenmodellierung der Input- und Outputdaten.
	         
	         Methoden:
	         \begin{center}
	         	\setlength\tabcolsep{5pt}
	         	\renewcommand{\arraystretch}{1.5}
	         	
	         	\begin{tabularx}{\textwidth}{|l|l|l|X|}
	         		\hline
	         		\rowcolor[gray]{0.75}[4.85pt]
	         		Typ & Name & Rückgabetyp & Beschreibung \\ \hline 
	         		&&& \\
	         		\hline
	         	\end{tabularx}
	         \end{center}
	         
	         Atttribute:
	         \begin{center}
	         	\setlength\tabcolsep{5pt}
	         	\renewcommand{\arraystretch}{1.5}
	         	
	         	\begin{tabularx}{\textwidth}{|l|l|l|X|}
	         		\hline
	         		\rowcolor[gray]{0.75}[4.85pt]
	         		Typ & Name & Rückgabetyp & Beschreibung \\ \hline 
	         		Attribut & id & int & Primärschlüssel in Datenbank, wird von Django automatisch generiert \\ \hline
	         		Attribut & task\_id & int & ID des zugehörigen Tasks\\ \hline
	         		Attribut & data\_IO & string & Identifikationswert, ob es sich bei dem Dateneintrag um einen Input oder einen Output handelt\\ \hline
	         		Attribut & data\_identifier & string & Eindeutige Kennung des Dateninputs bzw -outputs\\ \hline
	         		Attribut & data\_title & string & Anzeigetitel und Name des Datenobjekts\\ \hline
	         		Attribut & data\_abstract & string & Beschreibungstext des Datenobjekts\\ \hline
	         		Attribut & data\_type & string & Datentyp des Datenobjekts; Wert aus DataType Klasse \\ \hline
	         		Attribut & data\_min\_occurs & int & Minimale Anzahl an Vorkommen des gleichen Inputs oder Outputs \\ \hline
	         		Attribut & data\_max\_occurs & int & Maximale Anzahl an Vorkommen des gleichen Inputs oder Outputs \\ \hline
	         		Attribut & data\_value & string & Wert des Inputs oder Outputs als String. Alle Werte müssen vor der Ausführung des Tasks oder jeglicher weiterer Weiterverarbeitung explizit von string auf den benötigen Datentyp gecastet werden\\
	         		\hline
	         	\end{tabularx}
	         \end{center}

\newpage

    \section{Client Klassen}
    
        \subsection{services}
    
        \subsection{models}
    
    		\subsubsection{Workflow}
    		
    		Attribute:
                \begin{center}
                	\renewcommand{\arraystretch}{1.5}
    	            \setlength\tabcolsep{5pt}
                	\begin{tabularx}{\textwidth}{|l|l|X|}
                		\hline
                        \rowcolor[gray]{0.75}[4.85pt]	
                	    Typ & Name & Beschreibung \\ \hline
                		number & id &  \\ \hline
                		string & name &  \\ \hline
                		boolean & shared &  \\ \hline
                		string & description &  \\ \hline
                		number & creator_id &  \\ \hline
                		WorkflowVertex[] & vertices &  \\ \hline
                		WorkflowEdge[] & edges &  \\ \hline
                		number & created_timestamp &  \\ \hline
                		number & updated_timestamp &  \\ \hline
                	\end{tabularx}
                \end{center}
                
    		\subsubsection{WorkflowVertex}
    		
    		Attribute:
                \begin{center}
                	\renewcommand{\arraystretch}{1.5}
    	            \setlength\tabcolsep{5pt}
                	\begin{tabularx}{\textwidth}{|l|l|X|}
                		\hline
                        \rowcolor[gray]{0.75}[4.85pt]
                	    Typ & Name & Beschreibung \\ \hline
                		number & id &  \\ \hline
                		number & process_id &  \\ \hline
                		number & x &  \\ \hline
                		number & y &  \\ \hline
                		WorkflowVertexStatus & status &  \\ \hline
                		WorkflowVertex[] & vertices &  \\ \hline
                		WorkflowEdge[] & edges &  \\ \hline
                		WorkflowArtefact<'input'> & input_artefacts &  \\ \hline
                		WorkflowArtefact<'output'> & output_artefacts &  \\ \hline
                	\end{tabularx}
                \end{center}
                
    		\subsubsection{WorkflowEdge}
    		
    		Attribute:
                \begin{center}
                	\renewcommand{\arraystretch}{1.5}
    	            \setlength\tabcolsep{5pt}
                	\begin{tabularx}{\textwidth}{|l|l|X|}
                		\hline
                        \rowcolor[gray]{0.75}[4.85pt]
                	    Typ & Name & Beschreibung \\ \hline
                		number & id &  \\ \hline
                		WorkflowVertex & a &  \\ \hline
                		WorkflowVertex & b &  \\ \hline
                		number & input_id &  \\ \hline
                		number & output_id &  \\ \hline
                	\end{tabularx}
                \end{center}
                
    		\subsubsection{WorkflowArtefact<T>}
    		
    		Attribute:
                \begin{center}
                	\renewcommand{\arraystretch}{1.5}
    	            \setlength\tabcolsep{5pt}
                	\begin{tabularx}{\textwidth}{|l|l|X|}
                		\hline
                        \rowcolor[gray]{0.75}[4.85pt]
                	    Typ & Name & Beschreibung \\ \hline
                		number & id &  \\ \hline
                		number & workflow_id &  \\ \hline
                		number & paramerer_id &  \\ \hline
                		T & parameter_role &  \\ \hline
                		string & format &  \\ \hline
                		string & data &  \\ \hline
                		number & created_timestamp &  \\ \hline
                		number & updated_timestamp &  \\ \hline
                	\end{tabularx}
                \end{center}
                
    		\subsubsection{WorkflowVertexState}
    		
    		ENUM
                \begin{center}
                	\renewcommand{\arraystretch}{1.5}
    	            \setlength\tabcolsep{5pt}
                	\begin{tabularx}{\textwidth}{|l|X|}
                		\hline
                        \rowcolor[gray]{0.75}[4.85pt]
                	    Name & Beschreibung \\ \hline
                		WAITING &   \\ \hline
                		RUNNING &   \\ \hline
                		FINISHED  &  \\ \hline
                		FAILED  &  \\ \hline
                		DEPRECATED  &  \\ \hline
                	\end{tabularx}
                \end{center}
                
    		\subsubsection{Process}
    		
    		Attribute:
                \begin{center}
                	\renewcommand{\arraystretch}{1.5}
    	            \setlength\tabcolsep{5pt}
                	\begin{tabularx}{\textwidth}{|l|l|X|}
                		\hline
                        \rowcolor[gray]{0.75}[4.85pt]
                	    Typ & Name & Beschreibung \\ \hline
                		number & id &  \\ \hline
                		string & identifier & \\ \hline
                		string & title &  \\ \hline
                		string & abstract &  \\ \hline
                        ProcessParameter<'input'>[] & inputs &  \\ \hline
                        ProcessParameter<'output'>[] & outputs &  \\ \hline
                		number & wps_id &  \\ \hline
                		number & created_timestamp &  \\ \hline
                		number & updated_timestamp &  \\ \hline
                	\end{tabularx}
                \end{center}
                
    		\subsubsection{ProcessParameter<T>}
    		
    		Attribute:
                \begin{center}
                	\renewcommand{\arraystretch}{1.5}
    	            \setlength\tabcolsep{5pt}
                	\begin{tabularx}{\textwidth}{|l|l|X|}
                		\hline
                        \rowcolor[gray]{0.75}[4.85pt]
                	    Typ & Name & Beschreibung \\ \hline
                		number & id &  \\ \hline
                		T & type &  \\ \hline
                		ProcessParamenterType & paramerer_type &  \\ \hline
                		string & title &  \\ \hline
                		string & abstract &  \\ \hline
                		number & min_occurs &  \\ \hline
                		number & max_occurs &  \\ \hline
                	\end{tabularx}
                \end{center}
                
    		\subsubsection{ProcessParamenterType}
    		
    		ENUM
                \begin{center}
                	\renewcommand{\arraystretch}{1.5}
    	            \setlength\tabcolsep{5pt}
                	\begin{tabularx}{\textwidth}{|l|X|}
                		\hline
                        \rowcolor[gray]{0.75}[4.85pt]
                	    Name & Beschreibung \\ \hline
                		COMPLEX &   \\ \hline
                		LITERAL &   \\ \hline
                		BOUNDING_BOX  &  \\ \hline
                	\end{tabularx}
                \end{center}
                
    		\subsubsection{WPS}
    		
    		Attribute:
                \begin{center}
                	\renewcommand{\arraystretch}{1.5}
    	            \setlength\tabcolsep{5pt}
                	\begin{tabularx}{\textwidth}{|l|l|X|}
                		\hline
                        \rowcolor[gray]{0.75}[4.85pt]
                	    Typ & Name & Beschreibung \\ \hline
                		number & id &  \\ \hline
                		string & abstract &  \\ \hline
                		WPSProvider & provider &  \\ \hline
                	\end{tabularx}
                \end{center}
                
    		\subsubsection{WPSProvider}
    		
    		Attribute:
                \begin{center}
                	\renewcommand{\arraystretch}{1.5}
    	            \setlength\tabcolsep{5pt}
                	\begin{tabularx}{\textwidth}{|l|l|X|}
                		\hline
                        \rowcolor[gray]{0.75}[4.85pt]
                	    Typ & Name & Beschreibung \\ \hline
                		number & id &  \\ \hline
                		string & name &  \\ \hline
                		string & site &  \\ \hline
                	\end{tabularx}
                \end{center}
                
    		\subsubsection{User}
    		
    		Attribute:
                \begin{center}
                	\renewcommand{\arraystretch}{1.5}
    	            \setlength\tabcolsep{5pt}
                	\begin{tabularx}{\textwidth}{|l|l|X|}
                		\hline
                        \rowcolor[gray]{0.75}[4.85pt]
                	    Typ & Name & Beschreibung \\ \hline
                		number & id &  \\ \hline
                		UserGroup & group &  \\ \hline
                		string & username &  \\ \hline
                		string & first_name &  \\ \hline
                		string & last_name &  \\ \hline
                		string & email &  \\ \hline
                		number & created_timestamp &  \\ \hline
                		number & updated_timestamp &  \\ \hline
                	\end{tabularx}
                \end{center}
                
    		\subsubsection{UserGroup}
    		
    		ENUM
                \begin{center}
                	\renewcommand{\arraystretch}{1.5}
    	            \setlength\tabcolsep{5pt}
                	\begin{tabularx}{\textwidth}{|l|X|}
                		\hline
                        \rowcolor[gray]{0.75}[4.85pt]
                	    Name & Beschreibung \\ \hline
                		REGULAR &  \\ \hline
                		ADMIN &  \\ \hline
                	\end{tabularx}
                \end{center}
    
        \section{components}
	
	\section{Entwurfsmuster}
	die verwendeten entwurfsmuster sollen ja auch irgendwo erwähnt werden.
	also die identifikation von entwurfsmustern um struktur gröber zu beschreiben
	zb iterator für die dynamische einbindung der implementierten task klassen
	        