\chapter{Klassen und Pakete}

	\section{Server Klassen}

	\subsection{edu.kit.scc.pseworkflow.views}

		\subsubsection{WorkflowView}
		
		Beschreibt die Workflow-Schnittstelle für die REST-API.\newline

		Attribute:
		\begin{center}
			\renewcommand{\arraystretch}{1.5}
			\setlength\tabcolsep{5pt}
			\begin{tabularx}{\textwidth}{|l|l|X|}
				\hline
				\rowcolor[gray]{0.75}[4.85pt]					
				Name & Datentyp & Beschreibung \\ \hline
				model & Model & Model, die von Django automatisch für dieses View geladen wird \\ \hline		
			\end{tabularx}
		\end{center}
		
		Methoden:
		\begin{center}
			\setlength\tabcolsep{5pt}
			\renewcommand{\arraystretch}{1.5}
			
			\begin{tabularx}{\textwidth}{|l|l|p{30mm}|X|}
				\hline
				\rowcolor[gray]{0.75}[4.85pt]
				Name & Rückgabetyp & Parameter & Beschreibung \\ \hline 
				index & HTTPResponse & request: HTTPRequest & Gibt die HTML-Datei aus der Template-Datei zurück. Diese enthält eine Verlinkung auf Code, der auf den Client geladen und dort ausgeführt wird \\ \hline
				get & HTTPResponse & request: HTTPRequest\newline id: int & Sucht in der Datenbank nach dem Workflow mit der übergebenen ID und gibt ihn im JSON-Format zurück \\ \hline
				create & HTTPResponse & request: HTTPRequest & Erzeugt einen neuen Workflow und gibt seine ID zurück \\ \hline
				update & HTTPResponse & request: HTTPRequest\newline id: int & Speichert die Daten in die Datenbank; Überschreibt die Standard Methode. In dieser Methode wird die Workflow enthaltende JSON Datei geparst und danach werden einzelne Elemente eines Workflows separat gespeichert. \\ \hline
				delete & HTTPResponse & request: HTTPRequest\newline id: int & Löscht den Workflow mit der übergebenen ID \\ \hline
				start & HTTPResponse & request: HTTPRequest\newline id: int & Führt den Workflow mit der übergebenen ID aus \\ \hline
				stop & HTTPResponse & request: HTTPRequest\newline id: int & Stoppt die Ausführung des Workflows mit der übergebenen ID \\ \hline
			\end{tabularx}
		\end{center}
		
		\newpage

		\subsubsection{EditorView}
		
		Beschreibt die Seite in der der Nutzer seine Workflows bearbeiten kann.\newline
		
		Attribute:
		\begin{center}
			\renewcommand{\arraystretch}{1.5}
			\setlength\tabcolsep{5pt}
			\begin{tabularx}{\textwidth}{|l|l|X|}
				\hline
				\rowcolor[gray]{0.75}[4.85pt]					
				Name & Datentyp & Beschreibung \\ \hline
				template_name & string & Name der Template-Datei \\ \hline		
			\end{tabularx}
		\end{center}
		
		\subsubsection{SettingsView}
		
		Beschreibt die Seite in der der Nutzer WPS Server eintragen und löschen kann sowie verfügbare Processe davon in das System synchronisieren.\newline
		
		Attribute:
		\begin{center}
			\renewcommand{\arraystretch}{1.5}
			\setlength\tabcolsep{5pt}
			\begin{tabularx}{\textwidth}{|l|l|X|}
				\hline
				\rowcolor[gray]{0.75}[4.85pt]					
				Name & Datentyp & Beschreibung \\ \hline
				template_name & string & Name der Template-Datei \\ \hline		
			\end{tabularx}
		\end{center}

		\subsubsection{WorkflowsView}
		
		Beschreibt die Seite in der der Nutzer eine Liste seiner Workflows sieht.\newline

		Attribute:
		\begin{center}
			\renewcommand{\arraystretch}{1.5}
			\setlength\tabcolsep{5pt}
			\begin{tabularx}{\textwidth}{|l|l|X|}
				\hline
				\rowcolor[gray]{0.75}[4.85pt]					
				Name & Datentyp & Beschreibung \\ \hline
				template\_name & string & Name der Template Datei \\ \hline				
			\end{tabularx}
		\end{center}
		
		\subsubsection{UserView}
		
		Bietet Schnittstelle, durch die der Client abfragen kann, ob der Besucher eingeloggt ist.\newline
		
		Methoden:
		\begin{center}
			\setlength\tabcolsep{5pt}
			\renewcommand{\arraystretch}{1.5}
				\begin{tabularx}{\textwidth}{|l|l|l|X|}
				\hline
				\rowcolor[gray]{0.75}[4.85pt]
				Name & Rückgabetyp & Parameter & Beschreibung \\ \hline 
				index & HTTPResponse & request: HTTPRequest & Gibt die Accountdaten des eingeloggten Users oder eine Meldung, dass User nicht eingeloggt ist zurück \\ \hline
				\end{tabularx}
		\end{center}
		
		\newpage
		
		\subsubsection{ProcessView}
		
		Bietet Schnittstelle, durch die der Client die Processe abfragen und verwalten kann.\newline
		
		Methoden:
		\begin{center}
			\setlength\tabcolsep{5pt}
			\renewcommand{\arraystretch}{1.5}
				\begin{tabularx}{\textwidth}{|l|l|p{30mm}|X|}
				\hline
				\rowcolor[gray]{0.75}[4.85pt]
				Name & Rückgabetyp & Parameter & Beschreibung \\ \hline 
				index & HTTPResponse & request: HTTPRequest & Gibt eine Liste aller verfügbaren Processes zurück \\ \hline
				get & HTTPResponse & request: HTTPRequest\newline id: int & Gibt Details des Processes mit übergegebenem ID zurück \\ \hline
				create & HTTPResponse & request: HTTPRequest & Erstellt einen neuen Process und gibt seine ID in der Response zurück \\ \hline
				update & HTTPResponse & request: HTTPRequest\newline id: int & Speichert die in der Request übergegebenen Daten im Process mit der übergegebenen ID \\ \hline
				delete & HTTPResponse & request: HTTPRequest\newline id: int & Löscht den Process mit der übergegebenen ID \\ \hline
				\end{tabularx}
		\end{center}
		
		\newpage
		
		\subsubsection{WPSView}
		
		Bietet Schnittstelle, durch die der Client die WPS Server abfragen und verwalten kann.\newline
		
		Methoden:
		\begin{center}
			\setlength\tabcolsep{5pt}
			\renewcommand{\arraystretch}{1.5}
				\begin{tabularx}{\textwidth}{|l|l|p{30mm}|X|}
				\hline
				\rowcolor[gray]{0.75}[4.85pt]
				Name & Rückgabetyp & Parameter & Beschreibung \\ \hline 
				index & HTTPResponse & request: HTTPRequest & Gibt eine Liste aller verfügbarer WPS Server zurück \\ \hline
				get & HTTPResponse & request: HTTPRequest\newline id: int & Gibt Details des WPS mit der übergegebenen ID zurück \\ \hline
				create & HTTPResponse & request: HTTPRequest & Erstellt einen neuen WPS Server und gibt seine ID in der Response zurück \\ \hline
				update& HTTPResponse & request: HTTPRequest\newline id: int & Speichert die in der Request übergegebenen Daten im WPS Server mit der übergegebenen ID \\ \hline
				delete & HTTPResponse & request: HTTPRequest\newline id: int & Löscht den WPS Server mit der übergegebenen ID \\ \hline
				refresh & HTTPResponse & request: HTTPRequest\newline id: int & Lädt eine Liste aller verfügbarer Processes von dem WPS Server mit der übergegebenen ID und aktualisiert Processes in unserer Datenbank, wenn Änderungen vorliegen. Das erfolgt so:\newline Liste der Prozessen vom Server wird mit dieser aus unserer Datenbank verglichen \\ \hline
				\end{tabularx}
		\end{center}

	\subsection{edu.kit.scc.pseworkflow.cron}

%%% KLASSE CRON
		\subsubsection{WorkflowJob}
		
	Hier wird die externe Bibliothek django-cron verwendet, und diese Klasse überschreibet die Klasse CronJobBase.\newline
		Die Klasse erledigt Aufgaben, die ständig im Hintergrund laufen und nicht vom Benutzer gesteuert werden. Methoden in dieser Klasse werden in regelmäßigen Zeitabständen von \gls{Cron} aufgerufen (z.B. einmal pro Minute). Dazu gehören:
		\begin{itemize}
			\item Für jeden Laufenden Task (status = RUNNING) sein Status bei zugehörigem WPS-Server abfragen und, falls er geändert wurde, Änderungen in die Datenbank schreiben
			\item Alle noch nicht ausgeführte Tasks durchgehen und an den zugehörigen WPS Server schicken, falls er frei ist
		\end{itemize}
		
		An der Stelle lohnt es sich zu erwähnen, welche Bedeutung die Statusen der Tasks für Auführung der weiteren Tasks haben:
		
		\begin{itemize}
			\item READY - um an den WPS-Server abgeschickt werden zu dürfen, müssen die Tasks diesen Status haben. Sie kriegen den, falls
			\begin{itemize}
				\item sie am Anfang des Workflows stehen und keine Tasks vor sich haben, oder
				\item wenn vorheriger Task auf FINISHED gesetzt wird. Wenn es mehrere vorherige Tasks gibt, passiert es dann, wenn letzte von allen zu FINISHED wird
			\end{itemize}
			\item WAITING - der Task wartet auf Ergebnisse von vorherigen Tasks
			\item RUNNING - der Task wird jetzt ausgeführt auf einem WPS-Server
			\item FINISHED - der Task wurde erfolgreich auf einem WPS-Server beendet. Ggf. nächste Tasks können vom Status WAITING auf READY gesetzt werden
			\item FAILED - Alle Tasks, die Output von FAILED Task als Input bekommen, werden nicht ausgeführt
			\item DEPRECATED - Wenn der Workflow mindestens einen Task mit dem Status DEPRECATED enthält, wird seine Ausführung gestoppt, d. h. keine weitere Tasks werden an den WPS-Server geschickt
		\end{itemize}
		
		Attribute:
		\begin{center}
			\renewcommand{\arraystretch}{1.5}
			\setlength\tabcolsep{5pt}
			\begin{tabularx}{\textwidth}{|l|l|X|}
				\hline
				\rowcolor[gray]{0.75}[4.85pt]					
				Name & Datentyp & Beschreibung \\ \hline
				schedule & Schedule & Klasse von django-cron Bibliothek, beschreibt, wann der Job ausgeführt werden muss \\ \hline
				code & string & Code dieses Cron-Jobs \\ \hline
			\end{tabularx}
		\end{center}
		
		Methoden:
		\begin{center}
			\setlength\tabcolsep{5pt}
			\renewcommand{\arraystretch}{1.5}
			
			\begin{tabularx}{\textwidth}{|l|l|l|X|}
				\hline
				\rowcolor[gray]{0.90}[4.85pt]
				Name & Rückgabetyp & Parameter & Beschreibung \\ \hline
				do & void & - & Überschreibt die Metode in der Parentklasse. Hier passiert alles, was in der Klassenbeschreibung steht \\ \hline
			\end{tabularx}
		\end{center}

%%% ENDE DER KLASSE CRON

	\newpage
	
%%% MODELS

%%% KLASSE WORKFLOW
		\subsection{edu.kit.scc.pseworkflow.models}
			\subsubsection{Workflow}
			Diese Klasse beschreibt einen \gls{Workflow}. Ein Workflow enthält einen oder mehrere \gls{Task}s. \newline
			Diese Klasse erbt von der Django Klasse namens \glqq Model\grqq.\newline
					
			Attribute:
			\begin{center}
				\renewcommand{\arraystretch}{1.5}
				\setlength\tabcolsep{5pt}
				\begin{tabularx}{\textwidth}{|l|l|X|}
					\hline
					\rowcolor[gray]{0.75}[4.85pt]					
					Name & Datentyp & Beschreibung \\ \hline
					id & int & Primärschlüssel in der Datenbank. Wird von Django automatisch generiert \\ \hline
					name & string & Anzeigetitel und Name des Workflows\\ \hline
					description & string & Beschreibungstext des Workflows\\ \hline	
					percent\_done & int & Fortschritt des Workflows in Prozent\\ \hline
					created\_at & Date & Erzeugungsdatum des Workflows \\ \hline
					updated\_at & Date & Datum der letzten Veränderung \\ \hline 
					creator & User & Benutzer der den Workflow erstellt hat\\ \hline
					last\_modifier & User & Benutzer der zuletzten den Workflow verändert hat\\ \hline
				\end{tabularx}
			\end{center}

		
			 Methoden:
			\begin{center}
				\setlength\tabcolsep{5pt}
				\renewcommand{\arraystretch}{1.5}
				
				\begin{tabularx}{\textwidth}{|l|l|l|X|}
					\hline
					\rowcolor[gray]{0.75}[4.85pt]
					Name & Rückgabetyp & Parameter & Beschreibung \\ \hline 
					save & void & keine & Speichert die Daten in die Datenbank; Überschreibt die Standard Methode. In dieser Methode wird die Workflow enthaltende JSON Datei geparst und danach werden einzelne Elemente eines Workflows separat gespeichert. \\ 
					\hline
				\end{tabularx}
			\end{center}
%%% ENDE DER KLASSE WORKFLOW
%%% KLASSE SESSION

		\subsubsection{Session}
		
		Diese Klasse speichert die letzten Workflow den ein Nutzer bearbeitet hat.
		
		Attribute:
		\begin{center}
			\renewcommand{\arraystretch}{1.5}
			\setlength\tabcolsep{5pt}
			\begin{tabularx}{\textwidth}{|l|l|X|}
				\hline
				\rowcolor[gray]{0.75}[4.85pt]					
				Name & Datentyp & Beschreibung \\ \hline
				id & int & Die ID der Session \\ \hline
				user & User & Der zugehörige User \\ \hline
				last_workflow & Workflow & Der Workflow der in der Session erstellt wurde \\ \hline
			\end{tabularx}
		\end{center}
%%% ENDE DER KLASSE SESSION

\newpage

%%% KLASSE TASK
		\subsubsection{Task}
			Beschreibung eines einzelnen Tasks innerhalb eines Workflows, die in der Benutzeroberfläche als Knoten des Graphes dargestellt sind. Task ist eine Instanz eines WPS Prozesses, der von dem WPS Server zur Verfügung gestellt wird. Es kann mehrere Tasks geben, die später als derselbe WPS Prozess ausgeführt werden. \newline
			Diese Klasse erbt von der Django Klasse namens \glqq Model\grqq .
			
			Attribute:
			\begin{center}
				\setlength\tabcolsep{5pt}
				\renewcommand{\arraystretch}{1.5}
				
				\begin{tabularx}{\textwidth}{|l|l|X|}
					\hline
					\rowcolor[gray]{0.75}[4.85pt]
					Name & Datentyp & Beschreibung \\ \hline 
			   		id & int & Primärschlüssel in der Datenbank. Wird von Django automatisch generiert \\ \hline
			   		workflow & Workflow & Workflow, dessen Tasks die Kante verbindet \\ \hline
			   		process & Process & Der zugehörende WPS Prozess, der von dem WPS Server zur Verfügung gestellt wird. \\\hline
			   		x & int & X Koordinate des Tasks in der Benutzeroberfläche. Damit werden Tasks immer genau an der Stelle im Editor angezeigt, wo sie gespeichert wurden\\ \hline
			   		y & int & Y Koordinate des Tasks in der Benutzeroberfläche\\ \hline
%TODO TODO TODO
			   		status & Status & Ausführungsstatus des Tasks. Siehe workflow.Status Klasse \\ \hline
			   		title & string & Titel des Tasks \\ \hline
			   		abstract & string & Kurze Beschreibung des Tasks \\ \hline
			   		status\_url & string & Url des Tasks auf dem WPS Server. Dient der  regelmäßigen Aktualisierung des Ausführungsstatus \\ \hline
			   		started\_at & Date & Datum und die Uhrzeit des Ausführungstarts \\ \hline
				\end{tabularx}
			\end{center}			
%%% ENDE DER KLASSE TASK

\newpage

%%% ENUMERATION EXECUTION STATUS
		\subsubsection{Status (Enum)}	
			Beschreibt die möglichen Ausführungsstatus von Tasks. \newline
			
			\begin{center}
				\renewcommand{\arraystretch}{1.5}
				\setlength\tabcolsep{5pt}
				\begin{tabularx}{\textwidth}{|l|X|}
					\hline
					\rowcolor[gray]{0.75}[4.85pt]
					Name & Beschreibung \\ \hline
					READY & Der Task ist bereit ausgeführt zu werden \\ \hline
					WAITING & In diesem Zustand wartet ein Task auf den Input eines noch nicht beendeten Tasks \\ \hline
					RUNNING & Der Task ist momentan in Bearbeitung \\ \hline
					FINISHED & Der Task wurde erfolgreich ausgeführt \\ \hline
					FAILED & Bei der Ausführung des Tasks ist ein Fehler aufgetreten \\ \hline
					DEPRECATED & Der Task gehört zum Process, der von dem WPS-Server gelöscht wurde, und sollte deswegen aus dem Workflow rausgenommen bzw. ersetzt werden \\ \hline
				\end{tabularx}
			\end{center}
%%% ENDE DER ENUMERATION EXECUTION STATUS
%%% KLASSE EDGE
		\subsubsection{Edge}	
				Diese Klasse beschreibt eine Kante, die in der Benutzeroberfläche zwei Taskknoten miteinander verbindet. \\newline
				Diese Klasse erbt von der Django Klasse namens \glqq Model\grqq .\newline
				
				Attribute:
				\begin{center}
					\setlength\tabcolsep{5pt}
					\renewcommand{\arraystretch}{1.5}
					
					\begin{tabularx}{\textwidth}{|l|l|X|}
						\hline
						\rowcolor[gray]{0.75}[4.85pt]
						Name & Rückgabetyp & Beschreibung \\ \hline 
				   		id & int & Primärschlüssel in der Datenbank. Wird von Django automatisch generiert \\ \hline
				   		workflow & Workflow & Workflow, dessen Tasks die Kante verbindet \\ \hline
				   		from\_task & Task & Ausgangsknoten \\ \hline
				   		to\_task & Task & Eingangsknoten \\ \hline
				   		input & Input & Die Eingangsdaten eines Tasks\\ \hline
				   		output & Output & Das Ergebnis eines Tasks \\
				   		
				   		\hline
					\end{tabularx}
				\end{center}
%%% ENDE DER KLASSE EDGE

\newpage

%%% KLASSE PROCESS
		\subsubsection{Process}
			Diese Klasse beschreibt den WPS Prozess, der auf dem WPS Server als Code gespeichert ist. Alle Daten werden von dem WPS Server als XML Datei mittels HTTP Response (Describe Processes) empfangen und danach auf dem Django Server erstmal geparst und danach in der Datenbank gespeichert. \newline
			Diese Klasse erbt von der Django Klasse namens \glqq Model\grqq .\newline
			
			Attribute:
			\begin{center}
				\setlength\tabcolsep{5pt}
				\renewcommand{\arraystretch}{1.5}
				
				\begin{tabularx}{\textwidth}{|l|l|X|}
					\hline
					\rowcolor[gray]{0.75}[4.85pt]
					Name & Rückgabetyp & Beschreibung \\ \hline 
					id & int & Primärschlüssel in der Datenbank. Wird von Django automatisch generiert \\ \hline
					identifier & String & Eindeutige Kennung des WPS Prozesses \\ \hline
					wps & WPS & Beschreibt den WPS Server, der den WPS Prozess zur Verfügung stellt \\ \hline
					title & string & Titel des WPS Prozesses \\ \hline
					abstract & string & Beschreibung des WPS Prozesses\\
					\hline
				\end{tabularx}
			\end{center}
%%% ENDE DER KLASSE PROCESS 
%%% KLASSE WPS
		\subsubsection{WPS}
			Diese Klasse beschreibt den WPS Server, der die WPS Prozesse zur Verfügung stellt. \newline
			Diese Klasse erbt von der Django Klasse namens \glqq Model\grqq .\newline
			
			Attribute:
			\begin{center}
				\setlength\tabcolsep{5pt}
				\renewcommand{\arraystretch}{1.5}
				
				\begin{tabularx}{\textwidth}{|l|l|X|}
					\hline
					\rowcolor[gray]{0.75}[4.85pt]
					Name & Rückgabetyp & Beschreibung \\ \hline 
					id & int & Primärschlüssel in der Datenbank. Wird von Django automatisch generiert \\ \hline
					wps\_provider & WPSProvider & Der Besitzer des WPS Servers \\ \hline
					title & string & Name des WPS Servers \\ \hline
					abstract & string & Beschreibung des WPS Servers \\ \hline
					capabilities\_url & string & URL des WPS Servers für die GetCapabilities Operation \\ \hline
					describe\_url & string & URL des WPS Servers für die DescribeProcess Operation \\ \hline
					execute\_url & string & URL des WPS Servers für die Execute Operation \\ \hline
					
				\end{tabularx}
			\end{center}
%%% ENDE DER KLASSE WPS

\newpage

%%% KLASSE WPSPROVIDER
		\subsubsection{WPSProvider}
			Beschreibt den Besitzer des WPS Servers. Obwohl man mehrere optionale Attribute auf dem WPS Server eingeben kann, werden nur die wichtigsten davon in der Datenbank gespeichert. \newline
			Diese Klasse erbt von der Django Klasse namens \glqq Model\grqq .\newline
			
			Attribute:
			\begin{center}
				\setlength\tabcolsep{5pt}
				\renewcommand{\arraystretch}{1.5}
				
				\begin{tabularx}{\textwidth}{|l|l|X|}
					\hline
					\rowcolor[gray]{0.75}[4.85pt]
					Name & Rückgabetyp & Beschreibung \\ \hline 
					id & int & Primärschlüssel in der Datenbank. Wird von Django automatisch generiert \\ \hline
					provider\_name & string & Name des WPS Server Providers\\ \hline
					provider\_site & string & Internetadresse des WPS Providers \\ \hline
					individual\_name & string & Name der Person\\ \hline
					position\_name & string & Stelle der Person \\ \hline
				\end{tabularx}
			\end{center}
%%% ENDE DER KLASSE WPSPROVIDER

%%% KLASSE INPUTOUTPUT
		\subsubsection{<<abstract>> InputOutput}
			Beschreibt die Inputs und Outputs eines WPS Prozesses. Die Klasse ist abstrakt, dass heißt, dass es keine Tabelle namens InputOutput von Django erzeugt wird. \newline
			Diese Klasse erbt von der Django Klasse namens \glqq Model\grqq .\newline
			
			Attribute:
			\begin{center}
				\setlength\tabcolsep{5pt}
				\renewcommand{\arraystretch}{1.5}
				
				\begin{tabularx}{\textwidth}{|l|l|X|}
					\hline
					\rowcolor[gray]{0.75}[4.85pt]
					Name & Rückgabetyp & Beschreibung \\ \hline 
					id & int & Primärschlüssel in der Datenbank. Wird von Django automatisch generiert \\ \hline
					process & Process & WPS Prozess, zu dem die Inputs bzw. Outputs gehören \\ \hline
					role & Role & Stellt fest, ob die Instanz ein Input oder Output ist \\ \hline
					identifier & string & Eindeutige Kennung des Inputs oder Outputs \\ \hline
					title & string & Titel des Inputs oder Outputs \\ \hline
					abstract & string & Kurze Beschreibung des Inputs oder Outputs \\ \hline
					datatype & Datatype & Bestimmt den Datentyp des Inputs oder Outputs. 
%%%% TODO  TODO TODO
					Siehe .... \\ \hline
					min\_occurs & int & Minimales Vorkommen eines Inputs oder Outputs \\ \hline
					max\_occurs & int & Maximales Vorkommen eines Inputs oder Outputs \\ \hline
				\end{tabularx}
			\end{center}
			
%%% ENDE DER KLASSE INPUTOUTPUT

\newpage

%%% KLASSE INPUT
		\subsubsection{Input}
			Beschreibt den Input (Eingangsdaten) des WPS Prozesses. \newline
			Diese Klasse erbt die Attribute von der abstrakten Klasse InputOutput (siehe oben).

%%% ENDE DER KLASSE INPUT
%%% KLASSE OUTPUT 
		\subsubsection{Ouput}
			Beschreibt den Output (Endergebnis) des WPS Prozesses. \newline
			Diese Klasse erbt die Attribute von der abstrakten Klasse InputOutput (siehe oben).
 
%%% ENDE DER KLASSE OUTPUT
%%% KLASSE DATATYPE (Enumeration)
		\subsubsection{Datatype (Enum)}	
			Beschreibt die möglichen Datentypen des WPS Prozesses. \newline
			
			Attribute:
			\begin{center}
				\setlength\tabcolsep{5pt}
				\renewcommand{\arraystretch}{1.5}
				
				\begin{tabularx}{\textwidth}{|l|X|}
					\hline
					\rowcolor[gray]{0.75}[4.85pt] 
					Name & Beschreibung \\ \hline 
			   		LITERAL &  Literal Data ist ein String, normalerweise kurz. Es wird verwendet um numerische oder textuelle Parameter zu übergeben.   \\ \hline
			   		COMPLEX & Complex Data sind normalerweise die Raster- oder Vektorgrafiken, es können aber auch beliebige dateibasierte Daten sein \\ \hline
			   		BOUNDING_BOX & Koordinatenpaaren für 2D oder 3D Räume \\ \hline
				\end{tabularx}
			\end{center}

%%% ENDE DER KLASSE  DATATYPE
%%% ENUMERATION ROLE
		\subsubsection{Role (Enum)}
			Beschreibt die möglichen Werte des Attributs role der abstrakten InputOutput Klasse. \newline
			
			Attribute:
			\begin{center}
				\setlength\tabcolsep{5pt}
				\renewcommand{\arraystretch}{1.5}
				
				\begin{tabularx}{\textwidth}{|l|X|}
					\hline
					\rowcolor[gray]{0.75}[4.85pt]
					Name & Beschreibung \\ \hline 
			   		INPUT & Eingabedaten eine WPS Prozesses \\ \hline
			   		OUTPUT & Endergebnis eines WPS Prozesses \\ \hline
				\end{tabularx}
			\end{center}
%%% ENDE DER ENUMERATION ROLE

\newpage

%%% KLASSE ARTEFACT
		\subsubsection{<<abstract>> Artefact}
			Beschreibt die eigentlichen Eingangsdaten oder die Endergebnise eines Tasks. Klasse ist abstrakt, das heißt, dass es keine Tabelle in der Datenbank namens \glqq Artefact \grqq erzeugt wird. \newline 
			Diese Klasse erbt von der Django Klasse namens \glqq Model\grqq .\newline
			
			Attribute:
			\begin{center}
				\setlength\tabcolsep{5pt}
				\renewcommand{\arraystretch}{1.5}
				
				\begin{tabularx}{\textwidth}{|l|l|X|}
					\hline
					\rowcolor[gray]{0.75}[4.85pt]
					Name & Rückgabetyp & Beschreibung \\ \hline 
					id & int & Primärschlüssel in der Datenbank. Wird von Django automatisch generiert \\ \hline
					task & Task & Der Task, zu dem die Inputs oder Outputs gehören\\ \hline
					parameter\_id & int & Identifikator des Parameters \\ \hline
					role & Role & Stellt fest, ob die Instanz ein Input oder Output ist \\ \hline
					format & string & Format des Inputs oder Outputs \\ \hline
					data & string & Der Wert dieses Attributs hängt von dem Datentyp ab. Falls der Datentyp Literal Data ist wird ein String mit dem Ergebnis gespeichert. Falls der Datentyp Complex Data oder Bounding Box Data ist, also falls das Ergebnis zu groß ist, um es in der Datenbank zu speichern, wird in diesem Attribut eine URL gespeichert die zum Ergebnis auf dem WPS Server führt \\ \hline
					created\_at & datetime & Erstellungsdatum des Inputs/Outputs \\ \hline
					updated\_at & datetime & Datum und die Uhrzeit der letzten Änderung \\ \hline
				\end{tabularx}
			\end{center}
%%% ENDE DER KLASSE ARTEFACT
%%% KLASSE INPUTARTEFACT
		\subsubsection{InputArtefact}
			Beschreibt den die Eingangsdaten des Tasks. \newline
			Diese Klasse erbt alle Attribute von der abstrakten Klasse Artefact (siehe oben).
			
%%%ENDE DER KLASSE INPUTARTEFACT
%%% KLASSE OUTPUTARTEFACT 
		\subsubsection{OutputArtefact}
			Beschreibt das Endergebnis des Tasks. \newline
			Diese Klasse erbt alle Attribute von der abstrakten Klasse Artefact (siehe oben).
			
%%% ENDE DER KLASSE OUTPUTARTEFACT


%##########
%# Client
%##########

\newpage

	\section{Client Klassen}
	
		\subsection{services}
		
			\subsubsection{AuthService}
			
			Beschreibt den Authentifizierungsservice für Nutzer. Die Klasse ist zuständig für login und logout.\newline
			
			Attribute:
			\begin{center}
				\renewcommand{\arraystretch}{1.5}
				\setlength\tabcolsep{5pt}
				\begin{tabularx}{\textwidth}{|l|l|X|}
					\hline
					\rowcolor[gray]{0.75}[4.85pt]					
					Name & Datentyp & Beschreibung \\ \hline
					auth & Observable<User> & Ein Observable mit dem zugehörigen Nutzer \\ \hline
				\end{tabularx}
			\end{center}
			
			Methoden:
			\begin{center}
				\setlength\tabcolsep{5pt}
				\renewcommand{\arraystretch}{1.5}
					\begin{tabularx}{\textwidth}{|l|l|p{35mm}|X|}
					\hline
					\rowcolor[gray]{0.75}[4.85pt]
					Name & Rückgabetyp & Parameter & Beschreibung \\ \hline 
					constructor & - & HTTPClient & Konstruktor \\ \hline
					logout & Promise<boolean> & - & Nutzer-logout \\ \hline
					login & Promise<boolean> & email: string\newline password: string & Nutzer-login \\ \hline
					\end{tabularx}
			\end{center}
			
			
			\subsubsection{WPSService}
			
			Beschreibt die Schnittstelle für den Zugriff auf WPS-Services.\newline
			
			Methoden:
			\begin{center}
			\setlength\tabcolsep{5pt}
				\renewcommand{\arraystretch}{1.5}
					\begin{tabularx}{\textwidth}{|l|l|p{35mm}|X|}
					\hline
					\rowcolor[gray]{0.75}[4.85pt]
					Name & Rückgabetyp & Parameter & Beschreibung \\ \hline 
					constructor & - & HTTPClient & Konstruktor\\ \hline
					all & Observable<WPS[]> & - & Ein Observable mit allen WPS Services\\ \hline
					get & Observable<WPS> & id: number & Gibt ein Observable mit dem WPS Service mit der übergebenen ID zurück \\ \hline
					create & Observable<WPS> & process: Process & Erzeugt ein Observable mit einem WPS Service mit dem übergebenen Prozess \\ \hline
					remove & Promise<boolean> & id: number & Entfernt den WPS Service mit der übergebenen ID \\ \hline
					update & Observable<WPS> & id: number\newline Partial<WPS> & Aktualisiert den WPS Service mit der übergebenen ID mit dem zweiten übergebenen Parameter \\ \hline
					\end{tabularx}
			\end{center}
			
			\subsubsection{WorkflowService}
			
			Beschreibt die Schnittstelle für den Zugriff auf Workflows.\newline
				
			Methoden:
			\begin{center}
			\setlength\tabcolsep{5pt}
				\renewcommand{\arraystretch}{1.5}
					\begin{tabularx}{\textwidth}{|l|l|p{35mm}|X|}
					\hline
					\rowcolor[gray]{0.75}[4.85pt]
					Name & Rückgabetyp & Parameter & Beschreibung \\ \hline 
					constructor & - & HTTPClient & Konstruktor \\ \hline
					all & Observable<Workflow[]> & - & Ein Observable mit allen Workflows\\ \hline
					get & Observable<Workflow> & id: number & Gibt ein Observable mit dem Workflow mit der übergebenen ID zurück \\ \hline
					create & Observable<Workflow> & workflow: Workflow & Erzeugt ein Observable mit dem übergebenen Workflow \\ \hline
					remove & Promise<boolean> & id: number & Entfernt den Workflow mit der übergebenen ID \\ \hline
					update & Observable<Workflow> & id: number\newline Partial<Workflow> & Aktualisiert das Observable mit dem Workflow mit der übergebenen ID mit dem zweiten übergebenen Parameter \\ \hline
					validate & boolean & workflow: Workflow & Überprüft die Syntax des übergebenen Workflows und gibt zurück ob er korrekt ist oder nicht \\ \hline
					start & void & id: number & Startet den Workflow mit der übergebenen ID \\ \hline
					stop & void & id: number & Stoppt den Workflow mit der übergebenen ID \\ \hline
					\end{tabularx}
			\end{center}
%%% !!!			
\newpage
			
			\subsubsection{ProcessService}
			
			Beschreibt die Schnittstelle für den Zugriff auf Prozesse.\newline
			
			Attribute:
			\begin{center}
			\setlength\tabcolsep{5pt}
				\renewcommand{\arraystretch}{1.5}
					\begin{tabularx}{\textwidth}{|l|l|p{30mm}|X|}
					\hline
					\rowcolor[gray]{0.75}[4.85pt]
					Name & Rückgabetyp & Parameter & Beschreibung \\ \hline 
					constructor & - & HTTPClient & Konstruktor \\ \hline
					all & Observable<Process[]> & - & Gibt ein Observable mit allen Prozessen zurück \\ \hline
					get & Observable<Process> & id: number & Gibt ein Observable mit dem Prozess mit der übergebenen ID zurück \\ \hline
					create & Observable<Process> & process: Process & Erzeugt ein Observable mit dem übergebenen Prozess \\ \hline
					remove & Promise<boolean>  & id: number & Entfernt den Prozess mit der übergebenen ID \\ \hline
					update & Observable<Process> & id: number\newline Partial<Process> & Aktualisiert das Observablemit dem Prozess mit der übergebenen ID mit dem zweiten übergebenen Parameter \\ \hline
				\end{tabularx}
			\end{center}
   
%%% !!!
\newpage
		\subsection{models}
			
			\subsubsection{Workflow (Interface)}
			
			Dieses Interface beschreibt die Struktur der Workflows.\newline
			
			Attribute:
			\begin{center}
				\renewcommand{\arraystretch}{1.5}
				\setlength\tabcolsep{5pt}
				\begin{tabularx}{\textwidth}{|l|l|X|}
					\hline
					\rowcolor[gray]{0.75}[4.85pt]					
					Name & Datentyp & Beschreibung \\ \hline
					edges & WorkflowEdge[] & Die Verbindungen zwischen Tasks im Workflow \\ \hline
					tasks & Task[] & Die Tasks im Workflow \\ \hline
					creator_id & number & Die ID des Workflow-Erstellers \\ \hline
					shared & boolean & Diese Variable speichert ob der Workflow öffentlich oder privat ist \\ \hline
					name & string & Der Name des Workflows \\ \hline
					id & number & Die ID des Workflows \\ \hline
					created_at & number & Das Datum an dem der Workflow erstellt wurde \\ \hline
					updated_at & number & Das Datum an dem der Workflow das letzte mal aktualisiert wurde \\ \hline
				\end{tabularx}
			\end{center}
				
			\subsubsection{Edge (Interface)}
			
			Dieses Interface beschreibt die Struktur der Kanten.\newline
			
			Attribute:
			\begin{center}
				\renewcommand{\arraystretch}{1.5}
				\setlength\tabcolsep{5pt}
				\begin{tabularx}{\textwidth}{|l|l|X|}
					\hline
					\rowcolor[gray]{0.75}[4.85pt]					
					Name & Datentyp & Beschreibung \\ \hline
					 id & number & Die ID der Kante \\ \hline
					 a & Task & Der Task am Anfang der Kante \\ \hline
					 b & Task & Der Task am Ende der Kante \\ \hline
					 input_id & number & Die ID des Input-Punktes mit dem die Kante verbunden ist \\ \hline
					 output_id & number & Die ID des Output-Punktes mit dem die Kante verbunden ist \\ \hline
				\end{tabularx}
			\end{center}
%%%!!!
\newpage

			\subsubsection{Task (Interface)}
			
			Dieses Interface beschreibt die Struktur der Tasks die im Workflow die Knoten darstellen.\newline
			
			Attribute:
			\begin{center}
				\renewcommand{\arraystretch}{1.5}
				\setlength\tabcolsep{5pt}
				\begin{tabularx}{\textwidth}{|l|l|X|}
					\hline
					\rowcolor[gray]{0.75}[4.85pt]					
					Name & Datentyp & Beschreibung \\ \hline
					id & number & Die ID des Tasks \\ \hline
					y & number & Die y-Koordinate des Tasks\\ \hline
					x & number & Die x-Koordinate des Tasks \\ \hline
					status & TaskState & Der Status des Tasks \\ \hline
					input_artefacts & Artefact <'input'> & Die Input-Parameter des Tasks \\ \hline
					process_id & number & Die ID des Prozesses der im Task ausgeführt wird \\ \hline
					output_artefacts & Artefact <'output'> & Die Output-Parameter des Tasks \\ \hline
					created_at & number & Das Datum an dem der Task erstellt wurde \\ \hline
					updated_at & number & Das Datum an dem der Task das letzte mal aktualisiert wurde \\ \hline
				\end{tabularx}
			\end{center}
				
			\subsubsection{Artefact<T> (Interface)}
			
			Dieses Interface beschreibt die Input/Output Daten. Die Klasse hat den generischen Typ T als role Attribut. Der Typ kann entweder Input oder Output sein.\newline
			
			Attribute:
			\begin{center}
				\renewcommand{\arraystretch}{1.5}
				\setlength\tabcolsep{5pt}
				\begin{tabularx}{\textwidth}{|l|l|X|}
					\hline
					\rowcolor[gray]{0.75}[4.85pt]					
					Name & Datentyp & Beschreibung \\ \hline
					parameter_id & number & Die ID des Input- oder Output-Knotens zu der die Artefact-Instanz gehört \\ \hline
					task_id & number & Die ID des Tasks zu dem die Artefact-Instanz gehört \\ \hline
					workflow_id & number & Die ID des Workflows zu dem die Artefact-Instanz gehört \\ \hline
					role & T & Die role gibt an ob die Artefact-Instanz als Input oder als Output verwendet wird \\ \hline
					format & string & Gibt das Format der Artefact-Instanz an \\ \hline
					data & string & Übergibt die Daten die in der Artefact-Instanz verwendet werden \\ \hline
					created_at & number & Das Datum an dem die Artefact-Instanz erstellt wurde \\ \hline
					updated_at & number & Das Datum an dem die Artefact-Instanz das letzte mal aktualisiert wurde\\ \hline
				\end{tabularx}
			\end{center}
			
			\subsubsection{TaskState (Enum)}
			
			Beschreibt die möglichen Werte des Attributs status der Task Klasse.\newline
			
			\begin{center}
				\renewcommand{\arraystretch}{1.5}
				\setlength\tabcolsep{5pt}
				\begin{tabularx}{\textwidth}{|l|X|}
					\hline
					\rowcolor[gray]{0.75}[4.85pt]
					Name & Beschreibung \\ \hline
					READY & Der Task ist bereit ausgeführt zu werden \\ \hline
					WAITING & In diesem Zustand wartet ein Task auf den Input eines noch nicht beendeten Tasks \\ \hline
					RUNNING & Der Task ist momentan in bearbeitung \\ \hline
					FINISHED & Der Task wurde erfolgreich ausgeführt \\ \hline
					FAILED & Bei der Ausführung des Tasks ist ein Fehler aufgetreten \\ \hline
					DEPRECATED & Der Task gehört zum Process, der von dem WPS-Server gelöscht wurde, und sollte deswegen aus dem Workflow rausgenommen bzw. ersetzt werden \\ \hline
				\end{tabularx}
			\end{center}
				
			\subsubsection{Process (Interface)}
			
			Dieses Interface beschreibt die Prozesse die jeweils in einem Task enthalten sind.\newline
			
			Attribute:
			\begin{center}
				\renewcommand{\arraystretch}{1.5}
				\setlength\tabcolsep{5pt}
				\begin{tabularx}{\textwidth}{|l|l|X|}
					\hline
					\rowcolor[gray]{0.75}[4.85pt]					
					Name & Datentyp & Beschreibung \\ \hline
					id & number & Die ID des Prozesses \\ \hline
					title & string & Der Titel des Prozesses \\ \hline
					abstract & string & Eine kurze Beschreibung des Prozesses \\ \hline
					identifier & string & Die ID des Prozesses die vom WPS-Server festgelegt wird \\ \hline
					inputs & ProcessParameter<'input'>[] & Der Typ der Input-Parameter des Prozesses \\ \hline
					outputs & ProcessParameter<'output'>[] & Der Typ der Output-Parameter des Prozesses \\ \hline
					wps_id & number & Die ID des WPS-Servers der den Prozess bereit stellt \\ \hline
					created_timestamp & number & Das Datum an dem der Prozess erstellt wurde \\ \hline
					updated_timestamp & number & Das Datum an dem der Prozess das letzte mal aktualisiert wurde \\ \hline
				\end{tabularx}
			\end{center}
				
			\subsubsection{ProcessParameter<T> (Interface)}
			
			Dieses generische Interface mit Typ T beschreibt die jeweiligen Input/Output Parameter des Prozesses. Die Daten werden dabei in einem für WPS gültigen Format angegeben.\newline
			
			Attribute:
			\begin{center}
				\renewcommand{\arraystretch}{1.5}
				\setlength\tabcolsep{5pt}
				\begin{tabularx}{\textwidth}{|l|l|X|}
					\hline
					\rowcolor[gray]{0.75}[4.85pt]					
					Name & Datentyp & Beschreibung \\ \hline
					id & number & Die ID des Prozessparameters \\ \hline
					role & T & Die role gibt an, ob der Parameter des als Input oder als Output verwendet wird \\ \hline
					parameter_type & ProcessParameterType & Gibt an von welchem Typ der Parameter ist \\ \hline
					title & string & Der Titel des Prozessparameters \\ \hline
					abstract & string & Eine kurze Beschreibung des Prozessparameters \\ \hline
					min_occurs & number & Gibt an wie oft dieser Prozessparameter mindestens vorkommen muss \\ \hline
					max_occurs & number & Gibt an wie oft dieser Prozessparameter maximal vorkommen darf \\ \hline
				\end{tabularx}
			\end{center}
				
			\subsubsection{ProcessParamenterType (Enum)}
			
			Diese Aufzählung beschreibt die möglichen Werte die die parameter_type Variable der ProcessParameter<T>-Klasse haben kann.\newline
			
			\begin{center}
				\renewcommand{\arraystretch}{1.5}
				\setlength\tabcolsep{5pt}
				\begin{tabularx}{\textwidth}{|l|X|}
					\hline
					\rowcolor[gray]{0.75}[4.85pt]
					Name & Beschreibung \\ \hline
					COMPLEX & Oft Rasterbilddateien oder Vektorgrafiken \\ \hline
					LITERAL & Der Parameter besteht aus Zeichen \\ \hline
					BOUNDING_BOX & Sind in OGC OWS Common definiert als zwei Koordinatenpaare \\ \hline
				\end{tabularx}
			\end{center}
				
			\subsubsection{WPS (Interface)}
			
			Dieses Interface beschreibt jeweils einen WPS-Server.\newline
			
			Attribute:
			\begin{center}
				\renewcommand{\arraystretch}{1.5}
				\setlength\tabcolsep{5pt}
				\begin{tabularx}{\textwidth}{|l|l|X|}
					\hline
					\rowcolor[gray]{0.75}[4.85pt]					
					Name & Datentyp & Beschreibung \\ \hline
					id & number & Die ID des WPS-Servers \\ \hline
					provider & WPSProvider & Der Betreiber des WPS-Servers \\ \hline
					abstract & string & Eine kurze Beschreibung des WPS-Servers \\ \hline
					title & string & Der Titel des WPS-Servers \\ \hline
				\end{tabularx}
			\end{center}
				
			\subsubsection{WPSProvider (Interface)}
			
			Dieses Interface beschreibt jeweils einen WPS-Anbieter zu einem WPS-Server. Es wird im WPS Interface als Attribut provider verwendet.\newline
			
			Attribute:
			\begin{center}
				\renewcommand{\arraystretch}{1.5}
				\setlength\tabcolsep{5pt}
				\begin{tabularx}{\textwidth}{|l|l|X|}
					\hline
					\rowcolor[gray]{0.75}[4.85pt]					
					Name & Datentyp & Beschreibung \\ \hline
					id & number & Die ID des Anbieters des WPS-Servers  \\ \hline
					name & string & Der Name Anbieters des WPS-Servers \\ \hline
					url & string & Die URL zum Anbieter des WPS-Servers \\ \hline
				\end{tabularx}
			\end{center}
				
			\subsubsection{User (Interface)}
			
			Dieses Interface beschreibt den Aufbau einer Nutzer-Instanz.\newline
			
			\begin{center}
				\renewcommand{\arraystretch}{1.5}
				\setlength\tabcolsep{5pt}
				\begin{tabularx}{\textwidth}{|l|l|X|}
					\hline
					\rowcolor[gray]{0.75}[4.85pt]					
					Name & Datentyp & Beschreibung \\ \hline
					id & number & Die ID des Nutzers \\ \hline
					created\_at & number & Das Datum an dem die Nutzer-Instanz erstellt wurde \\ \hline
					updated\_at & number & Das Datum an dem die Nutzer-Instanz das letzte mal aktualisiert wurde\\ \hline
					last\_workflow\_id & number & Die ID des Workflows den der Nutzer zuletzt im Editor geladen hatte \\ \hline
					email & string & Die E-Mail Adresse des Nutzers \\ \hline
					last\_name & string & Der Nachname des Nutzers\\ \hline
					first\_name & string & Der Vorname des Nutzers\\ \hline
					username & string & Der Nutzername des Nutzers\\ \hline
					group & UserGroup & Die UserGroup des Nutzers\\ \hline			
				\end{tabularx}
			\end{center}
				
			\subsubsection{UserGroup (Enum)}
			
			Diese Aufzählung beschreibt mögliche Werte für die Benutzergruppe einer Nutzer-Instanz.\newline
			
			\begin{center}
				\renewcommand{\arraystretch}{1.5}
				\setlength\tabcolsep{5pt}
				\begin{tabularx}{\textwidth}{|l|X|}
					\hline
					\rowcolor[gray]{0.75}[4.85pt]
					Name & Beschreibung \\ \hline
					REGULAR & Ein normaler Nutzer \\ \hline
					ADMIN & Ein Nutzer mit Admin-Rechten \\ \hline
				\end{tabularx}
			\end{center}
	
		\subsection{components}
		
			\subsubsection{OnInit (Interface)}
			
			Dieses Interface wird von den Komponentenklassen implementiert und bietet eine Standardschnittstelle für die Initialisierung.\newline
			
			Methoden:
			\begin{center}
			\setlength\tabcolsep{5pt}
				\renewcommand{\arraystretch}{1.5}
					\begin{tabularx}{\textwidth}{|l|l|l|X|}
					\hline
					\rowcolor[gray]{0.75}[4.85pt]
					Name & Rückgabetyp & Parameter & Beschreibung \\ \hline 
					ngOnInit & void & - & Initialisierung der Komponente \\ \hline
					\end{tabularx}
			\end{center}
		
			\subsubsection{AppComponent}
			
			Beschreibt die Seite der Anwendung. Dafür wird immer geprüft ob der Nutzer, der auf die Komponente zugreift, auch eingeloggt ist.\newline
			
				Attribute:
				\begin{center}
					\renewcommand{\arraystretch}{1.5}
					\setlength\tabcolsep{5pt}
					\begin{tabularx}{\textwidth}{|l|l|X|}
						\hline
						\rowcolor[gray]{0.75}[4.85pt]					
						Name & Datentyp & Beschreibung \\ \hline
						authService & AuthService & Der Authentifikationsservice über den der Nutzer auf die Anwendung zugreift \\ \hline
					\end{tabularx}
				\end{center}
				
				Methoden:
				\begin{center}
				\setlength\tabcolsep{5pt}
					\renewcommand{\arraystretch}{1.5}
						\begin{tabularx}{\textwidth}{|l|l|l|X|}
						\hline
						\rowcolor[gray]{0.75}[4.85pt]
						Name & Rückgabetyp & Parameter & Beschreibung \\ \hline 
						constructor & - & au :AuthService & Konstruktor \\ \hline
						ngOnInit & void & - & Initialisierung der Komponente \\ \hline
						\end{tabularx}
				\end{center}

%%% !!!
\newpage

			\subsubsection{EditorPageComponent}
			
			Beschreibt die Komponenten der Editor-Seite die nicht durch Workflow-Komponenten dargestellt werden.\newline
			
			Attribute:
			\begin{center}
				\renewcommand{\arraystretch}{1.5}
				\setlength\tabcolsep{5pt}
				\begin{tabularx}{\textwidth}{|l|l|X|}
					\hline
					\rowcolor[gray]{0.75}[4.85pt]					
					Name & Datentyp & Beschreibung \\ \hline
					editorComponent  & EditorPageComponent & Referenz zur Editor-Komponente \\ \hline
					processListComponent  & ProcessListComponent & Referenz zur ProcessList-Komponente \\ \hline
					showProcessList  & boolean & Ob die Prozess Liste angezeigt wird \\ \hline
					snapshots  & Workflow[] & Eine Liste geänderten Workflows für die UNDO Funktionalität \\ \hline
					onStart  & EventEmitter<Workflow> & Emittet beim Klick auf den Startknopf \\ \hline
					onStop  & EventEmitter<Workflow> & Emittet beim Klick auf den Stopknopf \\ \hline
					onFinish  & EventEmitter<Workflow> & Emittet beim Fertigstellen der Ausführung \\ \hline
					onUndo  & EventEmitter<[Workflow, Workflow]> & Emittet beim Klick auf den UNDO-Knopf \\ \hline
				\end{tabularx}
			\end{center}
			
			Methoden:
			\begin{center}
			\setlength\tabcolsep{5pt}
				\renewcommand{\arraystretch}{1.5}
					\begin{tabularx}{\textwidth}{|l|l|p{38mm}|X|}
					\hline
					\rowcolor[gray]{0.75}[4.85pt]
					Name & Rückgabetyp & Parameter & Beschreibung \\ \hline
					constructor& - & ws: WorkflowService\newline ps: ProcessService & Konstruktor \\ \hline
					ngOnInit & void & - & Initialisierung der Komponente \\ \hline
					load & void & workflow: Workflow & Läd den übergebenen Workflow \\ \hline
					save & void & - & Speichert den übergebenen Workflow \\ \hline
					start & void & - & Startet den übergebenen Workflow \\ \hline
					stop & void & - & Stoppt den übergebenen Workflow \\ \hline
					undo & void & steps = 1 & Macht die angegebene Zahl an letzten Aktionen rückgängig \\ \hline
					\end{tabularx}
			\end{center}
			
			\subsubsection{WorkflowPageComponent}
			
			Beschreibt die Darstellung der Workflows-Seite in der der Nutzer seine Workflows betrachten kann.\newline
			
			Attribute:
			\begin{center}
				\renewcommand{\arraystretch}{1.5}
				\setlength\tabcolsep{5pt}
				\begin{tabularx}{\textwidth}{|l|l|X|}
					\hline
					\rowcolor[gray]{0.75}[4.85pt]					
					Name & Datentyp & Beschreibung \\ \hline
					workflows & Observable<Workflow[]> & Aktuelle Workflows \\ \hline
					selected & number & Die Anzahl an aktuell markierten Elementen auf der Workflow-Seite \\ \hline
				\end{tabularx}
			\end{center}
				
			Methoden:
			\begin{center}
			\setlength\tabcolsep{5pt}
				\renewcommand{\arraystretch}{1.5}
					\begin{tabularx}{\textwidth}{|l|l|p{38mm}|X|}
					\hline
					\rowcolor[gray]{0.75}[4.85pt]
					Name & Rückgabetyp & Parameter & Beschreibung \\ \hline
					constructor & - & ws: WorkflowService & Konstruktor \\ \hline
					ngOnInit & void & - & Initialisierung der Komponente \\ \hline
					rename & void & id: number\newline name: string & Ändert den Namen des Elements auf der Workflow-Seite mit der übergebenen ID in den übergebenen Namen \\ \hline
					delete & void & id: number & Löscht das Workflow Element mit der übergebenen ID \\ \hline
					run & void & id: number & Führt den Workflow mit der übergebenen ID aus \\ \hline
					edit & void & id: number & Läd den Workflow mit der übergebenen ID im Workflow-Editor \\ \hline
					select & void & id: number & Markiert den Element auf der Workflow-Seite mit der übergebenen ID \\ \hline
					\end{tabularx}
			\end{center}

%%% !!!
\newpage

			\subsubsection{AdminSettingsPage}
			
			Beschreibt die Seite auf der der Admin seine Einstellungen ändern kann.\newline
			
				Attribute:
				\begin{center}
					\renewcommand{\arraystretch}{1.5}
					\setlength\tabcolsep{5pt}
					\begin{tabularx}{\textwidth}{|l|l|X|}
						\hline
						\rowcolor[gray]{0.75}[4.85pt]					
						Name & Datentyp & Beschreibung \\ \hline
						wps & Observable<WPS[]> & Ein Observable mit den aktuellsten WPS Objekte \\ \hline
						selected & number & Die Anzahl an aktuell markierten Elementen auf der Einstellungsseite \\ \hline
					\end{tabularx}
				\end{center}
				
				Methoden:
				\begin{center}
				\setlength\tabcolsep{5pt}
					\renewcommand{\arraystretch}{1.5}
						\begin{tabularx}{\textwidth}{|l|l|l|X|}
						\hline
						\rowcolor[gray]{0.75}[4.85pt]
						Name & Rückgabetyp & Parameter & Beschreibung \\ \hline 
						constructor & - & ws: WPSService & Konstruktor \\ \hline
						ngOnInit & void & - & Initialisierung der Komponente \\ \hline
						refreshAllWPS & void & - & Läd alle WPS-Server und deren Services neu \\ \hline
						addWPS & void & id: number & Fügt den WPS-Server mit der übergebenen ID hinzu \\ \hline
						removeWPS & void & id: number & Entfernt den WPS-Server mit der übergebenen ID \\ \hline
						\end{tabularx}
				\end{center}
			
			\subsubsection{EditorComponent}
			
			Beschreibt den Teil des Editors in dem der Workflow bearbeitet wird. EditorComponent kann auch eine Instanz von EditorPageComponent sein.\newline
			
				Attribute:
				\begin{center}
					\renewcommand{\arraystretch}{1.5}
					\setlength\tabcolsep{5pt}
					\begin{tabularx}{\textwidth}{|l|l|X|}
						\hline
						\rowcolor[gray]{0.75}[4.85pt]					
						Name & Datentyp & Beschreibung \\ \hline
						workflow & Workflow & Der Workflow der im aktuellen Editor geladen ist \\ \hline
						background & ElementRef & Der Hintergrund der aktuellen Editor-Seite \\ \hline
						selectedProcessIndex & number & Der in der Liste ausgewählte Prozess \\ \hline
					\end{tabularx}
				\end{center}
				\newpage
				Methoden:
				\begin{center}
				\setlength\tabcolsep{5pt}
					\renewcommand{\arraystretch}{1.5}
						\begin{tabularx}{\textwidth}{|l|l|p{35mm}|X|}
						\hline
						\rowcolor[gray]{0.75}[4.85pt]
						Name & Rückgabetyp & Parameter & Beschreibung \\ \hline
						constructor & - & el: ElementRef & Konstruktor \\ \hline
						ngOnInit & void & - & Initialisierung der Komponente \\ \hline
						getEdgeSVG & string & edge: Edge & SVG String Darstellung der eines Edges  \\ \hline
						dragOver & boolean & event: DragEvent & Realisiert, dass Prozesse per Drag hinzugefügt werden können \\ \hline
						dragStart & boolean & index: number\newline event: MouseEvent & Start des Drag and Drops \\ \hline
						dragMove & boolean & event: MouseEvent & Wird durchgehend während des Drag and Drops ausgeführt \\ \hline
						dragEnd & boolean & event: MouseEvent & Ende des Drag and Drops \\ \hline
						\end{tabularx}
				\end{center}
			
			\subsubsection{ProcessComponent}
			
			Beschreibt jeweils einen Prozess der auf der Editor-Seite als Knoten eines Workflows platziert werden kann und in der ProcessListComponent angezeigt wird.\newline
			
				Attribute:
				\begin{center}
					\renewcommand{\arraystretch}{1.5}
					\setlength\tabcolsep{5pt}
					\begin{tabularx}{\textwidth}{|l|l|X|}
						\hline
						\rowcolor[gray]{0.75}[4.85pt]					
						Name & Datentyp & Beschreibung \\ \hline
						process & Process & Der zu repräsentierende Prozess \\ \hline
						draggable & boolean & Ob ein Prozess per Drag and Drop verschoben werden kann. \\ \hline
					\end{tabularx}
				\end{center}
				
				Methoden:
				\begin{center}
				\setlength\tabcolsep{5pt}
					\renewcommand{\arraystretch}{1.5}
						\begin{tabularx}{\textwidth}{|l|l|p{40mm}|X|}
						\hline
						\rowcolor[gray]{0.75}[4.85pt]
						Name & Rückgabetyp & Parameter & Beschreibung \\ \hline
						constructor & - & - & Konstruktor \\ \hline
						ngOnInit & void & - & Initialisierung der Komponente \\ \hline
						getParameterColor & string & type:\newline ProcessParameterType & Gibt eine Hintergrundfarbe je nach Input/Output Types zurück \\ \hline
						dragStart & boolean & event:DragEvent & Wird beim Starten des Drag and Drops ausgeführt \\ \hline
						openDetailDialog & void & - & Öfften eine Detailansicht des Prozesses \\ \hline
						\end{tabularx}
				\end{center}
			
			\subsubsection{TaskComponent}
			
			Beschreibt jeweils einen Task der auf der Editor-Seite als Knoten eines Workflows platziert wurde.\newline
			
				Attribute:
				\begin{center}
					\renewcommand{\arraystretch}{1.5}
					\setlength\tabcolsep{5pt}
					\begin{tabularx}{\textwidth}{|l|l|X|}
						\hline
						\rowcolor[gray]{0.75}[4.85pt]					
						Name & Datentyp & Beschreibung \\ \hline
						task & Task & der zu repräsentierende Task \\ \hline
						onClickInput & EventEmitter<ProcessParameter<'input'> > & Emittet beim Klick auf ein Input  \\ \hline
						onClickOutput & EventEmitter<ProcessParameter<'output'> > & Emittet beim Klick auf ein Output \\ \hline
					\end{tabularx}
				\end{center}
				
				Methoden:
				\begin{center}
				\setlength\tabcolsep{5pt}
					\renewcommand{\arraystretch}{1.5}
						\begin{tabularx}{\textwidth}{|l|l|p{40mm}|X|}
						\hline
						\rowcolor[gray]{0.75}[4.85pt]
						Name & Rückgabetyp & Parameter & Beschreibung \\ \hline
						constructor & - & - & Konstruktor \\ \hline
						ngOnInit & void & - & Initialisierung der Komponente \\ \hline
						getParameterColor & string & type:\newline ProcessParameterType & Gibt eine Hintergrundfarbe je nach Input/Output Types zurück \\ \hline
						getStatusColor & string & type: TaskState & Gibt die Hintergrundfarbe für ein gewissen Task Zustand \\ \hline
						openDetailDialog & void & - & Öffnet ein Dialogfenster mit einen Detailansicht des Tasks \\ \hline
						getInputPosition & [number, number] & id: number & Gibt die X, Y Position der Input Artefakts \\ \hline
						getOutputPosition & [number, number] & id: number & Gibt die X, Y Position der Output Artefakts \\ \hline
						clickInput & void & - & Wird beim Klick auf ein Output Artefakt ausgeführt wird \\ \hline
						clickOutput & void & - & Wird beim Klick auf ein Input Artefakt ausgeführt wird \\ \hline
						\end{tabularx}
				\end{center}
			
			\subsubsection{ProcessListComponent}
			
			Beschreibt die Komponente im Editor die die Liste der Prozesse enthält.\newline
			
				Attribute:
				\begin{center}
					\renewcommand{\arraystretch}{1.5}
					\setlength\tabcolsep{5pt}
					\begin{tabularx}{\textwidth}{|l|l|X|}
						\hline
						\rowcolor[gray]{0.75}[4.85pt]					
						Name & Datentyp & Beschreibung \\ \hline
						processes & Process[] & Liste aller Prozesse \\ \hline
					\end{tabularx}
				\end{center}
				
				Methoden:
				\begin{center}
				\setlength\tabcolsep{5pt}
					\renewcommand{\arraystretch}{1.5}
						\begin{tabularx}{\textwidth}{|l|l|l|X|}
						\hline
						\rowcolor[gray]{0.75}[4.85pt]
						Name & Rückgabetyp & Parameter & Beschreibung \\ \hline 
						constructor & - & - & Konstruktor \\ \hline
						ngOnInit & void & - & Initialisierung der Komponente \\ \hline
						\end{tabularx}
				\end{center}
			
			\subsubsection{ArtefactDialogComponent}
			
			Beschreibt den Teil des Editors in dem die Inputs/Outputs stehen.\newline
			
				Attribute:
				\begin{center}
					\renewcommand{\arraystretch}{1.5}
					\setlength\tabcolsep{5pt}
					\begin{tabularx}{\textwidth}{|l|l|X|}
						\hline
						\rowcolor[gray]{0.75}[4.85pt]					
						Name & Datentyp & Beschreibung \\ \hline
						task & Task & Ausgewählter Task \\ \hline
						role & string & Input oder Output Artefakt \\ \hline
						id & number & ID des Artefakts \\ \hline
						representation & ArtefactRepresentationStrategy & Die ArtefactRepresentationStrategy des Komponente \\ \hline
					\end{tabularx}
				\end{center}
				
				Methoden:
				\begin{center}
				\setlength\tabcolsep{5pt}
					\renewcommand{\arraystretch}{1.5}
						\begin{tabularx}{\textwidth}{|l|l|p{35mm}|X|}
						\hline
						\rowcolor[gray]{0.75}[4.85pt]
						Name & Rückgabetyp & Parameter & Beschreibung \\ \hline
						constructor & - & role: string\newline id: number & Konstruktor \\ \hline
						ngOnInit & void & - & Initialisierung der Komponente \\ \hline
						\end{tabularx}
				\end{center}
			
			\subsubsection{ArtefactRepresentationStrategy (Interface)}
			
			Beschreibt eine Schnittstelle zur implementierung der Darstellung verschiedener Elemente.\newline
			
				Attribute:
				\begin{center}
					\renewcommand{\arraystretch}{1.5}
					\setlength\tabcolsep{5pt}
					\begin{tabularx}{\textwidth}{|l|l|X|}
						\hline
						\rowcolor[gray]{0.75}[4.85pt]					
						Name & Datentyp & Beschreibung \\ \hline
						format & string & Das zu erkennende Format für das Artefakt \\ \hline
						role & string & Input oder Output Artefakt\\ \hline
					\end{tabularx}
				\end{center}
				
				Methoden:
				\begin{center}
				\setlength\tabcolsep{5pt}
					\renewcommand{\arraystretch}{1.5}
						\begin{tabularx}{\textwidth}{|l|l|l|X|}
						\hline
						\rowcolor[gray]{0.75}[4.85pt]
						Name & Rückgabetyp & Parameter & Beschreibung \\ \hline
						represent & TemplateRef & - & Gibt eine Repräsentation des Artefaktes zurück \\ \hline
						\end{tabularx}
				\end{center}
			
			\subsubsection{ErrorRepresentation}
			
			Beschreibt die Darstellung von Fehlern.\newline
			
				Attribute:
				\begin{center}
					\renewcommand{\arraystretch}{1.5}
					\setlength\tabcolsep{5pt}
					\begin{tabularx}{\textwidth}{|l|l|X|}
						\hline
						\rowcolor[gray]{0.75}[4.85pt]					
						Name & Datentyp & Beschreibung \\ \hline
						format & string & Das zu erkennende format der Repräsentation \\ \hline
						role & string & Input oder Output \\ \hline
					\end{tabularx}
				\end{center}
				
				Methoden:
				\begin{center}
				\setlength\tabcolsep{5pt}
					\renewcommand{\arraystretch}{1.5}
						\begin{tabularx}{\textwidth}{|l|l|l|X|}
						\hline
						\rowcolor[gray]{0.75}[4.85pt]
						Name & Rückgabetyp & Parameter & Beschreibung \\ \hline
						represent & TemplateRef & - & Gibt ein Template für die Fehler-Repräsentation zurück \\ \hline
						\end{tabularx}
				\end{center}
			
			\subsubsection{JSONRepresentation}
			
			Beschreibt die Darstellung von Daten im JSON-Format.\newline
			
				Attribute:
				\begin{center}
					\renewcommand{\arraystretch}{1.5}
					\setlength\tabcolsep{5pt}
					\begin{tabularx}{\textwidth}{|l|l|X|}
						\hline
						\rowcolor[gray]{0.75}[4.85pt]					
						Name & Datentyp & Beschreibung \\ \hline
						format & string & Das zu erkennende Format für das Artefakt \\ \hline
						role & string & Input oder Output Artefakt \\ \hline
					\end{tabularx}
				\end{center}
				
				Methoden:
				\begin{center}
				\setlength\tabcolsep{5pt}
					\renewcommand{\arraystretch}{1.5}
						\begin{tabularx}{\textwidth}{|l|l|l|X|}
						\hline
						\rowcolor[gray]{0.75}[4.85pt]
						Name & Rückgabetyp & Parameter & Beschreibung \\ \hline
						represent & TemplateRef & - & Gibt ein Template für die JSON-Repräsentation zurück \\ \hline
						\end{tabularx}
				\end{center}
			
			\subsubsection{XMLRepresentation}
			
			Beschreibt die Darstellung von Daten im XML-Format.\newline
			
				Attribute:
				\begin{center}
					\renewcommand{\arraystretch}{1.5}
					\setlength\tabcolsep{5pt}
					\begin{tabularx}{\textwidth}{|l|l|X|}
						\hline
						\rowcolor[gray]{0.75}[4.85pt]					
						Name & Datentyp & Beschreibung \\ \hline
						format & string & Das zu erkennende Format für das Artefakt \\ \hline
						role & string & Input oder Output Artefakt \\ \hline
					\end{tabularx}
				\end{center}
				
				Methoden:
				\begin{center}
				\setlength\tabcolsep{5pt}
					\renewcommand{\arraystretch}{1.5}
						\begin{tabularx}{\textwidth}{|l|l|l|X|}
						\hline
						\rowcolor[gray]{0.75}[4.85pt]
						Name & Rückgabetyp & Parameter & Beschreibung \\ \hline
						represent & TemplateRef & - & Gibt ein Template für die XML-Repräsentation zurück \\ \hline
						\end{tabularx}
				\end{center}
			
			\subsubsection{ImageRepresentation}
			
			Beschreibt die Darstellung von Daten im Bild-Format.\newline
			
				Attribute:
				\begin{center}
					\renewcommand{\arraystretch}{1.5}
					\setlength\tabcolsep{5pt}
					\begin{tabularx}{\textwidth}{|l|l|X|}
						\hline
						\rowcolor[gray]{0.75}[4.85pt]					
						Name & Datentyp & Beschreibung \\ \hline
						format & string & Das zu erkennende Format für das Artefakt \\ \hline
						role & string & Input oder Output Artefakt \\ \hline
					\end{tabularx}
				\end{center}
				
				Methoden:
				\begin{center}
				\setlength\tabcolsep{5pt}
					\renewcommand{\arraystretch}{1.5}
						\begin{tabularx}{\textwidth}{|l|l|l|X|}
						\hline
						\rowcolor[gray]{0.75}[4.85pt]
						Name & Rückgabetyp & Parameter & Beschreibung \\ \hline
						represent & TemplateRef & - & Gibt ein Template für die Bild-Repräsentation zurück \\ \hline
						\end{tabularx}
				\end{center}
