\chapter{Klassen und Pakete}

    % table Muster Attribute
    %
%     \begin{center}
%     	\renewcommand{\arraystretch}{1.5}
%         \setlength\tabcolsep{5pt}
%     	\begin{tabularx}{\textwidth}{|l|l|X|}
%     		\hline
%             \rowcolor[gray]{0.75}[4.85pt]            		
%     	    Name & Datentyp & Beschreibung \\ \hline
%            
%           \hline            		
%     	\end{tabularx}
%     \end{center}


        % table Muster Methoden
% 		\begin{center}
% 		    \setlength\tabcolsep{5pt}
%         	\renewcommand{\arraystretch}{1.5}
%             	\begin{tabularx}{\textwidth}{|l|l|l|X|}
%             	\hline
%             	\rowcolor[gray]{0.75}[4.85pt]
%         		Name & Rückgabetyp & Parameter & Beschreibung \\ \hline 
%
%                \hline
%             	\end{tabularx}
% 		\end{center}

    \section{Server Klassen}

    \subsection{edu.kit.scc.pseworkflow.views}

		\subsubsection{WorkflowView}

        Attribute:
        \begin{center}
        	\renewcommand{\arraystretch}{1.5}
            \setlength\tabcolsep{5pt}
        	\begin{tabularx}{\textwidth}{|l|l|X|}
        		\hline
                \rowcolor[gray]{0.75}[4.85pt]            		
        	    Name & Datentyp & Beschreibung \\ \hline
        	    model & Workflow &  \\ \hline
        		template\_name & string & index.html \\ \hline				
        	\end{tabularx}
        \end{center}

    
        Methoden:
        \begin{center}
        	\setlength\tabcolsep{5pt}
        	\renewcommand{\arraystretch}{1.5}
        	
        	\begin{tabularx}{\textwidth}{|l|l|X|X|}
        		\hline
        		\rowcolor[gray]{0.75}[4.85pt]
        		Name & Rückgabetyp & Parameter & Beschreibung \\ \hline 
        		index & HTTPResponse & request: HTTPRequest & ? \\ \hline
        		get & HTTPResponse & request: HTTPRequest
        		id: int & Sucht in der Datenbank nach dem Workflow mit der übergebenen ID und gibt ihn im JSON-Format zurück \\ \hline
        		create & HTTPResponse & request: HTTPRequest & Erzeugt einen neuen Workflow und gibt seine ID zurück \\ \hline
        		update & HTTPResponse & request: HTTPRequest
        		id: int & Speichert die Daten in die Datenbank; Überschreibt die Standard Methode. In dieser Methode wird die Workflow enthaltende JSON Datei geparst und danach werden einzelne Elemente eines Workflows separat gespeichert. \\ \hline
        		delete & HTTPResponse & request: HTTPRequest
        		id: int & Löscht den Workflow mit der übergebenen ID \\ \hline
        		start & HTTPResponse & request: HTTPRequest
        		id: int & Führt den Workflow mit der übergebenen ID aus \\ \hline
        		stop & HTTPResponse & request: HTTPRequest
        		id: int & Stoppt die Ausführung des Workflows mit der übergebenen ID \\ \hline
        	\end{tabularx}
        \end{center}

		\subsubsection{EditorView}
		
		Methoden:
		\begin{center}
		    \setlength\tabcolsep{5pt}
        	\renewcommand{\arraystretch}{1.5}
            	\begin{tabularx}{\textwidth}{|l|l|l|X|}
            	\hline
            	\rowcolor[gray]{0.75}[4.85pt]
        		Name & Rückgabetyp & Parameter & Beschreibung \\ \hline 
                index & HTTPResponse & request: HTTPRequest & ? \\ \hline
            	\end{tabularx}
		\end{center}
		
		\subsubsection{UserView}
		
		Methoden:
		\begin{center}
		    \setlength\tabcolsep{5pt}
        	\renewcommand{\arraystretch}{1.5}
            	\begin{tabularx}{\textwidth}{|l|l|l|X|}
            	\hline
            	\rowcolor[gray]{0.75}[4.85pt]
        		Name & Rückgabetyp & Parameter & Beschreibung \\ \hline 
                index & HTTPResponse & request: HTTPRequest & ? \\ \hline
            	\end{tabularx}
		\end{center}
		
		\subsubsection{SettingsView}
		
		Methoden:
		\begin{center}
		    \setlength\tabcolsep{5pt}
        	\renewcommand{\arraystretch}{1.5}
            	\begin{tabularx}{\textwidth}{|l|l|l|X|}
            	\hline
            	\rowcolor[gray]{0.75}[4.85pt]
        		Name & Rückgabetyp & Parameter & Beschreibung \\ \hline 
                index & HTTPResponse & request: HTTPRequest & ? \\ \hline
                updateProcessList & void & - & ? \\ \hline
                addService & void & - & ? \\ \hline
                updateService & void & \thead{id: int\\service: JSON} & ? \\ \hline
            	\end{tabularx}
		\end{center}
		
		\subsubsection{ProcessView}
		
		Methoden:
		\begin{center}
		    \setlength\tabcolsep{5pt}
        	\renewcommand{\arraystretch}{1.5}
            	\begin{tabularx}{\textwidth}{|l|l|l|X|}
            	\hline
            	\rowcolor[gray]{0.75}[4.85pt]
        		Name & Rückgabetyp & Parameter & Beschreibung \\ \hline 
                index & HTTPResponse & request: HTTPRequest & ? \\ \hline
                updateProcessList & void & - & ? \\ \hline
                get & HTTPResponse & request: HTTPRequest & ? \\ \hline
                create & HTTPResponse & request: HTTPRequest & ? \\ \hline
                update& HTTPResponse & request: HTTPRequest & \\ \hline
                delete & HTTPResponse & request: HTTPRequest & \\ \hline
            	\end{tabularx}
		\end{center}
		
		\subsubsection{ServiceView}
		
		Methoden:
		\begin{center}
		    \setlength\tabcolsep{5pt}
        	\renewcommand{\arraystretch}{1.5}
            	\begin{tabularx}{\textwidth}{|l|l|l|X|}
            	\hline
            	\rowcolor[gray]{0.75}[4.85pt]
        		Name & Rückgabetyp & Parameter & Beschreibung \\ \hline 
                index & HTTPResponse & request: HTTPRequest & ? \\ \hline
                updateProcessList & void & - & ? \\ \hline
                get & HTTPResponse & request: HTTPRequest & ? \\ \hline
                create & HTTPResponse & request: HTTPRequest & ? \\ \hline
                update& HTTPResponse & request: HTTPRequest & \\ \hline
                delete & HTTPResponse & request: HTTPRequest & \\ \hline
                refresh & HTTPResponse & request: HTTPRequest & \\ \hline
            	\end{tabularx}
		\end{center}

    \subsection{edu.kit.scc.pseworkflow.cron}

%%% KLASSE CRON
		\subsubsection{Cron}

        Die Klasse erledigt Aufgaben, die ständig im Hintergrund laufen und nicht vom Benutzer gesteuert werden. Methoden in dieser Klasse werden in regelmäßigen Zeitabständen von cron  aufgerufen (z.B. einmal pro Minute).
        \newline\newline
        Methoden:
        \begin{center}
        	\setlength\tabcolsep{5pt}
        	\renewcommand{\arraystretch}{1.5}
        	
        	\begin{tabularx}{\textwidth}{|l|l|l|X|}
        		\hline
        		\rowcolor[gray]{0.90}[4.85pt]
        		Name & Rückgabetyp & Parameter & Beschreibung \\ \hline
        		check_runnig_tasks & void & keine & Fragt für jeden Laufenden Task (status = RUNNING) sein Status bei zugehörigem WPS-Server ab und, falls, der Task erfolgreich beendet wurde (status wurde auf FINISHED gesetzt), setzt ggf. status von nächsten Tasks in diesem Workflow auf READY \\ \hline
        		schedule_next_tasks & void & keine & Geht alle Tasks im Status READY durch und schickt sie an den zugehörigen WPS Server, falls er frei ist \\ \hline
        	\end{tabularx}
        \end{center}

%%% ENDE DER KLASSE CRON

    \subsection{edu.kit.scc.pseworkflow.models.workflows}
    
        \subsubsection{Session}
        
        Attribute:
		\begin{center}
        	\renewcommand{\arraystretch}{1.5}
            \setlength\tabcolsep{5pt}
        	\begin{tabularx}{\textwidth}{|l|l|X|}
        		\hline
                \rowcolor[gray]{0.75}[4.85pt]            		
        	    Name & Datentyp & Beschreibung \\ \hline
        	    id & int & Die ID der Session \\ \hline
        	    user & User & Der zugehörige User \\ \hline
        	    last_workflow & Workflow & Der Workflow der in der Session erstellt wurde \\ \hline
        	\end{tabularx}
        \end{center}
        
        \subsubsection{Workflow}
        
        Attribute:
		\begin{center}
        	\renewcommand{\arraystretch}{1.5}
            \setlength\tabcolsep{5pt}
        	\begin{tabularx}{\textwidth}{|l|l|X|}
        		\hline
                \rowcolor[gray]{0.75}[4.85pt]            		
        	    Name & Datentyp & Beschreibung \\ \hline
        	    id & int & Die ID des Workflows \\ \hline
        	    name & string & Der Name des Workflows \\ \hline
        	    description & string & Die Beschreibung des Workflows \\ \hline
        	    tasks & Task[] & Eine Liste der Tasks des Workflows \\ \hline
        	    percent_done & int & Prozent der bisher erledigten Tasks \\ \hline
        	    created_at & datetime & Erstellungsdatum \\ \hline
        	    update_at & datetime & Datum der letzten Bearbeitung\\ \hline
        	    creator & User & Ersteller des Workflows \\ \hline
        	    last_modifier & User & Der Nutzer der den Workflow zuletzt bearbeitet hat \\ \hline
        	\end{tabularx}
        \end{center}
		
		Methoden:
		\begin{center}
		    \setlength\tabcolsep{5pt}
        	\renewcommand{\arraystretch}{1.5}
            	\begin{tabularx}{\textwidth}{|l|l|l|X|}
            	\hline
            	\rowcolor[gray]{0.75}[4.85pt]
        		Name & Rückgabetyp & Parameter & Beschreibung \\ \hline 
                save & void & - & Speichert den Workflow in der Datenbank \\ \hline
            	\end{tabularx}
		\end{center}
    	
    	\subsubsection{Artefact}
    	
    	Attribute:
		\begin{center}
        	\renewcommand{\arraystretch}{1.5}
            \setlength\tabcolsep{5pt}
        	\begin{tabularx}{\textwidth}{|l|l|X|}
        		\hline
                \rowcolor[gray]{0.75}[4.85pt]            		
        	    Name & Datentyp & Beschreibung \\ \hline
        	    & & \\ \hline
        	    & & \\ \hline
        	    & & \\ \hline
        	    & & \\ \hline
        	    & & \\ \hline
        	    & & \\ \hline
        	    & & \\ \hline
        	    & & \\ \hline
        	\end{tabularx}
        \end{center}
    	
    	\subsubsection{InputArtefact}
    	
    	\subsubsection{OutputArtefact}

    \subsection{edu.kit.scc.pseworkflow.models.processes}
    
        \subsubsection{Process}
        
        
        
        \subsubsection{Service}
        
        \subsubsection{ServiceProvider}
        
        \subsubsection{InputOutput}
        
        \subsubsection{Input}
        
        \subsubsection{Output}
        
        \subsubsection{Datatype}

    \subsection{edu.kit.scc.pseworkflow.models.sessions}

%%% KLASSE WORKFLOW
        \subsection{edu.kit.scc.pseworkflow.models.workflows.Workflow}
	        Diese Klasse beschreibt einen Workflow. Ein Workflow enthält einen oder mehrere Tasks. \newline
	        Diese Klasse erbt von der Django Klasse namens \glqq Model\grqq .
                    
            Attribute:
            \begin{center}
            	\renewcommand{\arraystretch}{1.5}
	            \setlength\tabcolsep{5pt}
            	\begin{tabularx}{\textwidth}{|l|l|X|}
            		\hline
                    \rowcolor[gray]{0.75}[4.85pt]            		
            	    Name & Datentyp & Beschreibung \\ \hline
            		id & int & Primärschlüssel in der Datenbank. Wird von Django automatisch generiert \\ \hline
					identifier & string & Eindeutige Kennung des Workflows\\ \hline
					name & string & Anzeigetitel und Name des Workflows\\ \hline
					description & string & Beschreibungstext des Workflows\\ \hline				tasks & Task & Liste der Tasks, die zu dem Workflow gehören\\ \hline
					percent\_done & int & Fortschritt des Workflows in Prozent\\ \hline
					created\_at & Date & Erzeugungsdatum des Workflows \\ \hline
					updated\_at & Date & Datum der letzten Veränderung \\ \hline 
					creator & User & Benutzer, der den Workflow erstellt hat\\ \hline
					%Attribut & wf\_shared\_with & string & Liste der Benutzer, die diesen Workflow %ebenfalls sehen und bearbeiten dürfen; Komma getrennte Liste\\ \hline
					last\_modifier & User & Benutzer, der am letzten den Workflow verändert hat\\ \hline
            	\end{tabularx}
            \end{center}

        
             Methoden:
	        \begin{center}
	        	\setlength\tabcolsep{5pt}
	        	\renewcommand{\arraystretch}{1.5}
	        	
	        	\begin{tabularx}{\textwidth}{|l|l|l|X|}
	        		\hline
	        		\rowcolor[gray]{0.75}[4.85pt]
	        		Name & Rückgabetyp & Parameter & Beschreibung \\ \hline 
	        		save & void & keine & Speichert die Daten in die Datenbank; Überschreibt die Standard Methode. In dieser Methode wird die Workflow enthaltende JSON Datei geparst und danach werden einzelne Elemente eines Workflows separat gespeichert. \\ 
	        		\hline
	        	\end{tabularx}
	        \end{center}
%%% ENDE DER KLASSE WORKFLOW
%%% KLASSE TASK
        \subsection{edu.kit.scc.pseworkflow.models.Task}	
    		Beschreibung eines einzelnen Tasks innerhalb eines Workflows, die in der Benutzeroberfläche als Knoten des Graphes dargestellt sind. Task ist eine Instanz eines WPS Prozesses, der von dem WPS Server zur Verfügung gestellt wird. Es kann mehrere Tasks geben, die später als derselbe WPS Prozess ausgeführt werden. \newline
    		Diese Klasse erbt von der Django Klasse namens \glqq Model\grqq .
    		
    		Atttribute:
			\begin{center}
				\setlength\tabcolsep{5pt}
				\renewcommand{\arraystretch}{1.5}
				
				\begin{tabularx}{\textwidth}{|l|l|X|}
					\hline
					\rowcolor[gray]{0.75}[4.85pt]
					Name & Datentyp & Beschreibung \\ \hline 
	           		id & int & Primärschlüssel in der Datenbank. Wird von Django automatisch generiert \\ \hline
	           		process & Process & Der zugehörende WPS Prozess, der von dem WPS Server zur Verfügung gestellt wird. \\\hline
	           		x & int & X Koordinate des Tasks in der Benutzeroberfläche. Damit werden Tasks immer genau an der Stelle im Editor angezeigt, wo die gespeichert wurden\\ \hline
	           		y & int & Y Koordinate des Tasks in der Benutzeroberfläche\\ \hline
%TODO TODO TODO
	           		status & Status & Ausführungsstatus des Tasks. Siehe workflow.Status Klasse \\ \hline
	           		input\_artefacts & InputArtefact[] & Liste der Eingabedaten \\ \hline
	           		output\_artefacts & OutputArtefact[] & Liste der Endergebnissen \\ \hline
	           		title & string & Titel des Tasks \\ \hline
	           		abstract & string & Kurze Beschreibung des Tasks \\ \hline
	           		status\_url & string & Url des Tasks auf dem WPS Server. Dient der  regelmäßigen Aktualisierung des Ausführungsstatus \\ \hline
	           		started\_at & Date & Datum und die Uhrzeit des Ausführungstarts \\ \hline
				\end{tabularx}
			\end{center}
    			
			Methoden:
			\begin{center}
				\setlength\tabcolsep{5pt}
				\renewcommand{\arraystretch}{1.5}
				
				\begin{tabularx}{\textwidth}{|l|l|l|X|}
					\hline
					\rowcolor[gray]{0.75}[4.85pt]
					Name & Rückgabetyp & Parameter & Beschreibung \\ \hline
				    save & void & keine & Speichert den Task in der Datenbank \\ 
					\hline
				\end{tabularx}
			\end{center}			
%%% ENDE DER KLASSE TASK
%%% ENUMERATION EXECUTION STATUS
		\subsection{edu.kit.scc.pseworkflow.models.Status (Enum)}	
			Beschreibt die möglichen Ausführungsstati von Workflows und Tasks. \newline
			
			Atttribute:
			\begin{center}
				\setlength\tabcolsep{5pt}
				\renewcommand{\arraystretch}{1.5}
				
				\begin{tabularx}{\textwidth}{|l|l|X|}
					\hline
					\rowcolor[gray]{0.75}[4.85pt]
					Name & Rückgabetyp & Beschreibung \\ \hline 
	           		status\_value & string & Ausführungstatus\newline
	           		READY = Bereit zur Ausführung, wartet auf keine Inputs von anderen Tasks \newline
	           		RUNNING = Befindet sich gerade in der Ausführung auf einem WPS Server \newline 
	           		FINISHED = Workflow oder Task komplett ausgeführt und Ergebnis liegt vor \newline
	           		FAILED = Bei der Ausführung des Tasks ist ein Fehler aufgetreten \newline
	           		DEPRECATED = Der WPS Prozess steht aus eigenen Gründen nicht mehr auf dem WPS Server zur Verfügung \\
	           		\hline
				\end{tabularx}
			\end{center}
%%% ENDE DER ENUMERATION EXECUTION STATUS
%%% KLASSE EDGE
        \subsection{edu.kit.scc.pseworkflow.models.Edge}	
    			Diese Klasse beschreibt eine Kante, die in der Benutzeroberfläche zwei Taskknoten miteinander verbindet. \\newline
    			Diese Klasse erbt von der Django Klasse namens \glqq Model\grqq .
    			
    			Atttribute:
    			\begin{center}
    				\setlength\tabcolsep{5pt}
    				\renewcommand{\arraystretch}{1.5}
    				
    				\begin{tabularx}{\textwidth}{|l|l|X|}
    					\hline
    					\rowcolor[gray]{0.75}[4.85pt]
    					Name & Rückgabetyp & Beschreibung \\ \hline 
    	           		id & int & Primärschlüssel in der Datenbank. Wird von Django automatisch generiert \\ \hline
    	           		from\_task & Task & Ausgangsknoten \\ \hline
    	           	    to\_task & Task & Eingangsknoten \\ \hline
    	           		input & Input & Die Eingangsdaten eines Tasks\\ \hline
    	           		output & Output & Das Ergebnis eines Tasks \\
    	           		
    	           		\hline
    				\end{tabularx}
    			\end{center}
    			
    			Methoden:
    			\begin{center}
    				\setlength\tabcolsep{5pt}
    				\renewcommand{\arraystretch}{1.5}
    				
    				\begin{tabularx}{\textwidth}{|l|l|l|X|}
    					\hline
    					\rowcolor[gray]{0.75}[4.85pt]
    					Name & Rückgabetyp & Parameter & Beschreibung \\ \hline
    					save & void & keine & Speichert die Kante in der Datenbank\\ 
    					\hline
    				\end{tabularx}
    			\end{center}
    			
    			
    
%%% ENDE DER KLASSE EDGE
%%% KLASSE PROCESS
		\subsection{edu.kit.scc.pseworkflow.models.Process}
			Diese Klasse beschreibt den WPS Prozess, der auf dem WPS Server als Code gespeichert ist. Alle Daten werden von dem WPS Server als XML Datei mittels HTTP Response (Describe Processes) empfangen und danach auf dem Django Server erstmal geparst und in der Datenbank gespeichert. \newline
			Diese Klasse erbt von der Django Klasse namens \glqq Model\grqq .
			
			Atttribute:
			\begin{center}
				\setlength\tabcolsep{5pt}
				\renewcommand{\arraystretch}{1.5}
				
				\begin{tabularx}{\textwidth}{|l|l|X|}
					\hline
					\rowcolor[gray]{0.75}[4.85pt]
					Name & Rückgabetyp & Beschreibung \\ \hline 
					id & int & Primärschlüssel in der Datenbank. Wird von Django automatisch generiert \\ \hline
					identifier & String & Eindeutige Kennung des WPS Prozesses \\ \hline
					inputs & Input[] &  Liste der Inputs, die für den WPS Prozess erforderlich sind\\ \hline
					outputs & Output[] & Liste der Outputs, die für den WPS Prozess erforderlich sind \\ \hline
					service & Service & Beschreibt den WPS Server, der den WPS Prozess zur Verfügung stellt   \\ \hline
					title & string & Titel des WPS Prozesses \\ \hline
					abstract & string & Beschreibung des WPS Prozesses\\
					\hline
				\end{tabularx}
			\end{center}
			
			Methoden:
			\begin{center}
				\setlength\tabcolsep{5pt}
				\renewcommand{\arraystretch}{1.5}
				
				\begin{tabularx}{\textwidth}{|l|l|l|X|}
					\hline
					\rowcolor[gray]{0.75}[4.85pt]
					Name & Rückgabetyp & Parameter & Beschreibung \\ \hline 
					save & void & keine & Speichert den WPS Prozess in der Datenbank  \\
					\hline
				\end{tabularx}
			\end{center}
			
			
%%% ENDE DER KLASSE PROCESS 
%%% KLASSE SERVICE

        \subsection{edu.kit.scc.pseworkflow.models.Service}
			Diese Klasse beschreibt den WPS Server, der die WPS Prozesse zur Verfügung stellt. \newline
			Diese Klasse erbt von der Django Klasse namens \glqq Model\grqq .
			
			Atttribute:
			\begin{center}
				\setlength\tabcolsep{5pt}
				\renewcommand{\arraystretch}{1.5}
				
				\begin{tabularx}{\textwidth}{|l|l|X|}
					\hline
					\rowcolor[gray]{0.75}[4.85pt]
					Name & Rückgabetyp & Beschreibung \\ \hline 
					id & int & Primärschlüssel in der Datenbank. Wird von Django automatisch generiert \\ \hline
					service\_provider & ServiceProvider & Der Besitzer des WPS Servers \\ \hline
					title & string & Name des WPS Servers \\ \hline
					abstract & string & Beschreibung des WPS Servers \\ \hline
					capabilities\_url & string & URL des WPS Servers für die GetCapabilities Operation \\ \hline
					describe\_url & string & URL des WPS Servers für die DescribeProcess Operation \\ \hline
					execute\_url & string & URL des WPS Servers für die Execute Operation \\ \hline
					
				\end{tabularx}
			\end{center}
			
			Methoden:
			\begin{center}
				\setlength\tabcolsep{5pt}
				\renewcommand{\arraystretch}{1.5}
				
				\begin{tabularx}{\textwidth}{|l|l|l|X|}
					\hline
					\rowcolor[gray]{0.75}[4.85pt]
					Name & Rückgabetyp & Parameter & Beschreibung \\ \hline 
					save & void & keine & Speichert den Service in der Datenbank \\
					\hline
				\end{tabularx}
			\end{center}

%%% ENDE DER KLASSE SERVICE
%%% KLASSE SERVICE PROVIDER
        \subsection{edu.kit.scc.pseworkflow.models.ServiceProvider}
			Beschreibt den Besitzer den WPS Servers. Obwohl man ganz viele optionale Attribute auf dem WPS Server eingeben kann, werden nur die wichtigste von denen in der Datenbank gespeichert. \newline
			Diese Klasse erbt von der Django Klasse namens \glqq Model\grqq .
			
			Atttribute:
			\begin{center}
				\setlength\tabcolsep{5pt}
				\renewcommand{\arraystretch}{1.5}
				
				\begin{tabularx}{\textwidth}{|l|l|X|}
					\hline
					\rowcolor[gray]{0.75}[4.85pt]
					Name & Rückgabetyp & Beschreibung \\ \hline 
					id & int & Primärschlüssel in der Datenbank. Wird von Django automatisch generiert \\ \hline
					provider\_name & string & Name des WPS Server Providers\\ \hline
					provider\_site & string & Internetadresse des WPS Providers \\ \hline
					individual\_name & string & Name der Person\\ \hline
					position\_name & string & Stelle der Person \\ \hline
				\end{tabularx}
			\end{center}
			
			Methoden:
			\begin{center}
				\setlength\tabcolsep{5pt}
				\renewcommand{\arraystretch}{1.5}
				
				\begin{tabularx}{\textwidth}{|l|l|l|X|}
					\hline
					\rowcolor[gray]{0.75}[4.85pt]
					Name & Rückgabetyp & Parameter & Beschreibung \\ \hline 
					save & void & keine & Speichert den Service Provider in der Datenbank \\
					\hline
				\end{tabularx}
			\end{center}
			
%%% ENDE DER KLASSE SERVICE PROVIDER
%%% KLASSE INPUTOUTPUT
		\subsection{edu.kit.scc.pseworkflow.models.InputOutput}
			Beshreibt die Inputs und Outputs eines WPS Prozesses. Die Klasse ist abstakt, dass heißt, dass es keine Tabelle namens InputOutput von Django erzeugt wird. \newline
			Diese Klasse erbt von der Django Klasse namens \glqq Model\grqq .
			
			Atttribute:
			\begin{center}
				\setlength\tabcolsep{5pt}
				\renewcommand{\arraystretch}{1.5}
				
				\begin{tabularx}{\textwidth}{|l|l|X|}
					\hline
					\rowcolor[gray]{0.75}[4.85pt]
					Name & Rückgabetyp & Beschreibung \\ \hline 
					id & int & Primärschlüssel in der Datenbank. Wird von Django automatisch generiert \\ \hline
					process & Process & WPS Prozess, zu dem die Inputs bzw. Outputs gehören \\ \hline
					acts_as & enum & Stellt fest, ob die Instanz ein Input oder Output ist \\ \hline
					identifier & string & Eindeutige Kennung des Inputs oder Outputs \\ \hline
					title & string & Titel des Inputs oder Outputs \\ \hline
					abstract & string & Kurze Beschreibung des Inputs oder Outputs \\ \hline
					datatype & enum & Bestimmt den Datentyp des Inputs oder Outputs. 
%%%% TODO  TODO TODO
					Siehe .... \\ \hline
					min\_occurs & int & Minimales Vorkommen eines Inputs oder Outputs \\ \hline
					max\_occurs & int & Maximales Vorkommen eines Inputs oder Outputs \\ \hline
				\end{tabularx}
			\end{center}
			
%%% ENDE DER KLASSE INPUTOUTPUT
%%% KLASSE DATATYPE (Enumeration)
        \subsection{edu.kit.scc.pseworkflow.models.Status (Enum)}	
			Beschreibt die möglichen Datentypen des WPS Prozesses. \newline
			
			Atttribute:
			\begin{center}
				\setlength\tabcolsep{5pt}
				\renewcommand{\arraystretch}{1.5}
				
				\begin{tabularx}{\textwidth}{|l|l|X|}
					\hline
					\rowcolor[gray]{0.75}[4.85pt]
					Name & Rückgabetyp & Beschreibung \\ \hline 
	           		datatype & string & Datentyp des WPS Prozesses\newline
	           		LITERAL =  Literal Data ist ein String, normalerweise kurzer. Es wird verwendet um numerische oder textuelle Parameter zu übergeben.   \newline
	           		COMPLEX = Complex Data sind normalerweise die Raster- oder Vektorgrafiken, es können aber auch beliebige dateibasierte Daten sein\newline 
	           		BOUNDING_BOX = Koordinatenpaaren für 2D oder 3D Räume \\ \hline
				\end{tabularx}
			\end{center}

%%% ENDE DER KLASSE  DATATYPE
%%% KLASSE INPUT
    	\subsection{edu.kit.scc.pseworkflow.models.Input}
			Beschreibt den Input (Eingangsdaten) des WPS Prozesses. \newline
			Diese Klasse erbt die Attribute von der abstrakten Klasse InputOutput (siehe oben).
			
			Methoden:
			\begin{center}
				\setlength\tabcolsep{5pt}
				\renewcommand{\arraystretch}{1.5}
				
				\begin{tabularx}{\textwidth}{|l|l|l|X|}
					\hline
					\rowcolor[gray]{0.75}[4.85pt]
					Name & Rückgabetyp & Parameter & Beschreibung \\ \hline 
					save & void & keine &Speichert den Input eines WPS Prozesses in der Datenbank \\
					\hline
				\end{tabularx}
			\end{center}

%%% ENDE DER KLASSE INPUT
%%% KLASSE OUTPUT 
        \subsection{edu.kit.scc.pseworkflow.models.Ouput}
			Beschreibt den Output (Endergebniss) des WPS Prozesses. \newline
			Diese Klasse erbt die Attribute von der abstrakten Klasse InputOutput (siehe oben).
			
			Methoden:
			\begin{center}
				\setlength\tabcolsep{5pt}
				\renewcommand{\arraystretch}{1.5}
				
				\begin{tabularx}{\textwidth}{|l|l|l|X|}
					\hline
					\rowcolor[gray]{0.75}[4.85pt]
					Name & Rückgabetyp & Parameter & Beschreibung \\ \hline 
				    save & void & keine & Speichert den Input eines WPS Prozesses in der Datenbank \\
					\hline
				\end{tabularx}
			\end{center}
 
%%% ENDE DER KLASSE OUTPUT
%%% ENUMERATION ACTSAS
		\subsection{edu.kit.scc.pseworkflow.models.ActsAs (Enum)}	
			Beschreibt die möglichen Werte des Attributs acts\_as der abstrakten InputOutput Klasse. \newline
			
			Atttribute:
			\begin{center}
				\setlength\tabcolsep{5pt}
				\renewcommand{\arraystretch}{1.5}
				
				\begin{tabularx}{\textwidth}{|l|l|X|}
					\hline
					\rowcolor[gray]{0.75}[4.85pt]
					Name & Rückgabetyp & Beschreibung \\ \hline 
	           		status\_value & string & Ausführungstatus\newline
	           		INPUT = Eingabedaten eine WPS Prozesses \newline
	           		OUTPUT = Endergebniss eines WPS Prozesses \\ \hline
				\end{tabularx}
			\end{center}
%%% ENDE DER ENUMERATION ACTSAS
%%% KLASSE ARTEFACT
        \subsection{edu.kit.scc.pseworkflow.models.Artefact (Abstrakt)}
			Beschreibt die eigentlichen Eingangsdaten oder die Endegebnisse eines Tasks. Klasse ist abstrakt, das heißt, dass es keine Tabelle in der Datenbank namens \glqq Artefact \grqq erzeugt wird. \newline 
			Diese Klasse erbt von der Django Klasse namens \glqq Model\grqq .
			
			Atttribute:
			\begin{center}
				\setlength\tabcolsep{5pt}
				\renewcommand{\arraystretch}{1.5}
				
				\begin{tabularx}{\textwidth}{|l|l|X|}
					\hline
					\rowcolor[gray]{0.75}[4.85pt]
					Name & Rückgabetyp & Beschreibung \\ \hline 
					id & int & Primärschlüssel in der Datenbank. Wird von Django automatisch generiert \\ \hline
					task & Task & Der Task, zu dem die Inputs oder Outputs gehören\\ \hline
					parameter\_id & int & Identifikator des Parameters ???? \\ \hline
					acts_as & enum & Stellt fest, ob die Instanz ein Input oder Output ist \\ \hline
					format & string & Format des Inputs oder Outputs \\ \hline
					data & string & Der Wert dieses Attributs hängt von dem Datentyp ab. Falls der Datentyp Literal Data ist, dann wird es wirklich ein String mit dem Ergebniss gespeichert. Falls der Datentyp Complex Data oder Bounding Box Data ist, also falls das Ergebniss zu groß ist, um es in der Datenbank zu speichern, wird in diesem Attribut eine URL gespeichert, die zum Ergebniss auf dem WPS Server führt \\ \hline
					created\_at & datetime & Erstellungsdatum des Inputs/Outputs \\ \hline
					updated\_at & datetime & Datum und die Uhrzeit der letzten Änderung \\ \hline
				\end{tabularx}
			\end{center}
			
			Methoden:
			\begin{center}
				\setlength\tabcolsep{5pt}
				\renewcommand{\arraystretch}{1.5}
				
				\begin{tabularx}{\textwidth}{|l|l|l|X|}
					\hline
					\rowcolor[gray]{0.75}[4.85pt]
					Name & Rückgabetyp & Parameter & Beschreibung \\ \hline 
					save & void & keine & Speichert den Artefact in der Datenbank \\
					\hline
				\end{tabularx}
			\end{center}
%%% ENDE DER KLASSE ARTEFACT
%%% KLASSE INPUTARTEFACT
        \subsection{edu.kit.scc.pseworkflow.models.OuputArtefact}
			Beschreibt den Output (Endergebniss) des WPS Prozesses. \newline
			Diese Klasse erbt alle Attribute von der abstrakten Klasse Artefact (siehe oben).
			
			Methoden:
			\begin{center}
				\setlength\tabcolsep{5pt}
				\renewcommand{\arraystretch}{1.5}
				
				\begin{tabularx}{\textwidth}{|l|l|l|X|}
					\hline
					\rowcolor[gray]{0.75}[4.85pt]
					Name & Rückgabetyp & Parameter & Beschreibung \\ \hline 
				    save & void & keine & Speichert den InputArtefact in der Datenbank \\
					\hline
				\end{tabularx}
			\end{center}
%%%ENDE DER KLASSE INPUTARTEFACT
%%% KLASSE OUTPUTARTEFACT 
        \subsection{edu.kit.scc.pseworkflow.models.OuputArtefact}
			Beschreibt das Endergebniss des Tasks. \newline
			Diese Klasse erbt alle Attribute von der abstrakten Klasse Artefact (siehe oben).
			
			Methoden:
			\begin{center}
				\setlength\tabcolsep{5pt}
				\renewcommand{\arraystretch}{1.5}
				
				\begin{tabularx}{\textwidth}{|l|l|l|X|}
					\hline
					\rowcolor[gray]{0.75}[4.85pt]
					Name & Rückgabetyp & Parameter & Beschreibung \\ \hline 
				    save & void & keine & Speichert den OutputArtefact in der Datenbank \\
					\hline
				\end{tabularx}
			\end{center}
%%% ENDE DER KLASSE OUTPUTARTEFACT

















		\subsection{edu.kit.scc.pseworkflow.models.Session}
			[[[Bitte ausfüllen von denen, die Ahnung davon haben]]]
			
			Methoden:
			\begin{center}
				\setlength\tabcolsep{5pt}
				\renewcommand{\arraystretch}{1.5}
				
				\begin{tabularx}{\textwidth}{|l|l|l|X|}
					\hline
					\rowcolor[gray]{0.75}[4.85pt]
					Typ & Name & Rückgabetyp & Beschreibung \\ \hline 
					&&& \\
					\hline
				\end{tabularx}
			\end{center}
			
			Atttribute:
			\begin{center}
				\setlength\tabcolsep{5pt}
				\renewcommand{\arraystretch}{1.5}
				
				\begin{tabularx}{\textwidth}{|l|l|l|X|}
					\hline
					\rowcolor[gray]{0.75}[4.85pt]
					Typ & Name & Rückgabetyp & Beschreibung \\ \hline 
					Attribut & id & int & Primärschlüssel in Datenbank, wird von Django automatisch generiert \\ \hline
					Attribut & user\_id & int & bitte ausfüllen \\ \hline
					Attribut & last\_workflow\_id & int & bitte ausfüllen \\
					\hline
				\end{tabularx}
			\end{center}
		
		
%		\subsection{edu.kit.scc.pseworkflow.models.User}
%			Datenmodell eines Users.
%			
%			Methoden:
%			\begin{center}
%				\setlength\tabcolsep{5pt}
%				\renewcommand{\arraystretch}{1.5}
%				
%				\begin{tabularx}{\textwidth}{|l|l|l|l|X|}
%					\hline
%					\rowcolor[gray]{0.75}[4.85pt]
%					Typ & Name & Rückgabetyp & Beschreibung \\ \hline 
%					Methode & save & void & Speichert die Daten in die Datenbank; Überschreibt die Standard Methode \\ \hline
%					Methode & comparePwHash & bool & Vergleicht den Hashwert eines eingegebenen Passworts mit dem gespeicherten Hashwert des Benutzerpassworts \\
%					\hline
%				\end{tabularx}
%			\end{center}
%			
%			Atttribute:
%			\begin{center}
%				\setlength\tabcolsep{5pt}
%				\renewcommand{\arraystretch}{1.5}
%				
%				\begin{tabularx}{\textwidth}{|l|l|l|l|X|}
%					\hline
%					\rowcolor[gray]{0.75}[4.85pt]
%					Typ & Name & Rückgabetyp & Beschreibung \\ \hline 
%					Attribut & id & int & Primärschlüssel in Datenbank, wird von Django automatisch generiert \\ \hline
%					Attribut & user\_login & string & Loginname des Benutzers \\
%					Attribut & user\_pw & string & Hashwert des Benutzerpassworts\\ \hline
%					Attribut & user\_name & string & Nachname des Benutzers \\ \hline
%					Attribut & user\_prename & string & Vorname des Benutzers \\ \hline
%					Attribut & user\_mail & string & Mailadresse des Benutzers \\ \hline
%					Attribut & user\_level & string & Rechtelevel des Benutzers; admin = Adminuser mit allen Konfigurationsrechten, normal = Normaler Benutzer ohne weitere Konfigurationsrechte \\
%					\hline
%				\end{tabularx}
%			\end{center}
%		

\newpage

    \section{Client Klassen}
    
        \subsection{services}
    
        \subsection{models}
    
    		\subsubsection{Workflow}
    		
    		Attribute:
                \begin{center}
                	\renewcommand{\arraystretch}{1.5}
    	            \setlength\tabcolsep{5pt}
                	\begin{tabularx}{\textwidth}{|l|l|X|}
                		\hline
                        \rowcolor[gray]{0.75}[4.85pt]	
                	    Typ & Name & Beschreibung \\ \hline
                		number & id &  \\ \hline
                		string & name &  \\ \hline
                		boolean & shared &  \\ \hline
                		string & description &  \\ \hline
                		number & creator_id &  \\ \hline
                		WorkflowVertex[] & vertices &  \\ \hline
                		WorkflowEdge[] & edges &  \\ \hline
                		number & created_timestamp &  \\ \hline
                		number & updated_timestamp &  \\ \hline
                	\end{tabularx}
                \end{center}
                
    		\subsubsection{WorkflowVertex}
    		
    		Attribute:
                \begin{center}
                	\renewcommand{\arraystretch}{1.5}
    	            \setlength\tabcolsep{5pt}
                	\begin{tabularx}{\textwidth}{|l|l|X|}
                		\hline
                        \rowcolor[gray]{0.75}[4.85pt]
                	    Typ & Name & Beschreibung \\ \hline
                		number & id &  \\ \hline
                		number & process_id &  \\ \hline
                		number & x &  \\ \hline
                		number & y &  \\ \hline
                		WorkflowVertexStatus & status &  \\ \hline
                		WorkflowVertex[] & vertices &  \\ \hline
                		WorkflowEdge[] & edges &  \\ \hline
                		WorkflowArtefact<'input'> & input_artefacts &  \\ \hline
                		WorkflowArtefact<'output'> & output_artefacts &  \\ \hline
                	\end{tabularx}
                \end{center}
                
    		\subsubsection{WorkflowEdge}
    		
    		Attribute:
                \begin{center}
                	\renewcommand{\arraystretch}{1.5}
    	            \setlength\tabcolsep{5pt}
                	\begin{tabularx}{\textwidth}{|l|l|X|}
                		\hline
                        \rowcolor[gray]{0.75}[4.85pt]
                	    Typ & Name & Beschreibung \\ \hline
                		number & id &  \\ \hline
                		WorkflowVertex & a &  \\ \hline
                		WorkflowVertex & b &  \\ \hline
                		number & input_id &  \\ \hline
                		number & output_id &  \\ \hline
                	\end{tabularx}
                \end{center}
                
    		\subsubsection{WorkflowArtefact<T>}
    		
    		Attribute:
                \begin{center}
                	\renewcommand{\arraystretch}{1.5}
    	            \setlength\tabcolsep{5pt}
                	\begin{tabularx}{\textwidth}{|l|l|X|}
                		\hline
                        \rowcolor[gray]{0.75}[4.85pt]
                	    Typ & Name & Beschreibung \\ \hline
                		number & id &  \\ \hline
                		number & workflow_id &  \\ \hline
                		number & paramerer_id &  \\ \hline
                		T & parameter_role &  \\ \hline
                		string & format &  \\ \hline
                		string & data &  \\ \hline
                		number & created_timestamp &  \\ \hline
                		number & updated_timestamp &  \\ \hline
                	\end{tabularx}
                \end{center}
                
    		\subsubsection{WorkflowVertexState}
    		
    		ENUM
                \begin{center}
                	\renewcommand{\arraystretch}{1.5}
    	            \setlength\tabcolsep{5pt}
                	\begin{tabularx}{\textwidth}{|l|X|}
                		\hline
                        \rowcolor[gray]{0.75}[4.85pt]
                	    Name & Beschreibung \\ \hline
                		WAITING &   \\ \hline
                		RUNNING &   \\ \hline
                		FINISHED  &  \\ \hline
                		FAILED  &  \\ \hline
                		DEPRECATED  &  \\ \hline
                	\end{tabularx}
                \end{center}
                
    		\subsubsection{Process}
    		
    		Attribute:
                \begin{center}
                	\renewcommand{\arraystretch}{1.5}
    	            \setlength\tabcolsep{5pt}
                	\begin{tabularx}{\textwidth}{|l|l|X|}
                		\hline
                        \rowcolor[gray]{0.75}[4.85pt]
                	    Typ & Name & Beschreibung \\ \hline
                		number & id &  \\ \hline
                		string & identifier & \\ \hline
                		string & title &  \\ \hline
                		string & abstract &  \\ \hline
                        ProcessParameter<'input'>[] & inputs &  \\ \hline
                        ProcessParameter<'output'>[] & outputs &  \\ \hline
                		number & wps_id &  \\ \hline
                		number & created_timestamp &  \\ \hline
                		number & updated_timestamp &  \\ \hline
                	\end{tabularx}
                \end{center}
                
    		\subsubsection{ProcessParameter<T>}
    		
    		Attribute:
                \begin{center}
                	\renewcommand{\arraystretch}{1.5}
    	            \setlength\tabcolsep{5pt}
                	\begin{tabularx}{\textwidth}{|l|l|X|}
                		\hline
                        \rowcolor[gray]{0.75}[4.85pt]
                	    Typ & Name & Beschreibung \\ \hline
                		number & id &  \\ \hline
                		T & type &  \\ \hline
                		ProcessParamenterType & paramerer_type &  \\ \hline
                		string & title &  \\ \hline
                		string & abstract &  \\ \hline
                		number & min_occurs &  \\ \hline
                		number & max_occurs &  \\ \hline
                	\end{tabularx}
                \end{center}
                
    		\subsubsection{ProcessParamenterType}
    		
    		ENUM
                \begin{center}
                	\renewcommand{\arraystretch}{1.5}
    	            \setlength\tabcolsep{5pt}
                	\begin{tabularx}{\textwidth}{|l|X|}
                		\hline
                        \rowcolor[gray]{0.75}[4.85pt]
                	    Name & Beschreibung \\ \hline
                		COMPLEX &   \\ \hline
                		LITERAL &   \\ \hline
                		BOUNDING_BOX  &  \\ \hline
                	\end{tabularx}
                \end{center}
                
    		\subsubsection{WPS}
    		
    		Attribute:
                \begin{center}
                	\renewcommand{\arraystretch}{1.5}
    	            \setlength\tabcolsep{5pt}
                	\begin{tabularx}{\textwidth}{|l|l|X|}
                		\hline
                        \rowcolor[gray]{0.75}[4.85pt]
                	    Typ & Name & Beschreibung \\ \hline
                		number & id &  \\ \hline
                		string & abstract &  \\ \hline
                		WPSProvider & provider &  \\ \hline
                	\end{tabularx}
                \end{center}
                
    		\subsubsection{WPSProvider}
    		
    		Attribute:
                \begin{center}
                	\renewcommand{\arraystretch}{1.5}
    	            \setlength\tabcolsep{5pt}
                	\begin{tabularx}{\textwidth}{|l|l|X|}
                		\hline
                        \rowcolor[gray]{0.75}[4.85pt]
                	    Typ & Name & Beschreibung \\ \hline
                		number & id &  \\ \hline
                		string & name &  \\ \hline
                		string & site &  \\ \hline
                	\end{tabularx}
                \end{center}
                
    		\subsubsection{User}
    		
    		Attribute:
                \begin{center}
                	\renewcommand{\arraystretch}{1.5}
    	            \setlength\tabcolsep{5pt}
                	\begin{tabularx}{\textwidth}{|l|l|X|}
                		\hline
                        \rowcolor[gray]{0.75}[4.85pt]
                	    Typ & Name & Beschreibung \\ \hline
                		number & id &  \\ \hline
                		UserGroup & group &  \\ \hline
                		string & username &  \\ \hline
                		string & first_name &  \\ \hline
                		string & last_name &  \\ \hline
                		string & email &  \\ \hline
                		number & created_timestamp &  \\ \hline
                		number & updated_timestamp &  \\ \hline
                	\end{tabularx}
                \end{center}
                
    		\subsubsection{UserGroup}
    		
    		ENUM
                \begin{center}
                	\renewcommand{\arraystretch}{1.5}
    	            \setlength\tabcolsep{5pt}
                	\begin{tabularx}{\textwidth}{|l|X|}
                		\hline
                        \rowcolor[gray]{0.75}[4.85pt]
                	    Name & Beschreibung \\ \hline
                		REGULAR &  \\ \hline
                		ADMIN &  \\ \hline
                	\end{tabularx}
                \end{center}
    
        \section{components}
	
	\section{Entwurfsmuster}
	die verwendeten entwurfsmuster sollen ja auch irgendwo erwähnt werden.
	also die identifikation von entwurfsmustern um struktur gröber zu beschreiben
	zb iterator für die dynamische einbindung der implementierten task klassen
