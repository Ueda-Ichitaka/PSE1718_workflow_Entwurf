\chapter{Klassen und Pakete}

    \section{Pakete}
    
        \subsection{edu.kit.scc.pseworkflow.View}
        \subsection{edu.kit.scc.pseworkflow.Model}
        \subsection{edu.kit.scc.pseworkflow.Database}
        \subsection{edu.kit.scc.pseworkflow.Cron}
        \subsection{edu.kit.scc.pseworkflow.Utils}
        
    \section{Klassen}
%    
%        \subsection{edu.kit.scc.pseworkflow.View.Editor}
%        
%            Hier könnte ein erster beschreibender Text stehen
%            
%            Implementiert:
%            \begin{itemize}
%                \item ...
%            \end{itemize}
%            
%            Methoden:
%            \begin{center}
%	            \setlength\tabcolsep{5pt}
%	            \renewcommand{\arraystretch}{1.5}
%	            
%                \begin{tabularx}{\textwidth}{|l|l|l|l|X|}
%                    \hline
%                    \rowcolor[gray]{0.75}[4.85pt]
%                    Typ & Zugriff & Name & Rückgabetyp & Beschreibung \\ \hline 
%                    Method & public & exportToXml & void & Exports Workflow to XML File \\ \hline
%                    Property & public & exportPath & string & Gets or sets the export path bla bla bla line break test\\
%                    \hline
%                \end{tabularx}
%            \end{center}
%                
%            Attribute:
%            \begin{center}
%            	\renewcommand{\arraystretch}{1.5}
%	            \setlength\tabcolsep{5pt}
%            	\begin{tabularx}{\textwidth}{|l|l|l|l|X|}
%            		\hline
%                    \rowcolor[gray]{0.75}[4.85pt]            		
%            	    Typ & Zugriff & Name & Rückgabetyp & Beschreibung \\ \hline 
%            	   	&&&&\\
%            		\hline
%            	\end{tabularx}
%            \end{center}
%%            
%%       

		\subsection{edu.kit.scc.pseworkflow.models.ExecutionStatus}	
			Beschreibt die Möglichen Ausführungsstati von Workflows und Tasks.
			
			Methoden:
			\begin{center}
				\setlength\tabcolsep{5pt}
				\renewcommand{\arraystretch}{1.5}
				
				\begin{tabularx}{\textwidth}{|l|l|l|l|X|}
					\hline
					\rowcolor[gray]{0.75}[4.85pt]
					Typ & Zugriff & Name & Rückgabetyp & Beschreibung \\ \hline 
					Methode & public & save & void & Speichert die Daten in die Datenbank; Überschreibt die Standard Methode \\ 
					\hline
				\end{tabularx}
			\end{center}
			
			Atttribute:
			\begin{center}
				\setlength\tabcolsep{5pt}
				\renewcommand{\arraystretch}{1.5}
				
				\begin{tabularx}{\textwidth}{|l|l|l|l|X|}
					\hline
					\rowcolor[gray]{0.75}[4.85pt]
					Typ & Zugriff & Name & Rückgabetyp & Beschreibung \\ \hline 
	           		Attribut & public & id & int & Primärschlüssel in Datenbank, wird von Django automatisch generiert \\ \hline
	           		Attribut & public & status\_value & string & Ausführungstatus; READY = Bereit zur Ausführung, RUNNING = Wird ausgeführt, FINISHED = Workflow oder Task komplett ausgeführt und Ergebnis liegt vor, FAILED = Bei der Ausführung ist ein Fehler aufgetreten \\
	           		\hline
				\end{tabularx}
			\end{center}

		\subsection{edu.kit.scc.pseworkflow.models.DataType}
			Auflistung der möglichen Datentypen für Input und Output, die PyWPS bereitstellt.
			
			Methoden:
			\begin{center}
				\setlength\tabcolsep{5pt}
				\renewcommand{\arraystretch}{1.5}
				
				\begin{tabularx}{\textwidth}{|l|l|l|l|X|}
					\hline
					\rowcolor[gray]{0.75}[4.85pt]
					Typ & Zugriff & Name & Rückgabetyp & Beschreibung \\ \hline 
					Methode & public & save & void & Speichert die Daten in die Datenbank; Überschreibt die Standard Methode \\ 
					\hline
				\end{tabularx}
			\end{center}
			
			Atttribute:
			\begin{center}
				\setlength\tabcolsep{5pt}
				\renewcommand{\arraystretch}{1.5}
				
				\begin{tabularx}{\textwidth}{|l|l|l|l|X|}
					\hline
					\rowcolor[gray]{0.75}[4.85pt]
					Typ & Zugriff & Name & Rückgabetyp & Beschreibung \\ \hline 
					Attribut & public & id & int & Primärschlüssel in Datenbank, wird von Django automatisch generiert \\ \hline
					Attribut & public & type\_value & string & Datentyp aus PyWPS; LiteralData umfasst Text, Zahlen und boolsche Werte, ComplexData umfasst Vektordateien und BOundingBoxData sind Boxdaten/Rasterdaten [?] \\
					\hline
				\end{tabularx}
			\end{center}
		
		\subsection{edu.kit.scc.pseworkflow.models.ProcessHandler}
			Die Namen der von den PyWPS-Servern bereitgestellten Prozesse, entspricht ProcessIdentifier im PyWPS GML.
			
			Methoden:
			\begin{center}
				\setlength\tabcolsep{5pt}
				\renewcommand{\arraystretch}{1.5}
				
				\begin{tabularx}{\textwidth}{|l|l|l|l|X|}
					\hline
					\rowcolor[gray]{0.75}[4.85pt]
					Typ & Zugriff & Name & Rückgabetyp & Beschreibung \\ \hline 
					Methode & public & save & void & Speichert die Daten in die Datenbank; Überschreibt die Standard Methode \\ 
					\hline
				\end{tabularx}
			\end{center}
			
			Atttribute:
			\begin{center}
				\setlength\tabcolsep{5pt}
				\renewcommand{\arraystretch}{1.5}
				
				\begin{tabularx}{\textwidth}{|l|l|l|l|X|}
					\hline
					\rowcolor[gray]{0.75}[4.85pt]
					Typ & Zugriff & Name & Rückgabetyp & Beschreibung \\ \hline 
					Attribut & public & id & int & Primärschlüssel in Datenbank, wird von Django automatisch generiert \\ \hline
					Attribut & public & handler\_name & string & Name der Klasse der Funktion, die der PyWPS-Server zur Verfügung stellt; entspricht ProcessIdentifier im PyWPS GML \\
					\hline
				\end{tabularx}
			\end{center}
		
		\subsection{edu.kit.scc.pseworkflow.models.User}
			Datenmodell eines Users.
			
			Methoden:
			\begin{center}
				\setlength\tabcolsep{5pt}
				\renewcommand{\arraystretch}{1.5}
				
				\begin{tabularx}{\textwidth}{|l|l|l|l|X|}
					\hline
					\rowcolor[gray]{0.75}[4.85pt]
					Typ & Zugriff & Name & Rückgabetyp & Beschreibung \\ \hline 
					Methode & public & save & void & Speichert die Daten in die Datenbank; Überschreibt die Standard Methode \\ 
					Methode & public & comparePwHash & bool & Vergleicht den Hashwert eines eingegebenen Passworts mit dem gespeicherten Hashwert des Benutzerpassworts \\
					\hline
				\end{tabularx}
			\end{center}
			
			Atttribute:
			\begin{center}
				\setlength\tabcolsep{5pt}
				\renewcommand{\arraystretch}{1.5}
				
				\begin{tabularx}{\textwidth}{|l|l|l|l|X|}
					\hline
					\rowcolor[gray]{0.75}[4.85pt]
					Typ & Zugriff & Name & Rückgabetyp & Beschreibung \\ \hline 
					Attribut & public & id & int & Primärschlüssel in Datenbank, wird von Django automatisch generiert \\ \hline
					Attribut & public & user\_login & string & Loginname des Benutzers \\
					Attribut & public & user\_pw & string & Hashwert des Benutzerpassworts\\ \hline
					Attribut & public & user\_name & string & Nachname des Benutzers \\ \hline
					Attribut & public & user\_prename & string & Vorname des Benutzers \\ \hline
					Attribut & public & user\_mail & string & Mailadresse des Benutzers \\ \hline
					Attribut & public & user\_level & string & Rechtelevel des Benutzers; admin = Adminuser mit allen Konfigurationsrechten, normal = Normaler Benutzer ohne weitere Konfigurationsrechte \\
					\hline
				\end{tabularx}
			\end{center}
		
		
		\subsection{edu.kit.scc.pseworkflow.models.Task}
			Beschreibung eines einzelnen Tasks innerhalb eines Workflows.
			
			Methoden:
			\begin{center}
				\setlength\tabcolsep{5pt}
				\renewcommand{\arraystretch}{1.5}
				
				\begin{tabularx}{\textwidth}{|l|l|l|l|X|}
					\hline
					\rowcolor[gray]{0.75}[4.85pt]
					Typ & Zugriff & Name & Rückgabetyp & Beschreibung \\ \hline 
					Methode & public & save & void & Speichert die Daten in die Datenbank; Überschreibt die Standard Methode \\ 
					\hline
				\end{tabularx}
			\end{center}
			
			Atttribute:
			\begin{center}
				\setlength\tabcolsep{5pt}
				\renewcommand{\arraystretch}{1.5}
				
				\begin{tabularx}{\textwidth}{|l|l|l|l|X|}
					\hline
					\rowcolor[gray]{0.75}[4.85pt]
					Typ & Zugriff & Name & Rückgabetyp & Beschreibung \\ \hline 
					Attribut & public & id & int & Primärschlüssel in Datenbank, wird von Django automatisch generiert \\ \hline
					Attribut & public & workflow\_id & int & Fremdschlüssel; ID des Workflows, zu dem dieser Task gehört \\ \hline
					Attribut & public & task\_identifier & string & Eindeutige Kennung des Tasks \\ \hline
					Attribut & public & task\_title & string & Anzeigetitel und Name des Tasks \\ \hline
					Attribut & public & task\_abstract & string & Beschriebungstext des einzelnen Tasks \\ \hline
					Attribut & public & task\_status & string & Ausführungsstatus des Tasks \\ \hline
					Attribut & public & task\_handler & string & Name der PyWPS Klasse, die diesen Task ausführt \\ \hline
					Attribut & public & task\_percent\_done & int & Fortschritt des Tasks in Prozent \\
					\hline
				\end{tabularx}
			\end{center}
 
        \subsection{edu.kit.scc.pseworkflow.models.Workflow}
	        Diese Klasse beschreibt einen Workflow. Ein Workflow enthält einen oder mehrere Tasks.
	        
	        Methoden:
	        \begin{center}
	        	\setlength\tabcolsep{5pt}
	        	\renewcommand{\arraystretch}{1.5}
	        	
	        	\begin{tabularx}{\textwidth}{|l|l|l|l|X|}
	        		\hline
	        		\rowcolor[gray]{0.75}[4.85pt]
	        		Typ & Zugriff & Name & Rückgabetyp & Beschreibung \\ \hline 
	        		Methode & public & save & void & Speichert die Daten in die Datenbank; Überschreibt die Standard Methode \\ 
	        		\hline
	        	\end{tabularx}
	        \end{center}
                    
            Attribute:
            \begin{center}
            	\renewcommand{\arraystretch}{1.5}
	            \setlength\tabcolsep{5pt}
            	\begin{tabularx}{\textwidth}{|l|l|l|l|X|}
            		\hline
                    \rowcolor[gray]{0.75}[4.85pt]            		
            	    Typ & Zugriff & Name & Rückgabetyp & Beschreibung \\ \hline
            		Attribut & public & id & int & Primärschlüssel in Datenbank, wird von Django automatisch generiert \\ \hline
					Attribut & public & wf\_identifier & string & Eindeutige Kennung des Workflows\\ \hline
					Attribut & public & wf\_title & string & Anzeigetitel und Name des Workflows\\ \hline
					Attribut & public & wf\_abstract & string & Beschreibungstext des Workflows\\ \hline					
					Attribut & public & wf\_status & string & Ausführungsstatus des Workflows\\ \hline
					Attribut & public & wf\_owner & string & Besitzer und Bearbeiter des Workflows\\ \hline
					Attribut & public & wf\_shared\_with & string & Liste der Benutzer, die diesen Workflow ebenfalls sehen und bearbeiten dürfen; Komma getrennte Liste\\ \hline
					Attribut & public & wf\_num\_tasks & int & Anzahl der Tasks, die zu diesem Workflow gehören\\ \hline
					Attribut & public & wf\_percent\_done & int & Fortschritt des Workflows in Prozent\\ \hline
					Attribut & public & wf\_executable & boolean & Hilfsattribut für Automatischen Workflow Scheduler; Scheduler sendet den Workflow erst zur Ausführung, wenn wf\_executable = true; Standardwert false \\				
					\hline            		
            	\end{tabularx}
            \end{center}
            
            
         \subsection{edu.kit.scc.pseworkflow.models.Data}   
	         Datenmodellierung der Input- und Outputdaten.
	         
	         Methoden:
	         \begin{center}
	         	\setlength\tabcolsep{5pt}
	         	\renewcommand{\arraystretch}{1.5}
	         	
	         	\begin{tabularx}{\textwidth}{|l|l|l|l|X|}
	         		\hline
	         		\rowcolor[gray]{0.75}[4.85pt]
	         		Typ & Zugriff & Name & Rückgabetyp & Beschreibung \\ \hline 
	         		Methode & public & save & void & Speichert die Daten in die Datenbank; Überschreibt die Standard Methode \\ 
	         		\hline
	         	\end{tabularx}
	         \end{center}
	         
	         Atttribute:
	         \begin{center}
	         	\setlength\tabcolsep{5pt}
	         	\renewcommand{\arraystretch}{1.5}
	         	
	         	\begin{tabularx}{\textwidth}{|l|l|l|l|X|}
	         		\hline
	         		\rowcolor[gray]{0.75}[4.85pt]
	         		Typ & Zugriff & Name & Rückgabetyp & Beschreibung \\ \hline 
	         		Attribut & public & id & int & Primärschlüssel in Datenbank, wird von Django automatisch generiert \\ \hline
	         		Attribut & public & task\_id & int & ID des zugehörigen Tasks\\ \hline
	         		Attribut & public & data\_IO & string & Identifikationswert, ob es sich bei dem Dateneintrag um einen Input oder einen Output handelt\\ \hline
	         		Attribut & public & data\_identifier & string & Eindeutige Kennung des Dateninputs bzw -outputs\\ \hline
	         		Attribut & public & data\_title & string & Anzeigetitel und Name des Datenobjekts\\ \hline
	         		Attribut & public & data\_abstract & string & Beschreibungstext des Datenobjekts\\ \hline
	         		Attribut & public & data\_type & string & Datentyp des Datenobjekts; Wert aus DataType Klasse \\ \hline
	         		Attribut & public & data\_min\_occurs & int & Minimale Anzahl an Vorkommen des gleichen Inputs oder Outputs \\ \hline
	         		Attribut & public & data\_max\_occurs & int & Maximale Anzahl an Vorkommen des gleichen Inputs oder Outputs \\ \hline
	         		Attribut & public & data\_value & string & Wert des Inputs oder Outputs als String. Alle Werte müssen vor der Ausführung des Tasks oder jeglicher weiterer Weiterverarbeitung explizit von string auf den benötigen Datentyp gecastet werden\\
	         		\hline
	         	\end{tabularx}
	         \end{center}
            
            
%        \subsection{edu.kit.scc.pseworkflow.View.Overview}
%        \subsection{edu.kit.scc.pseworkflow.View.EditorView}
%        \subsection{edu.kit.scc.pseworkflow.View.OverviewView}
%                    
%        \subsection{edu.kit.scc.pseworkflow.   .Workflow}    
%        \subsection{edu.kit.scc.pseworkflow.   .Task}
%        \subsection{edu.kit.scc.pseworkflow.   .WorkflowElement}
%        \subsection{edu.kit.scc.pseworkflow.   .Edge}
%        \subsection{edu.kit.scc.pseworkflow.   .User}
%        
%        \subsection{edu.kit.scc.pseworkflow.   .Database}
%        \subsection{edu.kit.scc.pseworkflow.   .DatabaseWorker}
%        
%        \subsection{edu.kit.scc.pseworkflow.   .Session}
%        \subsection{edu.kit.scc.pseworkflow.   .Client}
%        \subsection{edu.kit.scc.pseworkflow.   .Server}
	
	\section{Entwurfsmuster}
	die verwendeten entwurfsmuster sollen ja auch irgendwo erwähnt werden.
	also die identifikation von entwurfsmustern um struktur gröber zu beschreiben
	zb iterator für die dynamische einbindung der implementierten task klassen
	        