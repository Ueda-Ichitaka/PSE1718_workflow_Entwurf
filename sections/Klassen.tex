\chapter{Klassen und Pakete}

    \section{Packages}
        \begin{itemize}
            \item edu.kit.scc.pseworkflow.View
            \item edu.kit.scc.pseworkflow.Model
            \item klassen und pakete können so aufgelistet werden
            \item oder wie unten im beispiel klassen
        \end{itemize}
        
    \section{Klassen}
    
        \subsection{edu.kit.scc.pseworkflow.View.Editor}
        
            Hier könnte ein erster beschreibender Text stehen
            
            Implementiert:
            \begin{itemize}
                \item ...
            \end{itemize}
            
            Methoden:
            \begin{center}
            \renewcommand{\arraystretch}{1.5}
                \begin{tabular}{|l|l|l|l|l|}
                    \hline 
                    \rowcolor[gray]{0.7}
                    Type & Access Modifier & Name & Return Type & Description \\ \hline 
                    Method & public & exportToXml & void & Exports Workflow to XML File \\ \hline
                    Property & public & exportPath & string & Gets or sets the export path \\
                    \hline
                \end{tabular}
            \end{center}
                
            Attribute:
            mit einer Tabelle wie oben \\
            ist die tabelle detailliert genug, wie es das tipps doc verlangt? es steht ja alles drin