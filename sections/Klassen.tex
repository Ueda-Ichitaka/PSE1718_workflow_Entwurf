\chapter{Klassen und Pakete}

    \section{Pakete}
    
        \subsection{edu.kit.scc.pseworkflow.View}
        \subsection{edu.kit.scc.pseworkflow.Model}
        \subsection{edu.kit.scc.pseworkflow.Database}
        \subsection{edu.kit.scc.pseworkflow.Cron}
        \subsection{edu.kit.scc.pseworkflow.Utils}
        
    \section{Klassen}
    
        \subsection{edu.kit.scc.pseworkflow.View.Editor}
        
            Hier könnte ein erster beschreibender Text stehen
            
            Implementiert:
            \begin{itemize}
                \item ...
            \end{itemize}
            
            Methoden:
            \begin{center}
	            \setlength\tabcolsep{5pt}
	            \renewcommand{\arraystretch}{1.5}
	            
                \begin{tabularx}{\textwidth}{|l|l|l|l|X|}
                    \hline
                    \rowcolor[gray]{0.75}[4.85pt]
                    Typ & Zugriff & Name & Rückgabetyp & Beschreibung \\ \hline 
                    Method & public & exportToXml & void & Exports Workflow to XML File \\ \hline
                    Property & public & exportPath & string & Gets or sets the export path bla bla bla line break test\\
                    \hline
                \end{tabularx}
            \end{center}
                
            Attribute:
            \begin{center}
            	\renewcommand{\arraystretch}{1.5}
	            \setlength\tabcolsep{5pt}
            	\begin{tabularx}{\textwidth}{|l|l|l|l|X|}
            		\hline
                    \rowcolor[gray]{0.75}[4.85pt]            		
            	    Typ & Zugriff & Name & Rückgabetyp & Beschreibung \\ \hline 
            	   	&&&&\\
            		\hline
            	\end{tabularx}
            \end{center}
%            
%        
%        \subsection{edu.kit.scc.pseworkflow.View.Overview}
%        \subsection{edu.kit.scc.pseworkflow.View.EditorView}
%        \subsection{edu.kit.scc.pseworkflow.View.OverviewView}
%                    
%        \subsection{edu.kit.scc.pseworkflow.   .Workflow}    
%        \subsection{edu.kit.scc.pseworkflow.   .Task}
%        \subsection{edu.kit.scc.pseworkflow.   .WorkflowElement}
%        \subsection{edu.kit.scc.pseworkflow.   .Edge}
%        \subsection{edu.kit.scc.pseworkflow.   .User}
%        
%        \subsection{edu.kit.scc.pseworkflow.   .Database}
%        \subsection{edu.kit.scc.pseworkflow.   .DatabaseWorker}
%        
%        \subsection{edu.kit.scc.pseworkflow.   .Session}
%        \subsection{edu.kit.scc.pseworkflow.   .Client}
%        \subsection{edu.kit.scc.pseworkflow.   .Server}
	
	\section{Entwurfsmuster}
	die verwendeten entwurfsmuster sollen ja auch irgendwo erwähnt werden.
	also die identifikation von entwurfsmustern um struktur gröber zu beschreiben
	zb iterator für die dynamische einbindung der implementierten task klassen
	        