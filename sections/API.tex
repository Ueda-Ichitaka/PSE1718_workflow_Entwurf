\chapter{URLs}
    \section{REST API}
    Der Server bietet eine \gls{REST} API für die Kommunikation zwischen Client und Server an. 
    Bei REST Anfragen werden dem Client die Daten als JSON geschickt und umgekehrt schickt der Client dem Server JSON Repräsentationen von der speichernden Daten. 
    
    \begin{center}
	    \setlength\tabcolsep{5pt}
    	\renewcommand{\arraystretch}{1.5}
        	\begin{tabularx}{\textwidth}{|l|l|l|X|}
        	\hline
        	\rowcolor[gray]{0.75}[4.85pt]
    		Method & URL & Handler & Beschreibung \\ \hline 
            GET & /user & UserView.index & Rückgabe des eingeloggten Benutzers \\ \hline
            \hline 
            GET & /workflow & WorkflowView.index & Rückgabe aller Workflows eines Benutzers \\ \hline
            POST & /workflow & WorkflowView.create & Erstellen eines neuen Workflows  \\ \hline
            GET & /workflow/:id & WorkflowView.get & Rückgabe eines einzelnen Workflows \\ \hline
            PATCH & /workflow/:id & WorkflowView.update & Bearbeiten eines Workflows \\ \hline
            DELETE & /workflow/:id & WorkflowView.delete & Löschen eines Workflows \\ \hline
            \hline 
            GET & /process & ProcessView.index & Rückgabe aller Prozesse \\ \hline
            POST & /process & ProcessView.create & Erstellen eines neuen Prozesses \\ \hline
            GET & /process/:id & ProcessView.get & Rückgabe eines einzelnen Prozesses \\ \hline
            PATCH & /process/:id & ProcessView.update & Bearbeiten eines Prozesses \\ \hline
            DELETE & /process/:id & ProcessView.delete & Löschen eines Prozesses \\ \hline
            \hline 
            GET & /wps & WPSView.index & Rückgabe aller WPS Services \\ \hline
            POST & /wps & WPSView.create & Erstellen eines neuen WPS Services \\ \hline
            GET & /wps/:id & WPSView.get & Rückgabe eines einzelnen WPS Services \\ \hline
            PATCH & /wps/:id & WPSView.update & Bearbeiten eines WPS Services \\ \hline
            DELETE & /wps/:id & WPSView.delete & Löschen eines WPS Services \\ \hline
        	\end{tabularx}
	\end{center}

%%% !!!
\newpage

	\section{URL Commands}
	Neben der REST API kann der Server auch bestimmte Commands ausführen. Diese Commands werden ausschließlich per GET Anfragen an
	den Server geschickt und verarbeitet.
	
	\begin{center}
	    \setlength\tabcolsep{5pt}
    	\renewcommand{\arraystretch}{1.5}
        	\begin{tabularx}{\textwidth}{|l|l|l|X|}
        	\hline
        	\rowcolor[gray]{0.75}[4.85pt]
    		Method & URL & Handler & Beschreibung \\ \hline 
            GET & /workflow_start/:id & WorkflowView.start & Starten eines Worflows \\ \hline
            GET & /workflow_stop/:id & WorkflowView.stop & Stoppen eines Worflows \\ \hline
            GET & /wps_refresh & WPSView.refresh & Aktualisieren der gespeicherten WPS Service Daten \\ \hline
        	\end{tabularx}
	\end{center}
	
	\section{URLs mit HTML Response}
	Da das Routing vom Client übernommen wird muss der Server bloß sicherstellen, dass die URLs zu den einzelnen Seiten existieren
	und die index.html liefern.
	
	\begin{center}
	    \setlength\tabcolsep{5pt}
    	\renewcommand{\arraystretch}{1.5}
        \begin{tabularx}{\textwidth}{|l|l|X|}
        	\hline
        	\rowcolor[gray]{0.75}[4.85pt]
    		Method & URL & Handler \\ \hline 
            GET & / & EditorView.index \\ \hline
            GET & /editor & EditorView.index \\ \hline
            GET & /workflows & Workflows.index \\ \hline
            GET & /settings & Settings.index \\ \hline
        \end{tabularx}
	\end{center}